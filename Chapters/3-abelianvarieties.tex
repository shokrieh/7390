% !TEX root = 7390.tex




\section{Abelian varieties}\label{Chapters/3-abelianvarieties}

\subsection{Definition and basic properties}

\begin{definition}
$X=V/\Lambda$ is called an \textbf{abelian variety} if it admits a positive definite line bundle, i.e. there exists $H\in NS(X)$ such that $H$ is positive definite. 
\end{definition}

Remark: We will see this agrees with our initial definition, i.e. such complex tori are ``algebraic/algebraizable''. (Hint: If $f:X\injects \PP^N$, then $f^*\mathcal{O}(1)$ must be positive definite.)

\begin{definition}
\noindent
\begin{enumerate}
\item A \textbf{polarization} on $X$ is $c_1(L)$ where $L$ is positive definite.
\item The \textbf{type} of a polarization is defined to be the type $(d_1,\cdots, d_g)$ of $L$ (or $c_1(L)$.)
\item A polarization is called \textbf{principal} if its type is $(1,\cdots, 1)$. 
\item For $(X,L)$ where $X$ is a complex torus and $L\in \Pic(X)$ is positive definite and $H=c_1(L)$, the pair $(X,H)$ is called a \textbf{polarized abelian variety} (P.A.V.).
\item A \textbf{homomorphism} of polarized abelian varieties is a map $f:(Y,M)\to (X,L)$ such that $f:Y\to X$ is a homomorphism of complex tori and $f^* c_1(L)=c_1(M)$. (We have seen that $c_1(f^*L)=f^* c_1(L)$.)
\end{enumerate}
\end{definition}

Remark: 
$f^*L\sim_{an} M$. This forces any homomorphism $f$ to have a finite kernel. (Otherwise $f^*L$ would be degenerate and so would $M$ be degenerate.) Conversely, if $f:Y\to X$ is a homomorphism of complex tori with finite kernel and $L\in \Pic(X)$ is positive definite, then $f^*L\in \Pic(Y)$ is positive definite. We call this the ``induced'' polarization. 

\begin{corollary}
\noindent
\begin{enumerate}
\item Any complex subtorus of an abelian variety is an abelian variety.
\item Let $X$ be an abelian variety and $Y$ be a complex torus. If $X$ is isogenous to $Y$, then $Y$ is also an abelian variety. 
\item For any abelian variety $X$, $\hat{X}=\Pic^0(X)$ is also an abelian variety.
\end{enumerate}
\end{corollary}

\begin{proof}
(3) follows from the fact that any non-degenerate $\varphi_L:X\to \hat{X}$ has finite kernel.
\end{proof}

\begin{lemma}
Every polarization is ``induced by'' a principal polarization via an isogeny. 
\end{lemma}

\begin{proof}
Let $(X,L)$ be a polarized abelian variety of type $(d_1,\cdots, d_g)$. Recall [insert label] we have an isogeny $p_1: X\to X_1$ where $X_1=V/(\Lambda_1\bigoplus \Lambda_2)$ and we have $M_1\in \Pic(X_1)$ with $P^* M_1\isom L$. Clearly $M_1$ is positive definite. 

We claim that $\text{type}(M_1)=(1,\cdots,1)$. We need Riemann-Roch to show this:
\[
d_1\cdots d_g = \chi(L)=(\deg p_1)\chi(M_1)=d_1\cdots d_g \chi(M_1)
\]
which implies that $\chi(M_1)=1=(-1)^s Pf(M_1)=Pf(M_1)$. 
\end{proof}

\begin{example}
When $X=\CC/\Lambda$ where $\Lambda=\ZZ \omega_1 \bigoplus \ZZ \omega_2\isom \ZZ^2$. (Without loss of generality, we may assume that $\omega_1=1, \omega_2 = \tau$ with $Im \tau>0$.) Define $H:\CC\times \CC\to \CC$ by 
$$(v,w)\mapsto \frac{v\overline{ w}}{Im(\overline{\omega_1}\omega_2)}.$$
$H$ is obviously positive definite.
It is easy to show that $H$ is Hermitian and that $Im(H(\Lambda,\Lambda))\subset \ZZ$. (That is, $H$ is a Riemann form.) Hence, $\CC/\Lambda$ is an abelian variety. 

(Remark: $\CC^g/\Lambda$ where $g>1$ is usually not an abelian variety.)
\end{example}

Recall the theorem of the square says that for $\bar{v}\bar{w}\in X$, we have
\[
t^*_{\bar{v}+\bar{w}}L \isom t^*_{\bar{v}}L \bigotimes t^*_{\bar{w}}L \bigotimes L^{-1}.
\]
Note that $t^*_{\bar{w}}L \bigotimes L^{-1}$ can be re-written as $t^*_xL\bigotimes t^*_{-x}L\isom L^2$, hence we have 
\[
t^*_{\bar{v}}L \bigotimes t^*_{\bar{w}}L \bigotimes t^*_{-\bar{v}-\bar{w}}L \isom L^3.
\]
This generalizes to the following lemma:

\begin{lemma}
For $\bar{v_1},\cdots,\bar{v_n}\in X$, if $\sum_{i=1}^n \bar{v_i}=0$, then $\bigotimes_{i=1}^n t^*_{v_i} L\isom L^n$.
\end{lemma}
\begin{proof}
Since $L=L(H,\chi)$, $\bigotimes_{i=1}^n t^*_{v_i} L
=L(nH,\prod \chi e^{2\pi i Im H(v_i,\cdot)})=L(nH,\chi^n)
\isom L^n$.
\end{proof}

Remarks:
\begin{enumerate}
\item I will give an interpretation in terms of divisors.
\item One can use this to give an explicit basis for $H^0(X,L^n)$ in terms of basis for $H^0(X,L)$. For instance, if $(X,L)$ is of type $(1,\cdots,1)$, then $H^0(X,L)$ is one dimensional.
\end{enumerate}

\subsection{Riemann relations}

We now turn to the following question: What does it mean, in terms of a period matrix $\Pi$, for $X$ to be an abelian variety?
Let $X=V/\Lambda$. Let $\{e_1,\cdots , e_g\}$ be a $\CC$-basis for $V$ and $\{\lambda_1,\cdots, \lambda_{2g}\}$ be a $\ZZ$-basis for $\Lambda$. Then we have 
$$X=\frac{\CC^g}{\Pi \ZZ^{2g}}$$
where $\Pi$ is the $g\times 2g$ period matrix. 

\begin{theorem}[Riemann relations]
$X$ is an abelian variety if and only if there exists a non-degenerate antisymmetric (i.e. alternating) matrix $A\in M_{2g\times 2g}(\ZZ)$ such that 
\begin{enumerate}
\item (Hermitian condition) $\Pi A^{-1} \Pi^t=0$; and
\item (positive definite condition) $i\Pi A^{-1}\overline{\Pi}^t>0$.
\end{enumerate}
\end{theorem}

\begin{proof}
We sketch the proof. Let $E:\Lambda\times \Lambda \to \ZZ$ be an arbitrary non-degenerate alternating form on $\Lambda$ (in $NS(X)$). Let $A$ be its matrix with respect to $\lambda_i$'s: $a_{ij}=E(\lambda_i,\lambda_j)$. Define $H:\CC^g\times \CC^g\to \CC$ by 
\[
H(u,v):=E(iu,v)+iE(u,v) \in \RR^{2g}
\]
where $E$ is the natural extension to $\RR^{2g}$. Define 
\[
\mathcal{J}=\binom{\Pi}{\overline{\Pi}}^{-1}_{2g\times 2g} \left(
\begin{array}{cc}
iIg & \ \\
\ & -i I_g\\
\end{array}
\right)\binom{\Pi}{\overline{\Pi}}.
\]
Such a construction is useful because $i\Pi = \Pi \mathcal{J}$. Note also that $E(\Pi x,\Pi y)=x^t A y$ for $x,y\in \ZZ^{2g}$. 

We claim that $A$ is Hermitian if and only if $\Pi A^{-1} \Pi^t=0$. We have seen that $H$ is Hermitian if and only if $E(iu,iv)=E(u,v)$ for all $u,v\in \CC^{g}\isom \RR^{2g}$. So $H$ is Hermitian if and only if $\mathcal{J}^t A \mathcal{J}=A$.

Next, we claim that assume $\Pi A^{-1} \Pi^t=0$, then the matrix of $H$ with respect to $\{e_1,\cdots,e_g\}$ is $2i(\Pi A^{-1} \overline{\Pi}^t)^{-1}$. Therefore, $H(u,v)=E(iu,v)+i E(u,v)$ is positive definite.
\end{proof}

Remark: If we assume that $\{\lambda_i\}$ is symplectic, and if we write $A=\left(
\begin{array}{cc}
0 & D\\
-D & 0 \\
\end{array}
\right)$
and $\Pi = ((\Pi_1)_{g\times g} | (\Pi_2)_{g\times g})$. Then the Riemann relations can be stated as 
\begin{enumerate}
\item $\Pi_2 D^{-1} \Pi_1^t-\Pi_1 D^{-1}\Pi_2^t=0$.
\item $i(\Pi_2 D^{-1}\overline{\Pi_1}^t-\Pi D^{-1}\overline{\Pi}^t)>0$. 
\end{enumerate}

\subsection{Divisors and Maps to $\PP^n$ for abelian varieties}
Let $X=V/\Lambda$ be an abelian variety of dimension $g$ with (positive definite) polarization $L\in \Pic(X)$. Define the rational (meromorphic?) map $\Psi_L: X\dashrightarrow \PP^N$ by
$$x\mapsto [\sigma_0(x):\cdots:\sigma_N(x)]$$
whenever there exists $i$ such that $\sigma_i(x)\not= 0$ where $\text{span}_\CC\{\sigma_0,\cdots,\sigma_N\}=H^0(X,L)$. 

By fixing a factor of automorphy, we may identify $H^0(X,L)$ with theta functions: fix a basis $\vartheta_0,\cdots,\vartheta_N$ for $H^0(X,L)$ and write $\Psi_L(\overline{v})=[\vartheta_0(v):\cdots:\vartheta_N(v)]$ where $\overline{v}=v$ modulo $\Lambda$. 

The big question is: when is $\Psi_L$ an embedding? Our goal is to show the following theorem.
\begin{theorem}[Lefschetz]
If $L$ is of the type $(d_1,\cdots,d_g)$ and $d_i\geq 3$, then $\Psi_L$ is an embedding.
\end{theorem}

\begin{example}
It is a good exercise to show that if $L$ is positive definite and of the type $(d_1,\cdots,d_g)$, then 
\begin{align*}
N+1 &=\dim H^0(X,L^{\otimes k})=\chi(L^{\otimes k})=Pf(L^{\otimes k})\\
&=Pf(k c_1(L))=(kd_1)(kd_2)\cdots (kd_g)=k^g(d_1d_2\cdots d_g).
\end{align*}

Now assume $L$ is principal (i.e. of type $(1,\cdots,1)$), so $\dim H^0(X,L^k)=k^g$. This gives that $N=k^g-1$. So for example when $k=3$, $N=3^g-1$. When $g=1$, we can embed elliptic curves inside $\PP^2$ which says that the quantity $N=k^g-1$ is sharp in some sense. 
\end{example}

If $L$ is principally polarized, then $\dim H^0(X,L)=1$, so there exists a unique (up to scaling) theta function $\theta$ generating $H^0(X,L)$, which is called a \emph{Riemann's theta function}. 

It is a good exercise to give $\vartheta_0,\cdots,\vartheta_N$ a basis for $H^0(X,L^{k})$ in terms of Riemann's theta function. (Hint: Recall that $L^{\otimes k}\isom \bigotimes_{i=1}^k t_{\bar{v}_1}^* L$ and $\sum_{i=1}^k \bar{v}_i=0$. So one needs to take appropriate shifts of $\theta$ (by, say $(\frac{1}{k}\Lambda/\Lambda)$) and appropriate products to get a basis.)

\subsection{Basic properties of divisors on Abelian Varieties}
we said that if $X$ is an abelian variety and $L$ is a positive-definite line bundle and $\vartheta_0,\cdots, \vartheta_N$ are a basis of theta-functions for $H^0(X,L)$, then we have a rational map $\Psi_L: X\dashrightarrow \PP^N$ by
$$\bar{p}\mapsto [\vartheta_0(p):\cdots:\vartheta_N(p)]$$. We want to study when this is defined everywhere.

\begin{lemma}
Let $X$ be a complex torus.
Suppose $\bar{v_1},\cdots,\bar{v_n}\in X$ such that $$\sum \bar{v_i}=0.$$ Suppose $D\in |L|$ is an effective divisor, then 
$$\sum_{i=1}^n t_{\bar{v_i}}^* D\sim nD.$$
\end{lemma}

\begin{proof}
$\bigotimes_{i=1}^n t_{\bar{v_i}}^* L \isom L^{\otimes n}$.
\end{proof}

\begin{remark}
\noindent
\begin{enumerate}
\item $t_x^* D = D-x$.
\item $y\in t_x^*D \Leftrightarrow x+y\in D\Leftrightarrow x\in t_y^*D$.  
\end{enumerate}
\end{remark}

\begin{proposition}
Let $X$ be a complex torus. Let $L$ be a positive definite line bundle of type $(d_1,\cdots, d_g)$. If $d_1\geq 2$, then $\Psi_L$ is holomorphic (i.e. $|L|$ is base-point free). 
\end{proposition}

\begin{example}
If $L$ is principal, then $L^2$ is base point free.
\end{example}

\begin{remark}
Why do we have $2$ in the above example? This is because of the Theorem of the square!
\end{remark}

\begin{remark}
Recall that $L\isom L_1^n$ for some $L_1$ if and only if $X[n]\subset K(L)=(\bigotimes \ZZ/d_i\ZZ)^2$ where $K(L)=\ker(\varphi_L:X\to \hat{X})$. It follows that $L=L_1^{d_1}$ where $d_1\geq 2$. 
\end{remark}

Now we will give the proof of the above proposition.
\begin{proof}
(In the proof, we use the notation $u\in D$ to denote $u\in Supp(D)$.)
Let $x\in X$. We need to show that there exists $E\in |L|$ not containing $x$ (i.e. $x\not\in Supp(E)$).  
By the above remark, $L=M^{d_1}$ for some $M$. Then $M$ is also positive definite (because $L$ is). So there exists $D\in |M|$. (Note that $\dim H^0(X,M)\geq 1$ and so $|D|\not=\emptyset$.) Consider $t_x^* D=D-x$. We choose $x_1,\cdots, x_{d_1-1}\not\in t_x^* D$. Let $x_{d_1}=-\sum_{i=1}^{d_1-1} x_i$ be such that $x_{d_1}\not\in t_x^*D$ by continuity of summation. 
Therefore, $x\not\in t_{x_i}^* D$ for all $1\leq i \leq d_1$. So $x\not\in \sum_{i=1}^{d_1} t_{x_i}^* D$ and hence $x\not\in \cup Supp(t_{x_i}^* D)$. But $\sum_{i=1}^{d_1} t_{x_i}^* D \sim d_1D\in |L|$. 

\end{proof}

\begin{definition}
$D=\sum a_i Y_i$ is called \textbf{reduced} if all nonzero $a_i=1$.
\end{definition}

\begin{lemma}
Let $L\in \Pic(X)$ where $X$ is a complex torus. Suppose $|L|\not=\emptyset$ and $L$ is positive definite (i.e. $L$ is positive semi-definite with $L|_{K(L)_0}$ trivial). A ``general member'' of $|L|$ is reduced. More precisely, if $D=nE+F\in |L|$ with $E>0,F\geq 0$ and $n\geq 2$, then $D$ is reduced.
\end{lemma}

\begin{proof}
Assume $D=nE+F\in |L|$ with $E>0, F\geq 0$ and $n\geq 2$. But $nE\sim \sum_{i=1}^n t_{x_i}^* E$ for $\sum x_i=0$. Since we may pick $x_1,\cdots, x_{n-1}$ arbitrarily, we are done.
\end{proof}

\begin{lemma}
Suppose $L\in\Pic(X)$ is positive definite, then there exists an open dense $\mathcal{U}\subset |L|\isom \PP^N$ such that if $D\in \mathcal{U}$, then $t_x^* D=D$ only holds for $x=0$.  
\end{lemma}

\begin{remark}
Let $D$ be an effective divisor. Define $H(D)=\{x\in X:\ t_x^*D=D\}$. The above lemma says that for ``almost all'' $D$, we have $H(D)=\{0\}$. Also, we have $H(D)$ is Zariski closed. Moreover, if $L=\mathcal{L}(D)=\mathcal{O}_X(L)$, then the following are equivalent:
\begin{enumerate}
\item $L$ is positive definite.
\item $K(L)$ is finite.
\item $H(D)$ is finite.
\end{enumerate}
\end{remark}

The proof of the lemma uses $K(L)$ is finite: if there exists $x\not=0$ such that $t_x^* D = D$, then $x\in K(L)$ and $G:=\langle x\rangle$ is finite. Then the idea is to consider the projection map $X\to X/G$ which is of degree $|G|\geq 2$ and we will omit the rest of the proof.

\subsection{Decomposition of Polarized abelian variety}
Let $(X,L)$ be a polarized abelian variety. We want to decompose it to ``irreducible'' polarized abelian varieties. One benefit of doing so is for studying $X\to \PP^N$. (If $d_1=2$, then $\Psi_L$ is sometimes an embedding. If it is not an embedding, then on irreducibles $\Psi_L$ factors through the \emph{Kummer} variety $X/\langle -1\rangle$.) Let $|L|=|M|+F_1+\cdots+F_r$ where $|M|$ is the ``moving part'' and $F_1+\cdots+F_r$ is the ``fixed component''. Since generating elements are reduced ($F_i\not=F_j$ for $i\not=j$), one can consider $M$ and $N_i=\mathcal{L}(F_i)$ for $1\leq i\leq r$ and therefore we have $h^0(X,M)>1$ and $h^0(X,N_i)=1$. So we have $M|_{K(M)_0}$ is trivial and $N_i|_{K(N_i)_0}$ is trivial and hence $P_M:X\to X_M:=X/K(M)_0$ and $P_{N_i}:X\to X_{N_i}:=X/K(N_i)_0$ and so there exists descents $\overline{M}$ on $X_M$ and $\overline{N_i}$ on $X_{N_i}$. So we get polarized abelian varieties $(X_M,\overline{M}),(X_{N_i},\overline{N_i})_{1\leq i \leq r}$. 

\begin{theorem}
The homomorphism 
$$P=(P_M,P_{N_1},\cdots,P_{N_r}):X\to X_m\times X_{N_1}\times\cdots \times X_{N_r}$$
induces an isomorphism of polarized abelian varieties
$$P:(X,L)\to (X_M\times X_{N_1}\times \cdots \times X_{N_r}, q_M^* \overline{M}\bigotimes \cdots \bigotimes q_{N_r}^* \overline{N_r})$$
which we informally write 
$$(X,L)\isom (X_M,\overline{M})\times \cdots \times (X_{N_r},\overline{N_r}).$$
Here $q_M: X_M\times X_{N_1}\times \cdots \times X_{N_r} \rightarrow X_M$ and $q_{N_i}: X_{M}\times X_{N_1}\times \cdots \times X_{N_r} \rightarrow X_{N_i}$ denote the natural projection maps.
\end{theorem}

For the proof, one uses new inequalities for $(L_1,\cdots, L_g)$, Riemann Roch, and other tools that we have seen so far. Since this is not used in Lefschetz theorem, I will skip the proof.

\subsection{Gauss map}

Let $(X,L)$ be a polarized abelian variety of dimension $g$. Take $D\in |L|$ such that $D$ reduced (the theorem of squares guarantees existence; in fact, one can pick any general member). Let $D_{sm}$ be the smooth locus of $supp(D)$ which has dimension $g-1$. For $\bar{\omega}\in D_{sm}$, the tangent $T_{D,\bar{\omega}}$ to $D$ at $\bar{\omega}$ is a $(g-1)$-dimensional $\CC$-vector space. Let $W_{\bar{\omega}}$ be the translation of $T_{D,\bar{\omega}}$ (via $t_{-\bar{\omega}}$) to the origin; it is a $(g-1)$-dimensional subspace of the $g$-dimensional tangent space $T_{x,0}=V$ at $0$. Then we define the \textbf{Gauss map} $G:D_{sm}\to \PP(V^*)$ by 
\[
\bar{\omega}\mapsto W_{\bar{\omega}}^\perp.
\]

Explicitly, we can identify $H^0(X,L)$ with the space of theta functions. Then the covering map $\pi:V\to X=V/\Lambda$ satisfies 
\[
\pi^*D=div(\vartheta)
\]
for some holomorphic function $\vartheta$ on the universal cover. (Note that $div(\vartheta)$ is periodic with respect to $\Lambda$ because $\vartheta(v+\lambda)=a_L(\lambda,v)\vartheta(v)$. So the ``zero set'' is periodic.) Let $\{v_1,\cdots,v_g\}$ be coordinate functions with respect to some basis. Let $\bar{\omega}=\omega+\Lambda$. Then 
\[
T_{D,\bar{\omega}}=\{(v_1,\cdots,v_g):\ \sum_{i=1}^g \frac{\partial \vartheta}{\partial v_i}(\omega)(v_i-\omega_i)=0\}.
\]
So the dual to $T_{D,\bar{\omega}}$ is $\left(\frac{\partial \vartheta}{\partial v_i}(\omega)\right)_{1\leq i\leq g}$. So the Gauss map maybe identified with $G:D_{sm}\to \PP^{g-1}$ defined by 
\[
\bar{\omega}\mapsto \left(\frac{\partial \vartheta}{\partial v_i}(\omega)\right)_{1\leq i\leq g}.
\]

It is easy to check that $G$ is holomorphic and independent of the choice of factors of automorphy. 

\begin{proposition}
Let $L$ be positive definite and $D\in |L|$ be reduced. Then $Im(G:D_{sm}\to \PP(V^*)\isom \PP^{g-1}$ is \emph{not} contained in a hyperplane. 
\end{proposition}

\begin{proof}
Assume not; i.e. assume there exists some nonzero $t\in V$ contained in all tangent spaces $T_{D,\bar{\omega}}$ for $\bar{\omega}\in D_{sm}$. Choose basis for $V$ such that $t=(1,0,\cdots,0)$. Fix $\vartheta$ canonical: $a_L=a_{L(H,\chi)}$. So $\frac{\partial \vartheta}{\partial v_i}(\omega)=0$ for all $\omega\in V$ for which $\vartheta(\omega)=0$ (because it is true for an open dense subset). 

Let $f=\frac{1}{\vartheta}\cdot \frac{\partial \vartheta}{\partial v_i}$. Since $D$ is reduced, $f$ is holomorphic on $V$. From $\vartheta(v+\lambda)=a_{L(H,\chi)} \vartheta(v)$ we obtain 
\[
f(v+\lambda)=f(v)+\pi H(t,\lambda)
\]
for all $v\in V$ and $\lambda\in \Lambda$.
Since $df$ is $\Lambda$-periodic, it is a pullback of a holomorphic differential form on $X$. Hence 
\[
df=\sum \alpha_i dv_i
\]
for some $\alpha_i\in \CC$. So $f=\sum \alpha_i v_i+C$. By $f(v+\lambda)=f(v)+\pi H(t,\lambda)$, we obtain
\[
\sum \alpha_i \lambda_i=\pi H(t,\lambda)
\]
where $\lambda_1,\cdots,\lambda_g$ are coordinates of $\lambda$. So $f(v)=\pi H(t,v)+C$. Note that $f$ is holomorphic, but $H(t,\cdot)$ is anti-$\CC$-linear (as $H$ is Hermitian). Since $H$ is non-degenerate, we must have $t=0$ and this is a contradiction.
\end{proof}

\subsection{Projective embedding and Lefschetz theorem}
Let $(X,L)$ be a polarized abelian variety of type $(d_1,\cdots, d_g)$. Let $\Psi_L:X\to \PP^N$ be the associated map. We have seen that $d_1\geq 2$ and $\Psi_L$ is holomorphic. 

\begin{theorem}[Lefschetz]
If $d_1\geq 3$, then $\Psi_L$ is an embedding. 
\end{theorem} 

\begin{proof}
Recall that we have $\Psi_L:X\to \PP(W^*)$ which is defined by $P\mapsto \{D\in |W|:\ P\in D\}$. We need to show that $\Psi_L$ is injective and for all $x\in X$, $d\Psi_{L,x}$ is injective. 

Assume $\Psi_L(y_1)=\Psi_L(y_2)$ for $y_1,y_2\in X$; i.e. for any $D\in |L|$, we have $y_1\in D$ if and only if $y_2\in D$. Since $X[d_1]\subset K(L)\isom (\bigoplus \ZZ/d_i)^2$, we know $L=M^{d_1}$ for some $M$ positive definite. For a general member $D_M\in |M|$, we have $-D_M$ is reduced and $-t_x^* D_M=D_M$ only for $x=0$. Pick any point $x_1\in t_{y_1}^*D_M$. Then pick $x_2,\cdots, x_{d_1}\in X$ (where $d_1\geq 3$) with $\sum_{1}^{d_1} x_i=0$ such that $y_2\not\in t_{x_i}^* D_M$ for $2\leq i \leq d_1$. (This can be done because summation is continuous and $d_1\geq 3$.) Let 
\[
E=\sum_{i=1}^{d_1} t_{x_i}^* D_M\in |M^{d_1}|=|L|. 
\]
Since $y_1\in t_{x_1}^* D_M$ we have $y_1\in E$. By assumption, $y_2\in E$ but $y_2\not\in t_{x_i}^* D_M$ for $2\leq i \leq d_1$. So $y_2\in t_{x_1}^* D_M$ and hence $x_1\in t_{y_2}^* D_M$. Therefore, $t_{y_1}^* D_M\subset t_{y_2}^* D_M$ and by symmetry we also have $t_{y_1}^* D_M\supset t_{y_2}^* D_M$. Since $D_M$ is reduced, we get $t_{y_1}^* D_M=t_{y_2}^* D_M$ as divisors. So $t_{y_1-y_2}^* D_M=d_M$ and thus $y_1-y_2=0$. Therefore, $\Psi_L$ is injective.

Next, let $0\not=t\in T_{X,x}$ be the tangent vector at $x\in X$. We need to show that there exist $D\in |L|$ and $x\in D$ such that $t$ is \emph{not} tangent to $D$ at $x$. Assume not, i.e. $t$ is tangent to $D$ at $x$ for all $D\in |L|$ with $x\in D$. Pick $D_M\in |M|$ generic and so it is reduced. Pick $x_1\in t_{x}^* D_M$ and choose (as above) $x_2,\cdots, x_{d_1}\in X$ such that $\sum_1^{d_1} x_i=0$ and $x\not\in t_{x_i}^* D_M$ for $2\leq i \leq d_1$. Let $E'=\sum_{i=1}^{d_1} t_{x_i}^* D_M\in |L|$. Note that $x\in E'$. By assumption, $t$ is tangent to $E'$ at $x$. So $t$ is tangent to $t_{x_1}^* D_M$ at $x$. Since this is true for all $x_1\in t_x^* D_M$, it follows that $t$ is tangent to $D_M$ at all points $u\in D_M$. (This is because for $u\in D_M$, $0\in D_M-u=t_u^* D_M$ and hence $x\in t_{u-x}^* D_M$.) This implies that the image of the Gauss map is contained in a hyperplane.    
\end{proof}

\begin{definition}
A line bundle is called \textbf{very ample} if $\Psi_L$ is an embedding. A line bundle is called \textbf{ample} if $L^{\bigotimes n}$ is very ample for some $n\geq 1$. 
\end{definition}

\begin{proposition}
Let $X$ be a complex torus and $L\in \Pic(X)$. Then the following are equivalent:
\begin{enumerate}
\item $L$ is ample.
\item $L$ is positive definite.
\item $H^0(X,L)\not=0$ and $K(L)$ is finite.
\item $H^0(X,L)\not=0$ and $(L^g)>0$.
\end{enumerate}
\end{proposition}

\begin{corollary}
If $L\in \Pic(X)$ is ample, then $L^3$ is very ample.
\end{corollary}
\begin{proof}
Since $L$ is ample, by the above proposition, $L$ is positive definite and hence of type $(d_1,\cdots, d_g)$ with $d_1\geq 1$. Hence $L^3$ is of type $(3d_1,\cdots, 3d_g)$ with $3d_1\geq 3$. 
\end{proof}

\begin{theorem}
Let $X$ be a complex torus. The following are equivalent:
\begin{enumerate}
\item $X$ is an abelian variety (ie. has positive definite $L$).
\item $X$ admits the structure of a projective algebraic variety.
\end{enumerate}
\end{theorem}

\begin{proof}
Chow's theorem (GAGA) says if $Y$ is a complete algebraic variety and $Z$ is a closed analytic subset of $Y^{an}$, then there exists an algebraic subvariety $Z$ of $Y$ such that $Z^{an}=Z$. 
\end{proof}


\subsection{$L^2$}
Let $(X,L)\isom (X_M,M)\times (X_{N_1}, \overline{N_1})\times \cdots \times (X_{N_r},\overline{N_r})$. If suffices to understand $(X_M,\overline{M})$ and $(X_{N_i},\overline{N_i})$. And thus it suffices to understand 
\begin{enumerate}
\item $L=M$. (no fixed component)
\item $L=N_i$. (irreducible principal polarization)
\end{enumerate} 

Let $K_x:=X/\langle (-1)_x\rangle$ be a \emph{Kummer variety} which is an algebraic variety with $2^{2g}$ singular points.

\begin{theorem}
When $L=M$, $L^2$ gives an embedding. When $L=N_i$, if $L$ is symmetric, then $\Psi:K_X\to \PP^{2^g-1}$ is an embedding. 
\end{theorem}