% !TEX root = 7390.tex




\section{Abelian varieties}\label{Chapters/abelianvarieties}

\subsection{Definition and basic properties}

\begin{definition}
$X=V/\Lambda$ is called an \textbf{abelian variety} if it admits a positive definite line bundle, i.e. there exists $H\in NS(X)$ such that $H$ is positive definite. 
\end{definition}

Remark: We will see this agrees with our initial definition, i.e. such complex tori are ``algebraic/algebraizable''. (Hint: If $f:X\injects \PP^N$, then $f^*\mathcal{O}(1)$ must be positive definite.)

\begin{definition}
\noindent
\begin{enumerate}
\item A \textbf{polarization} on $X$ is $c_1(L)$ where $L$ is positive definite.
\item The \textbf{type} of a polarization is defined to be the type $(d_1,\cdots, d_g)$ of $L$ (or $c_1(L)$.)
\item A polarization is called \textbf{principal} if its type is $(1,\cdots, 1)$. 
\item For $(X,L)$ where $X$ is a complex torus and $L\in \Pic(X)$ is positive definite and $H=c_1(L)$, the pair $(X,H)$ is called a \textbf{polarized abelian variety} (P.A.V.).
\item A \textbf{homomorphism} of polarized abelian varieties is a map $f:(Y,M)\to (X,L)$ such that $f:Y\to X$ is a homomorphism of complex tori and $f^* c_1(L)=c_1(M)$. (We have seen that $c_1(f^*L)=f* c_1(L)$.)
\end{enumerate}
\end{definition}

Remark: 
$f^*L\sim_{an} M$. This forces any homomorphism $f$ to have a finite kernel. (Otherwise $f^*L$ would be degenerate and so would $M$ be degenerate.) Conversely, if $f:Y\to X$ is a homomorphism of complex tori with finite kernel and $L\in \Pic(X)$ is positive definite, then $f^*L\in \Pic(Y)$ is positive definite. We call this the ``induced'' polarization. 

\begin{corollary}
\noindent
\begin{enumerate}
\item Any complex subtorus of an abelian variety is an abelian variety.
\item Let $X$ be an abelian variety and $Y$ be a complex torus. If $X$ is isogenous to $Y$, then $Y$ is also abelian variety. 
\item For any abelian variety $X$, $\hat{X}=\Pic^0(X)$ is also an abelian variety.
\end{enumerate}
\end{corollary}

\begin{proof}
(c) follows from the fact that any non-degenerate $\varphi_L:X\to \hat{X}$ has finite kernel.
\end{proof}

\begin{lemma}
Every polarization is ``induced by'' a principal polarization via an isogeny. 
\end{lemma}

\begin{proof}
Let $(X,L)$ be a polarized abelian variety of type $(d_1,\cdots, d_g)$. Recall [insert label] we have an isogeny $p_1: X\to X_1$ where $X_1=V/(\Lambda_1\bigoplus \Lambda_2)$ and we have $M_1\in \Pic(X_1)$ with $P^* M_1\isom L$. Clearly $M_1$ is positive definite. 

We claim that $\text{type}(M_1)=(1,\cdots,1)$. We need Riemann-Roch to show this:
\[
d_1\cdots d_g = \chi(L)=(\deg p_1)\chi(M_1)=d_1\cdots d_g \chi(M_1)
\]
which implies that $\chi(M_1)=1=(-1)^s Pf(M_1)=Pf(M_1)$. 
\end{proof}

\begin{example}
When $X=\CC/\Lambda$ where $\Lambda=\ZZ \omega_1 \bigoplus \ZZ \omega_2\isom \ZZ^2$. (With loss of generality, we may assume that $\omega_1=1, \omega_2 = \tau$ with $Im \tau>0$.) Define $H:\CC\times \CC\to \CC$ by 
$$(v,w)\mapsto \frac{v\overline{ w}}{Im(\overline{\omega_1}\omega_2)}.$$
$H$ is obviously positive definite.
It is easy to show that $H$ is Hermitian and that $Im(H(\Lambda,\Lambda))\subset \ZZ$. (That is, $H$ is a Riemann form.) Hence, $\CC/\Lambda$ is an abelian variety. 

(Remark: $\CC^g/\Lambda$ where $g>1$ is usually not an abelian variety.)
\end{example}

Recall the theorem of the square says that for $\bar{v}\bar{w}\in X$, we have
\[
t^*_{\bar{v}+\bar{w}}L \isom t^*_{\bar{v}}L \bigotimes t^*_{\bar{w}}L \bigotimes L^{-1}.
\]
Note that $t^*_{\bar{w}}L \bigotimes L^{-1}$ can be re-written as $t^*_xL\bigotimes t^*_{-x}L\isom L^2$, hence we have 
\[
t^*_{\bar{v}}L \bigotimes t^*_{\bar{w}}L \bigotimes t^*_{-\bar{v}-\bar{w}}L \isom L^3.
\]
This generalizes to the following lemma:

\begin{lemma}
For $\bar{v_1},\cdots,\bar{v_n}\in X$, if $\sum_{i=1}^n \bar{v_i}=0$, then $\bigotimes_{i=1}^n t^*_{v_i} L\isom L^n$.
\end{lemma}
\begin{proof}
Since $L=L(H,\chi)$, $\bigotimes_{i=1}^n t^*_{v_i} L
=L(nH,\prod \chi e^{2\pi i Im H(v_i,\cdot)})=L(nH,\chi^n)
\isom L^n$.
\end{proof}

Remarks:
\begin{enumerate}
\item I will give an interpretation in terms of divisors.
\item one can use this to give an explicit basis for $H^0(X,L^n)$ in terms of basis for $H^0(X,L)$. For instance, if $(X,L)$ is of type $(1,\cdots,1)$, then $H^0(X,L)$ is one dimensional.
\end{enumerate}

\subsection{Riemann relations}

We now turn to the following question: What does it mean, in terms of a period matrix $\Pi$, for $X$ to be an abelian variety?
Let $X=V/\Lambda$. Let $\{e_1,\cdots , e_g\}$ be a $\CC$-basis for $V$ and $\{\lambda_1,\cdots, \lambda_{2g}\}$ be a $\ZZ$-basis for $\Lambda$. Then we have 
$$X=\frac{\CC^g}{\Pi \ZZ^{2g}/}$$
where $\Pi$ is the $g\times 2g$ period matrix. 

\begin{theorem}[Riemann relations]
$X$ is an abelian variety if and only if there exists a non-degenerate antisymmetric (i.e. alternating) matrix $A\in M_{2g\times 2g}(\ZZ)$ such that 
\begin{enumerate}
\item (Hermitian condition) $\Pi A^{-1} \Pi^t=0$; and
\item (positive definite condition) $i\Pi A^{-1}\overline{\Pi}^t>0$.
\end{enumerate}
\end{theorem}

\begin{proof}
We sketch the proof. Let $E:\Lambda\times \Lambda \to \ZZ$ be an arbitrary non-degenerate alternating form on $\Lambda$ (in $NS(X)$). Let $A$ be its matrix with respect to $\lambda_i$'s: $a_{ij}=E(\lambda_i,\lambda_j)$. Define $H:\CC^g\times \CC^g\to \CC$ by 
\[
H(u,v):=E(iu,v)+iE(u,v) \in \RR^{2g}
\]
where $E$ is the natural extension to $\RR^{2g}$. Define 
\[
\mathcal{J}=\binom{\Pi}{\overline{\Pi}}^{-1}_{2g\times 2g} \left(
\begin{array}{cc}
iIg & \ \\
\ & -i I_g\\
\end{array}
\right)\binom{\Pi}{\overline{\Pi}}.
\]
Such a construction is useful because $i\Pi = \Pi \mathcal{J}$. Note also that $E(\Pi x,\Pi y)=x^t A y$ for $x,y\in \ZZ^{2g}$. 

We claim that $A$ is Hermitian if and only if $\Pi A^{-1} \Pi^t=0$. We have seen that $H$ is Hermitian if and only if $E(iu,iv)=E(u,v)$ for all $u,v\in \CC^{g}\isom \RR^{2g}$. So $H$ is Hermitian if and only if $\mathcal{J}^t A \mathcal{J}=A$.

Next, we claim that assume $\Pi A^{-1} \Pi^t=0$, then the matrix of $H$ with respect to $\{e_1,\cdots,e_g\}$ is $2i(\Pi A^{-1} \overline{\Pi}^t)^{-1}$. Therefore, $H(u,v)=E(iu,v)+i E(u,v)$ is positive definite.
\end{proof}

Remark: If we assume that $\{\lambda_i\}$ is symplectic, and if we write $A=\left(
\begin{array}{cc}
0 & D\\
-D & 0 \\
\end{array}
\right)$
and $\Pi = ((\Pi_1)_{g\times g} | (\Pi_2)_{g\times g})$. Then the Riemann relations can be stated as 
\begin{enumerate}
\item $\Pi_2 D^{-1} \Pi_1^t-\Pi_1 D^{-1}\Pi_2^t=0$.
\item $i(\Pi_2 D^{-1}\overline{\Pi_1}^t-\Pi D^{-1}\overline{\Pi}^t)>0$. 
\end{enumerate}

\subsection{Divisors and Maps to $\PP^n$ for abelian varieties}
Let $X=V/\Lambda$ be an abelian variety of dimension $g$ with (positive definite) polarization $L\in \Pic(X)$. Define the rational (meromorphic?) map $\Psi_L: X\dashrightarrow \PP^N$ by
$$x\mapsto [\sigma_0(x):\cdots:\sigma_N(x)]$$
whenever there exists $i$ such that $\sigma_1(x)\not= 0$ where $\text{span}_\CC\{\sigma_0,\cdots,\sigma_N\}=H^0(X,L)$. 

By fixing a factor of automorphy, we may identify $H^0(X,L)$ with theta functions: fix a basis $\vartheta_0,\cdots,\vartheta_N$ for $H^0(X,L)$ and write $\Psi_L(\overline{v})=[\vartheta_0(v):\cdots:\vartheta_N(v)]$ where $\overline{v}=v$ modulo $\Lambda$. 

The big question is: when is $\Psi_L$ an embedding? Our goal is to show the following theorem.
\begin{theorem}[Lefschetz]
If $L$ is of the type $(d_1,\cdots,d_g)$ and $d_i\geq 3$, then $\Psi_L$ is an embedding.
\end{theorem}

\begin{example}
It is a good exercise to show that if $L$ is positive definite and of the type $(d_1,\cdots,d_g)$, then 
$$N+1=\dim H^0(X,L^{\otimes k})=\chi(L^{\otimes k})=Pf(L^{\otimes k})=Pf(k c_1(L))=(kd_1)(kd_2)\cdots (kd_g)=k^g(d_1d_2\cdots d_g).$$

Now assume $L$ is principal (i.e. of type $(1,\cdots,1)$), so $\dim H^0(X,L^k)=k^g$. This gives that $N=k^g-1$. So for example when $k=3$, $N=3^g-1$. When $g=1$, we can embed elliptic curves inside $\PP^2$ which says that the quantity $N=k^g-1$ is sharp in some sense. 
\end{example}

If $L$ is principally polarized, then $\dim H^0(X,L)=1$, so there exists a unique (up to scaling) theta function $\theta$ generating $H^0(X,L)$, which is called a \emph{Riemann's theta function}. 

It is a good exercise to give $\vartheta_0,\cdots,\vartheta_N$ a basis for $H^0(X,L^{k})$ in terms of Riemann's theta function. (Hint: Recall that $L^{\otimes k}\isom \bigotimes_{i=1}^k t_{\bar{v}_1}^* L$ and $\sum_{i=1}^k \bar{v}_i=0$. So one needs to ``shift and multiply'' $\theta$'s. Another hint: $\#\frac{1}{k} \Lambda / \Lambda = ?$ )