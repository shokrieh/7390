% !TEX root = 7390.tex




\section{Introduction}\label{Chapter:Introduction}

\subsection{About this Course}
This is an introductory course on the analytic and algebraic theory of abelian (and jacobian) varieties. We will start with the classical complex-analytic case to build some intuition. Then we will discuss the general theory over other fields. 

We will not follow any specific books, but the following resources were used while preparing for the lectures:

\cite{BirkenhakeLange}
\cite{BLRNeron}
\cite{FaltingsChai}
\cite{milneAV}
\cite{MumfordAV}
\cite{MumfordTata1}
\cite{MumfordTata2}
\cite{MumfordTata3}
\cite{CornellSilverman}

\subsection{What are Abelian Varieites?}

Origins of the theory of abelian varieties comes from a basic question in calculus! We know we can easily compute integrals of the form 
$$\int \frac{1}{\sqrt{1-x^2}}dx$$ 
by trigonometric substitution, but for integrals of the form 
$$\int \frac{1}{\sqrt{f(x)}}dx$$ 
where $\deg(f)\geq 3$, it turns out to be hard. However, even though people couldn't compute these integrals, they could see that there were identities of the form 
$$\int_0^a \frac{1}{\sqrt{f(x)}}dx +\int_0^b \frac{1}{\sqrt{f(x)}}dx=\int_0^{a* b} \frac{1}{\sqrt{f(x)}}dx$$
for some number $a*b$ obtained from $a$ and $b$. 


``Abelian Varieties are the simplest possible spaces, just tori's and thus groups'' - D-Mumford.


Abelian Varieties are useful in the following areas:
\begin{itemize}
\item Number Theory - class field theory; rationality versus transcendence; most of the serious things we know how to do in number theory involve working with moduli of abelian varieties (Fermat's last theorem, Faltings' theorem, etc).
\item Dynamical Systems - solutions to certain Hamilton systems.
\item Algebraic Geometry - If we're given a variety $X$ it's hard to understand, but we can get a handle on it by associating a canonical abelian variety $A(X)$ to it (such as Picard, Albanese, Intermediate Jacobian) and the good thing about $A(X)$ is that we can do a lot of linear algebra.
\item Physics - theta functions that solve heat equations; string theory.
\end{itemize}  

For the most part of this course, we will work over the base field $k=\CC$; we see lots of very interesting ideas in this case, and we don't need any particularly hard theory to get a handle on it. One reason the theory over $\CC$ is important is that an abelian variety $A$ in a precise sense is just $$A=A^{an}=\CC^g/\Lambda$$ 
where $\Lambda=\pi_1(A)\isom \ZZ^{2g}$ is a lattice, and so it is a torus. In other words, we have
$$0\to \Lambda\to \CC^g\to A\to 0.$$

Subtle remark: This identification makes sense in the ``analytic category'' but not in the algebraic category; the map $\CC^g\to A$ is not algebraic. So we can't study abelian varieties in this way solely through algebraic methods. In the analytic setting it's ``easy'' to understand line bundles, theta functions, etc. by going to $\CC^g$. (You can make sense of an analytification in nonarchimedean settings too, by using Berkovich spaces or formal schemes; this requires a lot more background but provides many important results.) Fortunately, there are some things from the complex-analytic setting which can be mimicked in the algebraic setting (e.g. the lattice $\Lambda$ can be related to the Tate module) and by using those algebraic analogues you can take the complex-analytic results over $\CC$ and try to reproduce them over other fields. 

Some more advanced topics that might be covered in detail later in the class include (depending on audience interest):
\begin{itemize}
\item The theory over general field.
\item Theta functions.
\item Neron models.
\item Non-archimedean uniformizations.
\item Moduli and compactifications.
\item Heights and metrized line bundles.
\item Degenerating families.
\end{itemize} 

Now, let's get to actual math. In scheme-theoretic language - 
for $k$ any field, a \emph{$k$-variety} is a geometrically integral $k$-scheme of finite type, and an \emph{abelian variety} over $k$ is a proper $k$-variety endowed with a structure of a $k$-group scheme. (This is the schematic definition. But we will not do it this way in this class.) We will be able to prove the following:

\begin{theorem}
Abelian Varieties are automatically abelian and projective.
\end{theorem}

Being abelian is easy to show, while being projective is much harder.

\subsection{Why study Abelian Varieties?}

More generally we can define an \emph{algebraic group} $G$ over $k$ as a connected, smooth $k$-group scheme. Examples include:
\begin{itemize}
\item affine algebraic groups (automatically subgroups of $\GL_n(k)$, i.e. linear algebraic groups)
\item abelian varieties (think of this as the projective case)
\end{itemize}

The following theorem says that these are the only building blocks:

\begin{theorem}[Chevalley's Theorem]
Let $G$ be an algebraic group over $k$. Suppose $k$ is perfect. Then there exists a unique short exact sequence 
$$0\to H \to G \to A \to 0$$
where $H$ is linear and $A=G/H$ is abelian. 
\end{theorem}

A proof can be found in B. Conrad's notes. 

\subsection{The Weierstrass Equation}
In the 1850s, Weierstrass studied $E=E^{an}=\CC/\Lambda$, which is a complex group (2-dimensional torus with complex multiplication). He asked whether $E^{an}$ is always ``algebraic/algebraizable'', and showed the answer is actually yes. In fact, he proved more. 

\begin{theorem}
$E=E^{an}$ has the structure of a smooth projective curve of genus $1$. Its affine equation is given by 
$$y^2=4x^3-60G_4x-140G_6$$
where $G_m=\sum_{\lambda\in\Lambda^*} \frac{1}{\lambda^m}$ for $m\in \ZZ$. More precisely, you can write down the Weierstrass $\wp$-function
$$\wp(z)=\frac{1}{z^2}+\sum_{\lambda\in\Lambda^*}\left(\frac{1}{(z-\lambda)^2}-\frac{1}{\lambda^2}\right)$$
and compute
$$y=\frac{dx}{dz}=\sum_{\lambda\in \Lambda} \frac{-2}{(2-\lambda)^3},$$
and see that the pair $(x,y)=(\wp(z),\wp'(z))$ satisfies the affine equation above. The mapping given by 
$$z+\Lambda\mapsto (\wp(z),\wp'(z))$$ induces a group isomorphism $\CC/\Lambda\isom E$ where $E$ is the projectivized elliptic curve.
\end{theorem}



What about higher dimensions? One direction is false: a general torus $\CC^g/\Lambda$ will not be algebraic. However, the converse is true; a general abelian variety $A$ will be (after analytification) of the form $\CC^g/\Lambda$, where $g=\dim(A)$. 

We will prove this theorem later, but it is not easy. 

\begin{theorem}
The Weierstrass parametrization gives a bijection between lattices $\Lambda$ in $\CC$, and the set of isomorphism classes of pairs $(E,\omega)$ where $E/\CC$ are elliptic curves and $\omega$ are holomorphic differential forms. 
\end{theorem}
