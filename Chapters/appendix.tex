% !TEX root = 7390.tex




\section{Appendix}\label{Chapters/appendix}

\subsection{Poincar\'{e} lemma, De Rham cohomology, Dolbeault cohomology}

If we are over $\RR$ we have the classical Poincar\'{e} lemma, which lets us get the connection between De Rham cohomology and singular cohomology. If we're over $\CC$, we have a $\bar{\partial}$-Poincar\'{e} lemma which tells us something about Dolbeault cohomology. We will say things about these, and then a bit more about Hodge theory (and what is the major simplification for abelian varieties versus Kalher manifolds).

\subsubsection{Classical Poincar\'{e} lemma} 
Let us look at $\CC^n$, which is $\RR^{2n}$ as a real manifold (or more generally any smooth manifold $M$). Let $T_0^*(\CC^n)$ (or $T_z^*(M)$) be the cotangent space, and pick the usual basis $\{dx_j,dy_j\}_{j=1}^n$. Over $\CC$ we will also want to work with the complex basis $\{dz_j,d\bar{z_j}\}_{j=1}^n$ with $dz_j=dx_j+idy_j$ and $d\bar{z_j}=dx_j-idy_j$. For the tangent space $T_0 \CC^n$ or $T_z M$, let $\{\partial/\partial z_j,\partial/\partial \bar{z_j}\}$ denote the dual basis. One can easily check that 
$$\frac{\partial}{\partial z_i}=\frac{1}{2}\left(\frac{\partial}{\partial x_i}-i\frac{\partial}{\partial y_i}\right),\ \ \ \frac{\partial}{\partial \bar{z_i}}=\frac{1}{2}\left(\frac{\partial}{\partial x_i}+i\frac{\partial}{\partial y_i}\right).$$

Remark: Over $\CC$, the Cauchy-Riemann equations tell us that if $f\in C^\infty(U)$ then $f$ is holomorphic if and only if $\partial f/\partial \bar{z}=0$. 

Given $f\in C^\infty(U)$ with $U\subset \CC^n$, define the \textbf{total differential} as 
$$df=\partial f+\bar{\partial}f=\Sigma_{j=1}^n \frac{\partial f}{\partial z_j}dz_i+\Sigma_{j=1}^n \frac{\partial f}{\partial \bar{z_j}}d\bar{z_j}.$$

\begin{theorem}
For $f\in C^\infty(U)$ with $U\subset \CC^n$, $f$ is holomorphic if and only if $\partial f=0$. 
\end{theorem}

Define the \textbf{holomorphic tangent bundle} as the vector subbundle generated by the $\{\partial/\partial z_j\}$. A \textbf{differential form} of degree $k$ (i.e. a $k$-form) is a smooth section of $\bigwedge^k T^*(M)$. That is, if $p\in M$, a $k$-form $\beta$ gives an alternating multilinear form $\bigoplus_{i=1}^l=k T_p(M)\to \RR$. We then get a sheaf of $k$-forms on a smooth manifold $M$, which we denote $\Omega_M^k$ (and $\Omega^0=\mathcal{O}_M$). 