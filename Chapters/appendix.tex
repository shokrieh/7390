% !TEX root = 7390.tex




\section{Appendix}\label{Chapters/appendix}

\subsection{Poincar\'{e} lemmas, De Rham cohomology, Dolbeault cohomology}

If we are over $\RR$ we have the classical Poincar\'{e} lemma, which lets us get the connection between De Rham cohomology and singular cohomology. If we're over $\CC$, we have a $\bar{\partial}$-Poincar\'{e} lemma which tells us something about Dolbeault cohomology. We will say things about these, and then a bit more about Hodge theory (and what is the major simplification for abelian varieties versus Kalher manifolds).

\subsubsection{Classical Poincar\'{e} lemma and De Rham cohomology} 
Let us look at $\CC^n$, which is $\RR^{2n}$ as a real manifold (or more generally any smooth manifold $M$). Let $T_0^*(\CC^n)$ (or $T_z^*(M)$) be the cotangent space, and pick the usual basis $\{dx_j,dy_j\}_{j=1}^n$. Over $\CC$ we will also want to work with the complex basis $\{dz_j,d\bar{z_j}\}_{j=1}^n$ with $dz_j=dx_j+idy_j$ and $d\bar{z_j}=dx_j-idy_j$. For the tangent space $T_0 \CC^n$ or $T_z M$, let $\{\partial/\partial z_j,\partial/\partial \bar{z_j}\}$ denote the dual basis. One can easily check that 
$$\frac{\partial}{\partial z_i}=\frac{1}{2}\left(\frac{\partial}{\partial x_i}-i\frac{\partial}{\partial y_i}\right),\ \ \ \frac{\partial}{\partial \bar{z_i}}=\frac{1}{2}\left(\frac{\partial}{\partial x_i}+i\frac{\partial}{\partial y_i}\right).$$

Remark: Over $\CC$, the Cauchy-Riemann equations tell us that if $f\in C^\infty(U)$ then $f$ is holomorphic if and only if $\partial f/\partial \bar{z}=0$. 

Given $f\in C^\infty(U)$ with $U\subset \CC^n$, define the \textbf{total differential} as 
$$df=\partial f+\bar{\partial}f=\sum_{j=1}^n \frac{\partial f}{\partial z_j}dz_i+\sum_{j=1}^n \frac{\partial f}{\partial \bar{z_j}}d\bar{z_j}.$$

\begin{theorem}
For $f\in C^\infty(U)$ with $U\subset \CC^n$, $f$ is holomorphic if and only if $\partial f=0$. 
\end{theorem}

A \textbf{differential form} of degree $k$ is a $C^\infty$-global section of $\bigwedge^kT^*=\Omega^k$. At each $p\in M$, it is an alternating multilinear form $T_p(M)\times\dots\times T_p(M)\rightarrow \mathbb{R}$. The set of $k$-forms on $M$ can be denoted as $\Omega^k(M)$, $A^k(M)$, $\Gamma(\Omega^k, M)$ or $H^0(M,\Omega^k)$. The differential maps $d:\Omega^k(M)\rightarrow\Omega^{k+1}(M)$ can be defined explicitly on coordinates, or alternatively characterized as the unique collection of $\mathbb{R}$-linear maps satisfying: 
\begin{enumerate}
\item $\forall f\in\Omega^0(M)=C^\infty(M)$, $df$ is the total differential of $f$, and $d(df)=0$;
\item We have $d(\alpha\wedge\beta)=d\alpha\wedge\beta+(-1)^{\deg(\alpha)}\alpha\wedge d\beta$.
\end{enumerate}


From these one can write down $d$ in local coordinate as $d(f dx_{i_1}\wedge\dots\wedge dx_{i_k})=df\wedge(dx_{i_1}\wedge\dots\wedge dx_{i_k})$, and verify that for any differential form $\alpha$, $dd\alpha=0$. Hence we have the de Rham cochain complex:
$$0\rightarrow\Omega^0(M)\xrightarrow{d}\Omega^1(M)\xrightarrow{d}\dots$$
whose cohomology 
$$H^p_{dR}(M)=\frac{\ker(d:\Omega^p(M)\to \Omega^{p+1}(M))}{Im(d:\Omega^{p-1}(M)\to \Omega^p(M))}$$ 
is called the \textbf{de Rham cohomology}. Elements of this kernel are called \textbf{closed} and elements of the image are called \textbf{exact}. (Example to keep in mind: If $S^1$ is the circle we have a closed form $d\theta$ which is not exact.) 
Note that $\Omega^k(M)=0$ for $k>\dim M$.  

\begin{lemma}[Poincar\'e Lemma] If $U\subset\mathbb{R}^n$ is contractible, then $H^p_{dR}(U)=0$ for $p\geq 1$.
\end{lemma}
	
This shows that $0\rightarrow\underline{\mathbb{R}}\xrightarrow{d}\Omega^0\xrightarrow{d}\Omega^1\xrightarrow{d}\dots$ is an exact sequence of sheaves. As a consequence, we have:
\begin{theorem}[de Rham theorem]
	If $M$ is a smooth manifold, then $\check{H}^p(M,\underline{\mathbb{R}})$ (which is our usual singular cohomology) is isomorphic to $H^p_{dR}(M)$. 
\end{theorem}

The idea of the proof is to break the above exact sequence up into short exact sequences in the standard way, then use the corresponding long exact sequences on cohomology.

\subsubsection{$\overline{\partial}$ Dolbeault cohomology}

Now we want to start using the complex structure on a complex manifold. Using holomorphic charts, we locally have differentials $\{dz_i,d\overline{z}_i\}$ spanning the cotangent space, with dual elements $\{\partial/\partial z_i,\partial/\partial \overline{z}_i\}$ spanning the tangent space. Using these we have the following splitting
$$T^*_p(M)={T^*}'_p(M)+{T^*}{''}_p(M),$$ 
where ${T^*}'$ is spanned by $dz_j$ and ${T^*}{''}$ is spanned by $\overline{dz_j}$. This gives a direct sum decomposition 
$$\bigwedge^nT^*_p(M)=\bigoplus_{p+q=n}\left(\bigwedge^p {T^*_p(M)}'\right)\otimes \left(\bigwedge^p {T^*_p}(M){''}\right).$$ 

with the $(p,q)$ part spanned by things of the form $dz_I\wedge d\overline{z}_J$ with $|I|=p$ and $|J|=q$. Using this decomposition, define a sheaf $\Omega^{p,q}$ of $C^\infty(p,q)$ forms: we set

$$\Omega^{p,q}(M)=\left\{ \varphi \in \Omega^n(M): \forall z, \varphi(z)\in \left(\bigwedge^p {T^*_p(M)}'\right)\otimes \left(\bigwedge^p {T^*_p}(M){''}\right)\right\}.$$

It is also denoted by $A^{p,q}(M)$, $\Gamma(\Omega^{p,q},M)$ or $H^0(M,\Omega^{p,q})$. This gives a filtration $\Omega^n(M)=\bigoplus_{p+q=n} \Omega^{p,q}(M)$.

What happens when we apply $d$ to something in $\Omega^{p,q}(M)$? We can see it maps into $\Omega^{p+1,q}(M)\oplus \Omega^{p,q+1}(M)$ because $\varphi(z)$ is in 
$$\left(\bigwedge^p {T^*_p(M)}'\right)\otimes \left(\bigwedge^p {T^*_p}(M){''}\right)\wedge T_z^*(M).$$


The decomposition of $\Omega^n$ induces a decomposition $d=\partial+\overline{\partial}$, where $\partial: \Omega^{p,q}\rightarrow\Omega^{p+1,q}$ and $\overline{\partial}:\Omega^{p,q}\rightarrow\Omega^{p,q+1}$. By writing them down in local coordinates one can check that $\overline{\partial}^2=0$, hence there is a cochain complex:
$$0\rightarrow\Omega^{p,0}(M)\xrightarrow{\partial}\Omega^{p,1}(M)\xrightarrow{\partial}\Omega^{p,2}(M)\cdots$$
and we define the \textbf{Dolbeault cohomology} as
$$H^{p,q}_{\overline{\partial}}(M)=\frac{\ker(\overline{\partial}: \Omega^{p,q}(M)\to \Omega^{p,q+1}(M))}{Im(\overline{\partial}:\Omega^{p,q-1}(M)\to \Omega^{p,q}(M))}$$

Now, we want to relate this to some other sheaf cohomology; our exact-seqeunce-based proof of the de Rham theorem suggests what we need. The 
\emph{$\overline{\partial}$-Poincar\'e lemma} says that on a polydisc $D$ in $\CC^n$ (a product of discs in $\CC$), $H^{p,q}_{\overline{\partial}}(D)=0$ for all $q\geq 1$. (Poincar\'{e} was studying the problem that if $g\in C^\infty(D)$ for $D\subset \CC$ then he wanted to find $f$ with $\partial f\partial \overline{z}=g$ which can be solved on a slightly smaller disc.)

Hence, as an analogue of the de Rham theorem, by letting $\Omega^p_{hol}$ be the sheaf of holomorphic $p$-forms, Poincar\'{e} lemma tells us that we have an exact sequence
$$0 \rightarrow\Omega^p_{\rm hol} \xhookrightarrow{}\Omega^{p,0} \xrightarrow{\overline{\partial}}\dots$$ 
with $\Omega^p_{hol}\to \Omega^{p,0}$ the inclusion and the other maps $\overline{\partial}$. Note that $\Omega^p_{hol}$ is indeed the kernel of $\overline{\partial}$ on $\Omega^p_{hol}$. By the same argument from our sheaf proof of the de Rham theorem we can prove the Dolbeault theorem:

\begin{theorem}[Dolbeault theorem]
$$H^{p,q}_{\overline{\partial}}(M)\isom H^q(M,\Omega^p_{hol})$$
for $p,q\geq 0$. 
\end{theorem}


\begin{example}
\noindent
\begin{itemize}
\item When $q\geq\dim(M)$, $H^{0,q}(M)=H^q(M,\mathcal{O}_M)=0$. (So if $q\geq \dim(M)$ both are zero.) 

\item When $q\geq 1$, $H^q(\mathbb{C}^n,\mathcal{O}_{\CC^n})=0$ by the above example and the $\overline{\partial}$-Poincar\'{e} lemma. 
\item When $D$ is a polydisc in $\mathbb{C}^n$, $H^{p,0}_{\overline{\partial}}(D,\Omega^p_D)=H^0(D,\mathcal{O}_D)\otimes\Omega^p_D$ is usually nontrivial, so the $q>1$ hypothesis in the Poincar\'{e} lemma matters! 	
\end{itemize}
\end{example}

\subsection{Hodge decomposition}
In this section, we will talk about the decomposition on complex tori. Assume $X=V/\Lambda$ is a complex torus. Let $e_1,\dots e_g$ be a complex basis of $V$ and $v_1,\cdots, v_g$ the corresponding coordinate functions. 
Then $H^n(X,\CC)$ is isomorphic to the set 
$$IF^n(X)=\bigoplus_{p+q=n} IF^{p,q}(X)$$
where $IF^{p,q}(X)$ (the ``invariant forms'') are the things of the form 
$$\sum_{|I|=p,|J|=q} a_{IJ}dv_I\wedge d\overline{v}_J$$
for $a_{IJ}\in \CC$. Note that because we are on a torus, it is easy to write down this decomposition.

 
For a more general (compact) $X$, we have the following more general theory: 
\begin{theorem}
	If $X$ is a compact complex manifold with a ``nice'' metric, then we have 
	$$H^n(X , \CC)=\bigoplus_{p+q=n}H^{p,q}(X)$$ 
where $H^{p,q}(X)$ is something that is isomorphic to $H^{p,q}_{\overline{\partial}}(X)$ and is isomorphic to the space ot harmonic forms. Also we have	
$$H^{p,q}=\overline{H^{q,p}}.$$
\end{theorem}


In the above theorem, ``nice'' metric means a Euclidean metric or a ``degree $2$ approximation'' of one, i.e. K\"{a}hler metric. For example, $X=V/\Lambda$ has the Euclidean metric and any complex projective variety has a K\"{a}hler metric (since $\PP^n$ has a Fubini-Study metric). For our situation of $X=V / \Lambda$, one can show the $IF^{p,q}(X)$ we wrote down is isomorphic to $H^{p,q}(X)$ is isomorphic to  
$H^{p,q}_{\overline{\delta}}(X)\isom H^{p+q}_d(X)\isom H^q(X,\Omega^p)$. (These are all hard; once we have shown these, we can then show that in the case of a complex torus, the harmonic things are just hte invariant forms and we recover what we had above.)

What is in the background of these isomorphisms? (We only consider the case $X=V/\Lambda$, but the following also contains all of the ideas we need in general.) We start with our Euclidean metric $ds^2=\Sigma dv_i\otimes d\overline{v}_i$. This has an associated $(1,1)$-form 
$$\omega=-\frac{1}{2}Im(ds^2)=\frac{i}{2}\sum_{i=1}^g dv_i\wedge d\overline{v}_i.$$
Then we get a volume form 
$$dv=\frac{1}{g!}\bigwedge^g \omega = (-1)^{\binom{g}{2}}\left(\frac{i}{2}\right)^g(dv_1\wedge d\overline{v}_1\wedge dv_2\wedge \cdots).$$

Now that we have a volume form we can define an inner product on $\Omega^{p,q}(M)$ (but not complete, so not a Hilbert space) by 
$$(\varphi,\psi)=\sum_{|I|=p,|J|=q} \int_X \varphi_{IJ}\overline{\psi}_{IJ}dv$$
for $\varphi=\sum \varphi_{IJ}dv_I\wedge \overline{v}_J$ and similarly for $\psi$. 
This makes it into a pre-Hilbert space (i.e. a non-complete inner product space) and we can then define an adjoint map $\overline{\delta}$ of $\overline{\partial}$ satisfying $(\varphi,\overline{\partial}\psi)=(p\overline{\varphi},\psi)$. Then the \textbf{Laplace-Beltrami operator} is given by
$$\Delta=\overline{\partial\delta}+\overline{\delta\partial}:\Omega^{p,q}(M)\to \Omega^{p,q}(M).$$
Note that $\Delta$ can also be written as $(\overline{\delta}+\overline{\partial})^2$.
In our situation in coordinates we can compute 
$$\Delta(\varphi dv_I\wedge dv_J)=-\sum_i \frac{\partial^2 \varphi}{\partial v_i\partial \overline{v}_i}(dv_I\wedge dv_J),$$
and so this is really the usual Laplacian.

The main point of this formal setup is that we can very easily show that a closed form $\psi$ whose ``norm'' $(\psi,\psi)$ is minimal in its class (in the de Rham or Dolbeault cohomology) is the unique solution in that class to $\overline{\delta}\psi=0$. Note that $\overline{\partial}\psi=0$ and $\overline{\delta}\psi=0$. Therefore $\Delta \psi=0$ and $\psi$ is Harmonic. (The converse is easy to prove too.) Therefore, we conclude that elements of $H^{p,q}(X)$, which are Harmonic $(p,q)$-forms, are unique representatives for classes in $H^{p,q}_{\overline{\partial}}(X)$. In order to understand these Harmonic forms, we need to solve partial differential equations. And there are two operators that can help solving these equations:
\begin{enumerate}
\item $H:\Omega^{p,q}(M)\to \Omega^{p,q}(M)$ which is defined by
$$\varphi dv_I\wedge d\overline{v}_J\mapsto \left(\frac{1}{\text{vol}(X)}\int_X \varphi dV\right)dv_I\wedge d\overline{v}_J.$$
This projects any form onto an invariant thing (certainly satisfies $H^2=H$) and in the abelian variety cse this projects onto $IF^{p,q}(M)$. 

\item The second operator is $G$, a Green's function for $\Delta$ such that it is an inverse to the extent we can have: $\Delta G=G\Delta = 1-H$ and $HG=GH=0$. In general, such a $G$ is hard to find! (For general K\"{a}hler manifolds, one can use the Sobolev lemma. See Griffiths and Harris Chapter $0$. [farbod cite]) For complex tori, we are okay because such a $G$ has a formula given in terms of Fourier analysis: Let $\varphi$ be a function on $X$, then we can lift it to $\tilde{\varphi}$ a periodic function on $V\isom \CC^g$. For such a $\tilde{\varphi}$, we have a fourier expansion and $G\tilde{\varphi}$ is just the same expression with renormalized coefficients. 

Once we have this, we can then prove the Hodge theorem, that every class has a unique harmonic representative (ultimately the hard part of the theory is showing that we have enough harmonic forms!) because for a form $\varphi$, we can explicitly write down
$$\varphi=H\varphi + \overline{\delta}\overline{\partial}\varphi+\overline{\partial}\overline{\delta}\varphi.$$ as our Hodge decomposition.
\end{enumerate}

\subsection{Divisors v.s. line bundles}
Let $(X,\mathcal{O}_X)$ ($\mathcal{O}_X$ is the holomorphic structural sheaf of $X$) be a complex $g$-dimensional manifold (which is not necessarily compact). Everything has algebraic analogue as I will remark. (Refer to chapter 2 and also $\S\S 4-6$ in Hartshorne.) 

Let $\mathcal{K}_X$ be the sheaf (of rings) of meromorphic functions. Algebraically, $\mathcal{K}_X(\mathcal{U})$ is $\mathcal{O}_X(\mathcal{U})$ localized by the multiplicative system of ``non-zero divisors''. This defines a presheaf that will be sheaffified. If $X$ is ``integral'', then $\mathcal{K}_X(\mathcal{U})$ is simply the fraction field of $\mathcal{O}_X(\mathcal{U})$.

We have an exact sequence of sheaves 
\[
0\to \mathcal{O}_X^\times \overset{i}{\to} \mathcal{K}_X^\times \overset{j}{\to} \mathcal{K}_X^\times/\mathcal{O}_X^\times \to 0.
\]
where $\mathcal{K}_X^\times$ is the sheaf of multiplicative subgroups of $\mathcal{K}_X$. The associated long exact sequence of cohomology is then given by 
\[
H^0(X,\mathcal{K_X^\times})\overset{j_*}{\to } H^0(X,\mathcal{K}_X^\times/\mathcal{O}_X^\times) \overset{\delta}{\to} H^1(X,\mathcal{O}_X^\times)\to \cdots
\]

If $X$ is a submanifold of $\PP^{N}$ (algebraically, if $X$ is projective), then $H^1(X,\sK^*_X) = 0$. So $\delta$ induces an isomorphism $\Pic(X) \simeq \CDiv(X)/\CPrin(X)$.

We are going to make all of these objects very explicit.

\subsubsection{$\Pic(X)$}
Let $L$ be a line bundle, there exists an open cover $\langle U_\alpha\rangle$ of $X$ with trivialization $\{\phi_\alpha : L(U_\alpha) \to U_\alpha \times \bC\}$. The associated transition functions $\{g_{\alpha\beta} : U_\alpha \cap U_\beta \to \bC\}$ are defined by
\begin{align*}
\phi_\alpha \circ \phi_\beta^{-1} : (U_\alpha \cap U_\beta) \times \bC &\to  (U_\alpha \cap U_\beta) \times \bC\\
(z,\cdot) &\mapsto (z,g_{\alpha\beta}(\cdot))
\end{align*}
and satisfy that $\{g_{\alpha\beta} : U_\alpha \cap U_\beta \to \bC^*\}$ are holomorphic, and $g_{\alpha\beta}g_{\beta\alpha} = 1$ and $g_{\alpha\beta}g_{\beta\gamma}g_{\gamma\alpha} = 1$, so $\{g_{\alpha\beta}\}$ is a $1$-\v{C}hech-cocycle.

Changing $\phi_\alpha$ to $\phi'_\alpha = f_\alpha\phi_\alpha$ gives a 1-coboundary $g'_{\alpha\beta} = (f_\alpha/f_\beta)g_{\alpha\beta}$.


Conversely, $\{g_{\alpha\beta}\}$ gives a line bundle $\bigsqcup U_\alpha \times \bC/\sim$


\subsubsection{$\CDiv(X)$}
$\CDiv(X) = H^0(X,\sK^*_X/\sO^*_X)$ (this makes sense for any scheme). A global section of $\sK^*_X/\sO^*_X$ (i.e., a Cartier divisor) can be presented by
\begin{enumerate}
\item An open cover $\{U_\alpha\}$ of $X$.
\item Meromorphic functions $f_\alpha$ on $U_\alpha$ (not identically zero). (\textbf{Remark:} Changing $f_\alpha$ to $s_\alpha f_\alpha$ for $s_\alpha \in \sO_X^*$ is ok!) such that for all $\alpha$ and $\beta$, $f_\alpha/f_\beta \in \sO_X^*(U_\alpha\cap U_\beta)$.
\end{enumerate}
We write $D = (\{U_\alpha\}, \{f_\alpha\})$.
\subsubsection{$\CPrin(X)$}
$\CPrin(X) = \Im(H^0(X,\sK^*_X) \xrightarrow{j_*} \CDiv(X))$, so $D \in \CPrin(X)$ if $D = (\{U_\alpha\}, \{f_\alpha\})$, where $f_\alpha = f|_{U_\alpha}$ for some $f \in H^0(X,\sK^*_X)$.

\begin{definition} 
 $D_1 = (\{U_\alpha^{(1)}\}, \{f_\alpha^{(1)}\})$ and $D_2 = (\{U_\alpha^{(2)}\}, \{f_\alpha^{(2)}\})$ are linearly equivalent, $D_1 \sim D_2$, if $D_1 - D_2 := (\{U_\alpha ^{(1)}\cap U_{\alpha'}^{(2)}\}, \{f_\alpha^{(1)}/f_{\alpha'}^{(2)}\})\in \CPrin(X)$. 
\end{definition}

\subsubsection{Weil Divisors}
In nice situations, for example, 
\begin{itemize}
\item Complex manifold
\item Noetherian, integral, separated, regular in codimension $1$ (if $\dim \sO_{X,p} = 1$, then $\sO_{X,p}$ is \emph{regular}, i.e., $\dim(\fm_p/\fm_p^2) = 1$.
\end{itemize}
Let $Y$ be a codimension $1$ analytic submanifold of $X$.
\begin{fact}
  Then $Y$ is an ``analytic hypersurface'', i.e., for any $p \in Y$, there is an open set $U$ containing $p$ such that $U\cap Y$ is the zero set of some holomorphic function $g$. 

$Y = Y_1 \cup \dotsb \cup Y_m$, where each $Y_i$ is the closure of a connected component of $Y\setminus Y_{\text{sing}}$. 

We say that $Y$ is irreducible if $m = 1$.
\end{fact}

\begin{definition}
  A \emph{Weil divisor} $D$ on $X$ is a ``locally finite'' formal linear combination $D = \sum_i a_i Y_i$, $a_i \in \bZ$, where the \emph{prime divisors} $Y_i$ are irreducible, analytic codimension $1$ submanifolds. (Algebraically, finite sum, prime divisors are closed integral (implies irreducible) codimension $1$ subschemes.

``locally finite'' means that for every $p \in X$, there is an open set $U$ containing $p$ such that only finitely many of the intersections $U \cap Y_i$ are nonempty. 

If $X$ is compact, locally finite is the same as the sum being finite.

$\WDiv(X) = \{\sum_i a_i Y_i \text{ locally finite, $a_i \in \bZ$}  \}$
\end{definition}
We will now answer two questions:
\begin{enumerate}
\item What is $\WPrin(X)$?
\item How does $\WDiv$ compare to $\CDiv$?
\end{enumerate}
\subsubsection{Answer 1}
Let $Y$ be a prime divisor. 

In the analytic case, for $p \in Y$, given $f$ holomorphic, we say that the order of $f$ along $Y$ at $p$ is $n$ if in $\sO_{X,p}$, $f = g^nh$, where $g$ is a local equation for $Y$ and $n$ is the largest such number.
\begin{fact}
  The order is independent of $p \in Y$, and let $\ord_Y f$ be this number.
\end{fact}
For meromorphic $f$, locally $f = g/h$, and we define $\ord_Y f = \ord_Y g - \ord_Y h$.

In the algebraic case, lLet $\eta \in Y$ be a generic point (e.g., $\eta = (0) \in \{(0), (p)\} = \Spec \bZ$. $\sO_\eta$ is DVR, so we define $\ord_Y f = \Val_\eta(f)$. 

Given $f$, we define $\Div(f) = \sum_Y \ord_Y(f) Y$. The collection of divisors is $\WPrin(X)$.
\begin{fact}
  The above sum is locally finite in the analytic case and finite in the algebraic case.
\end{fact}
If $\ord_Y(f) = n > 0$, we say that $f$ has a zero with multiplicity $n$ along $Y$, similarily, if $n < 0$, $f$ has a pole with multiplicity $n$ along $Y$.

\subsubsection{Answer 2}
In our situation (complex manifolds or integral, Noetherian, separable, regular in codimension $1$ and locally factorial), 
there exists an isomorphism $\CDiv \to \WDiv$ respecting principal divisors. 

Given $D = (\{U_\alpha\}, \{f_\alpha\}) \in \CDiv$, since $f_\alpha/f_\beta \in \sO^*_X(U_\alpha\cap U_\beta)$, for all prime divisors $Y$, we get $\ord_Y(f_\alpha) = \ord_Y(f_\beta)$, so we can define $D' = \sum_{Y \cap U_\alpha \neq \emptyset}\ord_Y(f_\alpha) Y \in \WDiv(X)$.

Conversely, given $D = \sum_i a_i Y_i \in \WDiv(X)$, we fix $\{U_\alpha\}$ such that $Y_i$ has a local defining function $g_{i\alpha} \in \sO(U_\alpha)$, and let $f_\alpha = \prod_i g_{i\alpha}^{a_i} \in \sK_X(U_\alpha)$, and let $D' = (\{U_\alpha\}, \{f_\alpha\}) \in \CDiv(X)$.



Now we have set up a correspondence between Cartier divisors $CDiv(X)$ and Weil divisors $WDiv(X)$.

\begin{lemma}
 These maps (corresponding to $Cdiv(X)~\leftrightarrow~WDiv(X)$) are homomorphisms and inverses of each other, respecting $WPrin$ and $CPrin$.
\end{lemma}

Note that in "nice situations", we will talk about $Div(X):= CDiv(X)$ and we can identify them as Weil divisors also. Then, $Prin(X)= CPrin(X)$, can be identified with $WPrin(X)$.

\begin{remark}
If $X$ is a Riemann surface, then the points are prime divisors. So $Div(X)$ is huge, but in general $Div(X)$ might even be empty. However, if $i: X \hookrightarrow \mathbb{P}^N$, then we obtain lots of divisors by intersecting $i(X)$ with hyperplanes. Converse is also true, if one has enough divisors then $\exists~ X \hookrightarrow \mathbb{P}^N$ (Kodaira embedding).

\end{remark}
Recall that we have a connecting homomorphism $\delta: \dfrac{CDiv(X)}{CPrin(X)} \rightarrow Pic(X)= H^1(X,\mathcal{O}^{*}_X)$. We will describe this map explicitly in next section, i.e. going from a divisor to a line bundle.

\subsubsection{Divisors vs Line bundles}

Consider a map $\delta: Div(X) \rightarrow Pic(X)$ denoted by $D \mapsto \mathcal{L}(D)$ or ($\mathcal{O}_X(D)$). It can be described explicitly as follows. Given a Cartier divisor, $({U_{\alpha}}, {f_{\alpha}})$ define $g_{\alpha\beta}=\frac{f_{\alpha}}{f_{\beta}} \in \mathcal{O}^{*}_{X}(U_{\alpha}\cap U_{\beta})$.\\

It is easy to check the following facts:

\begin{itemize}
\item[a.] $g_{\alpha\beta}$ is a \v{C}ech 1-cocycle.
\item[b.] The map is well-defined because changing $f_{\alpha}$ (for the same divisor $D$) gives us a 1-coboundary condition.
\item[c.] The map is a homomorphism and coincides with the connecting homomorphism $\delta$.
\item[d.] If D is principal, then $g_{\alpha\beta}=1$ and $\mathcal{L}(D)\simeq \mathcal{O}_X$, i.e. is trivial.
\item[e.] Conversely, if $\mathcal{L}(D)$ is trivial then 
\begin{center}
$g_{\alpha\beta}=\frac{h_{\alpha}}{h_{\beta}}$, $h_{\alpha}\in\mathcal{O}_X(U_{\alpha})$\\
$g_{\alpha\beta}~=~\frac{f_{\alpha}}{f_{\beta}}$\\
$f_{\alpha} h_{\alpha}^{-1}~=~f_{\beta} h_{\beta}^{-1}$ 
\end{center}
So, if we define $f=f_{\alpha} h_{\alpha}^{-1}$ on $U_{\alpha}$ then $f$ extends to a global meromorphic function. Hence, $D$ is principal divisor.

\end{itemize}

Observe that by (d) and (e) we get kernel $\delta=Prin(X)$.

\begin{definition}

Two divisors D and D' are called linearly equivalent, $D\sim D'$ if $$D-D'=div(f)$$ for some $f \in H^0(X,K^*_X)$.

\end{definition}

\begin{remark}

\begin{itemize}
\item[a.] If $D\sim D'$, then $\mathcal{L}(D)=\mathcal{L}(D')$.
\item[b.] $\mathcal{L}$(.) is functorial.
\item[c.] If $f:~X\rightarrow X'$ is a morphism then we can define a pullback $f^*:$ Div($X)\rightarrow$ Div($X'$) as follows. If $D=({U_{\alpha}}, {f_{\alpha}}) \in$ Div($X$) then it maps to $({f^{-1}(U_{\alpha})}, {f_{\alpha}\circ f})$. It makes sense if $f(X') \not\subset D$ as Weil Divisor.
In terms of Weil divisors $\sum a_{i}Y_{i} \mapsto \sum a_{i}f^{*}(Y)$, $f^*(Y)$ lies over hypersurface $f^{-1}(Y)$ but perhaps with multiplicity.
\item[d.] $f^*(\mathcal{L}(D))=\mathcal{L}(f^*(D))$

\end{itemize}
\end{remark}

\subsubsection{Divisors of Sections}

For a global meromorphic function $f \in H^0(X,K^*_{X})$, we defined $div(f)$. We now want to define a notion of $div(s)$ for a global meromorphic section of an arbitrary line bundle $L$ $s \in H^0(X, L\otimes_{\mathcal{O}_{X}}K_X)$.

The following theorem explains well the reason behind studying $div(s)$.

\begin{theorem}
 Let $L$ be a line divisor. Then, $L=\mathcal{L}(div(S))$. 
\end{theorem}

\begin{corollary}

$L \in Im\bigg(\delta:~\dfrac{Div(X)}{Prin(X)} \rightarrow Pic(X)\bigg)$ $\Leftrightarrow$ $L$ has a non-zero global meromorphic section. 
\end{corollary}

Let $L$ be a line bundle on $X$. Fix $\{U_{\alpha}\}$ with trivializations $\phi_{\alpha}:L(U_{\alpha})\rightarrow U_{\alpha} \times \mathbb{C}$ and transition functions $g_{\alpha\beta}\in {\mathcal{O}^*_{X}(U_{\alpha}\cap U_{\beta})}$.

Note that:
\begin{itemize}
\item[a.] A holomorphic (respectively, meromorphic) section of $L$ over $U$ open is given by a collection of functions $s_{\alpha} \in \mathcal{O}_{X}(U \cap U_{\alpha})~($respectively, $K_{X}(U \cap U_{\alpha}))$ satisfying $s_{\alpha}=g_{\alpha\beta}s_{\beta}$ on $U \cap U_{\alpha} \cap U_{\beta}$.
\item[b.] If $s$ and $s'$ are global meromorphic sections then $\dfrac{s}{s'}$ is a global meromorphic function. One consequence of this is if $D=div(s)$ and $D'=div(s')$ then $D-D'=div\bigg(\dfrac{s}{s'}\bigg)=div(f)$, which is a principal divisor hence, $D$ and $D'$ are linearly equivalent so, $\mathcal{L}(D)=\mathcal{L}(D')$.

\end{itemize}

Let $s$ be a meromorphic section of $L$ given by ${s_{\alpha}}$. Then $\dfrac{s_{\alpha}}{s_{\beta}}=g_{\alpha\beta} \in \mathcal{O}_{X}(U_{\alpha} \cap U_{\beta})$. So, $ord_{Y}(s_{\alpha})= ord_{Y}(s_{\beta})$. 

Therefore, we can define $div(s)= \sum_{U_{\alpha} \cap Y \neq \emptyset} ord(s_{\alpha})Y$.

\begin{example}
Let $D=({U_{\alpha}}, {f_{\alpha}}) \in Div(X)$. Then ${f_{\alpha}}$ gives a global section $s$ for $\mathcal{L}(D)$. Also, $div(s)$ is the element of $WDiv(X)$. The theorem and corollary from before hold in this case.

\end{example}

\begin{remark}

$D$ is called effective divisor if one of the following equivalent condition holds:

\begin{itemize}
\item[a.] $D=\sum a_{i}Y_{i}~a_{i}\geq0$
\item[b.] $D=({U_{\alpha}}, {f_{\alpha}}),~f_{\alpha}\in \mathcal{O}_{X}(U_{\alpha})$
\item[c.] $D=div(s)$, where $s$ is a global holomorphic section
\end{itemize}
\end{remark}

\subsubsection{Another interpretation of global holomorphic sections $(H^0(X,L))$ and effective divisors}

Let $D=\sum a_{i}Y_{i} \in Div(X)$
\begin{definition}

\begin{itemize}
\item[a.]$R(D)=\{f\in H^0(X,K^*_{X})\mid div(f)+D \geq 0\}$.
A function $f \in R(D)$ is holomorphic on $X \setminus \cup Y_{i}$ and $ord_{Y_{i}}(f) \geq -a_{i}$

\item[b.] The complete linear system of $D,~|D|=\{E\in Div(X) \mid E\sim D, E \geq 0\}$ 

\end{itemize}

\end{definition}

Let $L$ be a line bundle, define $|L|:=|D|$, if there exists a divisor $D$ such that $\mathcal{L}(D)=L$.

\begin{lemma}

Let $D=div(s_{0})$ for a global meromorphic section $s_{0}$ of $L$. Then
\begin{center}
$R(D) \rightarrow H^0(X,L)$\\
$f \mapsto fs_{0}$
\end{center}
is an isomorphism.
\end{lemma}

\begin{proof}
The map is well defined because $div(fs_{0})=div(f)+div(s_{0})=div(f)+D\geq0$. So, $fs_{0}$ is holomorphic. 
It is easy to see that the map is injective.
Let $s$ be a global holomorphic section. Define $f_{s}:=\dfrac{s}{s_{0}}$, we have shown before that $f_{s}$ is a global meromorphic function and $div(f_{s})=div(s)-div(s_{0}) \geq-D$. So, $f_{s} \in R(D)$. Hence, the map is surjective.
\end{proof}

\subsubsection{$R(D)$ vs $|D|$}

Let $E \in |D|$, then $\exists~f \in R(D)$ such that $E=D+div(f)$.
If $X$ is compact, any two such $f$ are related by a multiplicative constant. This is because, if $E=D+div(f)=D+div(h)$ then $div\bigg(\dfrac{f}{h}\bigg)=0$. Compactness implies that $\dfrac{f}{g}$ is constant.

\begin{corollary}
$|D|=\mathbb{P}(R(D)) \simeq \mathbb{P}(H^0(X,L))$. Since, $R(D)$ is a vector space, $|D|$ is projective space.
\end{corollary}

\subsubsection{linear system}

Recall that $|D|=\{E\geq 0:\ E\sim D\}$ is called a \textbf{complete linear system} of $D$. Here we introduce a more general notion.

\begin{definition}
A \textbf{linear system} is a family of effective divisor $\mathcal{E}=\{D_\lambda\}_{\lambda\in I}$ corresponding to a linear subspace of $\PP^N\isom \PP(H^0(X,L))$ for some $L$, i.e. $I\isom \PP^n$ for some $n\leq N$. That is, fix a subspace $V\subset H^0(X,L)$ and $\mathcal{E}=\PP(V)$.
\end{definition}

\begin{definition}
The \textbf{dimension} (or \textbf{rank}) of $\mathcal{E}=\{D_\lambda\}_{\lambda\in I\isom \PP^n}$ is the projective dimension $n$.
\end{definition}

For example, $\dim|D|=\dim H^0(X,L)-1$. We also have the following names:
\begin{itemize}
\item a \textbf{pencil} is a linear system of dimension $1$
\item a \textbf{net} is a linear system of dimension $2$
\item a \textbf{web} is a linear system of dimension $3$
\end{itemize}

\begin{definition}
Let $\mathcal{E}=\{D_\lambda\}_{\lambda\in I}\subset |D|$ be a linear system. The \textbf{base locus} of $\mathcal{E}$ is defined to be
$$\cap_{\lambda\in I} D_\lambda :=\{F\in Div(X):\ D_\lambda-F\geq 0\ \forall \lambda\in I\}.$$
We also take \textbf{fixed component} of $\mathcal{E}$ to be the divisor $F$ which is the base locus. 
\end{definition}

\begin{theorem}[Bertini]
If $D$ is a ``generic'' element of a linear system $\mathcal{E}$, then $D$ is smooth away from the base locus.
\end{theorem}

\subsubsection{Maps to $\PP^N$}
\begin{definition}
Let $\mathcal{E}=\PP(W)$ where $X$ is a compact complex manifold $W$ is a subspace of $H^0(X,L)$. We say that $\mathcal{E}$ is \textbf{base point free} if not all $s\in W$ (global holomorphic sections of $L$) vanish at any point $P\in X$. That is, there does not exist $P\in X$ such that for all $s\in W$ we have $s(P)=0$. 
\end{definition}

Let $X$ be a compact complex manifold. We have a one-to-one correspondence between non-degenerate maps $f:X\to \PP^N$ modulo projective transformations and line bundles $L\in \Pic(X)$ with subspace $W\subset H^0(X,L)$ such that the linear system $\mathcal{E}=\PP(N)$ is base-point free. 

We are going to give explicit maps that lead to the above correspondence. For a given line bundle $L\in \Pic(X)$ and $W\subset H^0(X,L)$, there are three versions of defining maps from $X$ to $\PP^N$ which we will list them all below:

\begin{enumerate}
\item $f:X\to \PP(W^*)$ given by 
$$P\mapsto \{s\in W: s(P)=0\}/\text{scaling}.$$
We denote $\{s\in W: s(P)=0\}$ by $\tilde{H_P}$. 
\item $f:X\to \PP(W^*)$ given by
$$P\mapsto \{D\in \mathcal{E}:\ P\in D\}.$$
We denote $\{D\in \mathcal{E}:\ P\in D\}$ by $H_P$. Note that $H_P$ is already scaled.
\item Let $s_0,\cdots, s_n$ be a basis for $W$. Then $f:X\to \PP^n$ is given by
$$P\mapsto [s_0(P):s_1(P):\cdots:s_n(P)].$$
Note that this is an explicit map. 
\end{enumerate}

Now, suppose we are given $f:X\to \PP^n$, we let $W=f^*(H^0(\PP^n,\mathcal{L}(H)))$ where $H$ is any hyperplane. Then $W\subset H^0(X,L)$ where $L=f^*(\mathcal{L}(H))$. It is easy to show that this construction is base point free. 

Remark: The choice of $\{s_0,\cdots, s_n\}$ for $W$ corresponds to the choice of $\{x_0,\cdots,x_n\}$ for $\PP^n$. Also, note that $\deg(f(X))=\int_X \bigwedge^n c_1(L)=(L^n)=:(D^n)$. 
