% !TEX root = 7390.tex




\section{Appendix}\label{Chapters/appendix}

\subsection{Poincar\'{e} lemma, De Rham cohomology, Dolbeault cohomology}

If we are over $\RR$ we have the classical Poincar\'{e} lemma, which lets us get the connection between De Rham cohomology and singular cohomology. If we're over $\CC$, we have a $\bar{\partial}$-Poincar\'{e} lemma which tells us something about Dolbeault cohomology. We will say things about these, and then a bit more about Hodge theory (and what is the major simplification for abelian varieties versus Kalher manifolds).

\subsubsection{Classical Poincar\'{e} lemma} 
Let us look at $\CC^n$, which is $\RR^{2n}$ as a real manifold (or more generally any smooth manifold $M$). Let $T_0^*(\CC^n)$ (or $T_z^*(M)$) be the cotangent space, and pick the usual basis $\{dx_j,dy_j\}_{j=1}^n$. Over $\CC$ we will also want to work with the complex basis $\{dz_j,d\bar{z_j}\}_{j=1}^n$ with $dz_j=dx_j+idy_j$ and $d\bar{z_j}=dx_j-idy_j$. For the tangent space $T_0 \CC^n$ or $T_z M$, let $\{\partial/\partial z_j,\partial/\partial \bar{z_j}\}$ denote the dual basis. One can easily check that 
$$\frac{\partial}{\partial z_i}=\frac{1}{2}\left(\frac{\partial}{\partial x_i}-i\frac{\partial}{\partial y_i}\right),\ \ \ \frac{\partial}{\partial \bar{z_i}}=\frac{1}{2}\left(\frac{\partial}{\partial x_i}+i\frac{\partial}{\partial y_i}\right).$$

Remark: Over $\CC$, the Cauchy-Riemann equations tell us that if $f\in C^\infty(U)$ then $f$ is holomorphic if and only if $\partial f/\partial \bar{z}=0$. 

Given $f\in C^\infty(U)$ with $U\subset \CC^n$, define the \textbf{total differential} as 
$$df=\partial f+\bar{\partial}f=\Sigma_{j=1}^n \frac{\partial f}{\partial z_j}dz_i+\Sigma_{j=1}^n \frac{\partial f}{\partial \bar{z_j}}d\bar{z_j}.$$

\begin{theorem}
For $f\in C^\infty(U)$ with $U\subset \CC^n$, $f$ is holomorphic if and only if $\partial f=0$. 
\end{theorem}

Define the \textbf{holomorphic tangent bundle} as the vector subbundle generated by the $\{\partial/\partial z_j\}$. A \textbf{differential form} of degree $k$ (i.e. a $k$-form) is a smooth section of $\bigwedge^k T^*(M)$. That is, if $p\in M$, a $k$-form $\beta$ gives an alternating multilinear form $\bigoplus_{i=1}^l=k T_p(M)\to \RR$. We then get a sheaf of $k$-forms on a smooth manifold $M$, which we denote $\Omega_M^k$ (and $\Omega^0=\mathcal{O}_M$). 

\subsection{de Rham cohomology}

A \textbf{differential form} of degree $k$ is a $C^\infty$-global section of $\bigwedge^kT^*=\Omega^k$. At each $p\in M$, it is an alternating multilinear form $T_p(M)\times\dots\times T_p(M)\rightarrow \mathbb{R}$. The set of $k$-forms on $M$ can be denoted as $\Omega^k(M)$, $A^k(M)$, $\Gamma(\Omega^k, M)$ or $H^0(M,\Omega^k)$. $d:\Omega^k\rightarrow\Omega^{k+1}$ is the unique $\mathbb{R}$-linear form such that: (1) $\forall f\in\Omega^0(M)=C^\infty(M)$, $df$ is the total differential, and $dd(f)=0$; (2) $d(\alpha\wedge\beta)=d\alpha\wedge\beta+(-1)^{deg(\alpha)}\alpha\wedge d\beta$. From these one can write down $d$ in local coordinate as $d(f dx_{i_1}\wedge\dots\wedge dx_{i_k})=df\wedge(dx_{i_1}\wedge\dots\wedge dx_{i_k})$, and verify that for any differential form $\alpha$, $dd\alpha=0$. Hence we have the de Rham cochain complex:
$$0\rightarrow\Omega^0(M)\xrightarrow{d}\Omega^1(M)\xrightarrow{d}\dots$$
whose cohomology is called the de Rham cohomology $H^*_{DR}(M)$. 

\begin{lemma}[Poincar\'e Lemma] $U\subset\mathbb{R}^n$ is contractible, then $H^k_{DR}(U)=0$ for $k\geq 1$.
\end{lemma}
	
This shows that $0\rightarrow\underline{\mathbb{R}}\xrightarrow{d}\Omega^0\xrightarrow{d}\Omega^1\xrightarrow{d}\dots$ is an exact sequence of sheaves. As a consequence, we have:
\begin{theorem}[de Rham theorem]
	If $M$ is a smooth manifold, $H^*_{DR}(M)=\check{H}^p(M,\underline{\mathbb{R}})$.
\end{theorem}

\subsection{Dolbeault cohomology}

Use the complex structure one can have splitting $T^*_p={T^*}'_p+{T^*}{''}_p$, where ${T^*}'$ is spanned by $dz_j$ and ${T^*}{''}$ is spanned by $\overline{dz_j}$. This gives spitting $\bigwedge^nT^*_p=\bigoplus_{p+q=n}\bigwedge^p({T^*}')\otimes \bigwedge^q({T^*}{''})$. A smooth section of $\bigwedge^p({T^*}')\otimes \bigwedge^q({T^*}{''})=\Omega^{p,q}$ is called a $(p,q)$-form. The set of smooth $(p,q)$ forms can be denoted by $A^{p,q}(M)$, $\Omega^{p,q}(M)$, $\Gamma(\Omega^{p,q},M)$ or $H^0(M,\Omega^{p,q})$. The decomposition of $\Omega^n$ induces a decomposition $d=\partial+\overline{\partial}$, where $\partial: \Omega^{p,q}\rightarrow\Omega^{p+1,q}$, $\overline{\partial}:\Omega^{p,q}\rightarrow\Omega^{p,q+1}$. By writing them down in local coordinates one can check $\overline{\partial}^2=0$, hence there is a chain complex:
$$0\rightarrow\Omega^{p,0}\xrightarrow{\partial}\Omega^{p,1}\xrightarrow{\partial}\Omega^{p,2}\dots$$
Its cohomology is denoted as $H^{p,*}_\partial(M)$. 
The $\overline{\partial}$-Poincar\'e lemma says that on a polydisc $D$, $H^{p,q}_\partial(D)=0$ when $q\geq 1$. Hence, let $\Omega^p_{hol}$ be the sheaf of holomorphic $p$-forms, then $0\rightarrow\Omega^p_{hol}\xhookrightarrow{}\Omega^{p,0}\xrightarrow{\overline{\partial}}\dots$ is exact, therefore we have $H^{p,q}_{\overline{\partial}}=H^q(M,\Omega^p_{hol})$. \\

\begin{example}
When $q\geq\dim(M)$, $H^{0,q}(M)=H^q(M,\mathcal{O}_M)=0$. When $q\geq 1$, $H^q(\mathbb{C}^n,\mathcal{O})=0$. When $D$ is a polydisc in $\mathbb{C}^n$, $H^{p,0}(D)=H^0(D,\mathcal{O}_D)\oplus\wedge^p\mathbb{C}^n$. 	
\end{example}

\subsection{Hodge decomposition}
The decomposition on complex tori can be described as follows: let $X=V/\Lambda$ be a complex torus, $e_1,\dots e_g$ is a complex bases of $V$ and the corresponding coordinate functions are $v_1,\dots v_g$. Then $H^n(X)=IF^n(X)=\bigwedge^n span\{dv_j,\overline{dv_j}\}=\bigoplus_{p+q=n}IF^{p,q}(X)$, where $IF^{p,q}$ is formed by the $\wedge$ of $p$ $dv_j$ and $q$ $\overline{dv_j}$.\\
 
In general, we have: 
\begin{theorem}
	If $X$ is a compact complex manifold with a Euclidean, or K\"ahler metric (e.g. smooth complex projective varieties), then $H^n(M)=\bigoplus_{p+q=n}H^{p,q}$, and $H^{p,q}=\overline{H^{q,p}}$.
\end{theorem}

Let $\omega$ be the K\"ahler form on $X$, then the volume form is $v={1\over n!}\bigwedge^n\omega$, where $n=\dim(X)$.