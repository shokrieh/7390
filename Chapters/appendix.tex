% !TEX root = 7390.tex




\section{Appendix}\label{Chapters/appendix}

\subsection{Poincar\'{e} lemmas, De Rham cohomology, Dolbeault cohomology}

If we are over $\RR$ we have the classical Poincar\'{e} lemma, which lets us get the connection between De Rham cohomology and singular cohomology. If we're over $\CC$, we have a $\bar{\partial}$-Poincar\'{e} lemma which tells us something about Dolbeault cohomology. We will say things about these, and then a bit more about Hodge theory (and what is the major simplification for abelian varieties versus Kalher manifolds).

\subsubsection{Classical Poincar\'{e} lemma and De Rham cohomology} 
Let us look at $\CC^n$, which is $\RR^{2n}$ as a real manifold (or more generally any smooth manifold $M$). Let $T_0^*(\CC^n)$ (or $T_z^*(M)$) be the cotangent space, and pick the usual basis $\{dx_j,dy_j\}_{j=1}^n$. Over $\CC$ we will also want to work with the complex basis $\{dz_j,d\bar{z_j}\}_{j=1}^n$ with $dz_j=dx_j+idy_j$ and $d\bar{z_j}=dx_j-idy_j$. For the tangent space $T_0 \CC^n$ or $T_z M$, let $\{\partial/\partial z_j,\partial/\partial \bar{z_j}\}$ denote the dual basis. One can easily check that 
$$\frac{\partial}{\partial z_i}=\frac{1}{2}\left(\frac{\partial}{\partial x_i}-i\frac{\partial}{\partial y_i}\right),\ \ \ \frac{\partial}{\partial \bar{z_i}}=\frac{1}{2}\left(\frac{\partial}{\partial x_i}+i\frac{\partial}{\partial y_i}\right).$$

Remark: Over $\CC$, the Cauchy-Riemann equations tell us that if $f\in C^\infty(U)$ then $f$ is holomorphic if and only if $\partial f/\partial \bar{z}=0$. 

Given $f\in C^\infty(U)$ with $U\subset \CC^n$, define the \textbf{total differential} as 
$$df=\partial f+\bar{\partial}f=\sum_{j=1}^n \frac{\partial f}{\partial z_j}dz_i+\sum_{j=1}^n \frac{\partial f}{\partial \bar{z_j}}d\bar{z_j}.$$

\begin{theorem}
For $f\in C^\infty(U)$ with $U\subset \CC^n$, $f$ is holomorphic if and only if $\partial f=0$. 
\end{theorem}

A \textbf{differential form} of degree $k$ is a $C^\infty$-global section of $\bigwedge^kT^*=\Omega^k$. At each $p\in M$, it is an alternating multilinear form $T_p(M)\times\dots\times T_p(M)\rightarrow \mathbb{R}$. The set of $k$-forms on $M$ can be denoted as $\Omega^k(M)$, $A^k(M)$, $\Gamma(\Omega^k, M)$ or $H^0(M,\Omega^k)$. The differential maps $d:\Omega^k(M)\rightarrow\Omega^{k+1}(M)$ can be defined explicitly on coordinates, or alternatively characterized as the unique collection of $\mathbb{R}$-linear maps satisfying: 
\begin{enumerate}
\item $\forall f\in\Omega^0(M)=C^\infty(M)$, $df$ is the total differential of $f$, and $d(df)=0$;
\item We have $d(\alpha\wedge\beta)=d\alpha\wedge\beta+(-1)^{\deg(\alpha)}\alpha\wedge d\beta$.
\end{enumerate}


From these one can write down $d$ in local coordinate as $d(f dx_{i_1}\wedge\dots\wedge dx_{i_k})=df\wedge(dx_{i_1}\wedge\dots\wedge dx_{i_k})$, and verify that for any differential form $\alpha$, $dd\alpha=0$. Hence we have the de Rham cochain complex:
$$0\rightarrow\Omega^0(M)\xrightarrow{d}\Omega^1(M)\xrightarrow{d}\dots$$
whose cohomology 
$$H^p_{dR}(M)=\frac{\ker(d:\Omega^p(M)\to \Omega^{p+1}(M))}{Im(d:\Omega^{p-1}(M)\to \Omega^p(M))}$$ 
is called the \textbf{de Rham cohomology}. Elements of this kernel are called \textbf{closed} and elements of the image are called \textbf{exact}. (Example to keep in mind: If $S^1$ is the circle we have a closed form $d\theta$ which is not exact.) 
Note that $\Omega^k(M)=0$ for $k>\dim M$.  

\begin{lemma}[Poincar\'e Lemma] If $U\subset\mathbb{R}^n$ is contractible, then $H^p_{dR}(U)=0$ for $p\geq 1$.
\end{lemma}
	
This shows that $0\rightarrow\underline{\mathbb{R}}\xrightarrow{d}\Omega^0\xrightarrow{d}\Omega^1\xrightarrow{d}\dots$ is an exact sequence of sheaves. As a consequence, we have:
\begin{theorem}[de Rham theorem]
	If $M$ is a smooth manifold, then $\check{H}^p(M,\underline{\mathbb{R}})$ (which is our usual singular cohomology) is isomorphic to $H^p_{dR}(M)$. 
\end{theorem}

The idea of the proof is to break the above exact sequence up into short exact sequences in the standard way, then use the corresponding long exact sequences on cohomology.

\subsubsection{$\overline{\partial}$ Dolbeault cohomology}

Now we want to start using the complex structure on a complex manifold. Using holomorphic charts, we locally have differentials $\{dz_i,d\overline{z}_i\}$ spanning the cotangent space, with dual elements $\{\partial/\partial z_i,\partial/\partial \overline{z}_i\}$ spanning the tangent space. Using these we have the following splitting
$$T^*_p(M)={T^*}'_p(M)+{T^*}{''}_p(M),$$ 
where ${T^*}'$ is spanned by $dz_j$ and ${T^*}{''}$ is spanned by $\overline{dz_j}$. This gives a direct sum decomposition 
$$\bigwedge^nT^*_p(M)=\bigoplus_{p+q=n}\left(\bigwedge^p {T^*_p(M)}'\right)\otimes \left(\bigwedge^p {T^*_p}(M){''}\right).$$ 

with the $(p,q)$ part spanned by things of the form $dz_I\wedge d\overline{z}_J$ with $|I|=p$ and $|J|=q$. Using this decomposition, define a sheaf $\Omega^{p,q}$ of $C^\infty(p,q)$ forms: we set

$$\Omega^{p,q}(M)=\left\{ \varphi \in \Omega^n(M): \forall z, \varphi(z)\in \left(\bigwedge^p {T^*_p(M)}'\right)\otimes \left(\bigwedge^p {T^*_p}(M){''}\right)\right\}.$$

It is also denoted by $A^{p,q}(M)$, $\Gamma(\Omega^{p,q},M)$ or $H^0(M,\Omega^{p,q})$. This gives a filtration $\Omega^n(M)=\bigoplus_{p+q=n} \Omega^{p,q}(M)$.

What happens when we apply $d$ to something in $\Omega^{p,q}(M)$? We can see it maps into $\Omega^{p+1,q}(M)\oplus \Omega^{p,q+1}(M)$ because $\varphi(z)$ is in 
$$\left(\bigwedge^p {T^*_p(M)}'\right)\otimes \left(\bigwedge^p {T^*_p}(M){''}\right)\wedge T_z^*(M).$$


The decomposition of $\Omega^n$ induces a decomposition $d=\partial+\overline{\partial}$, where $\partial: \Omega^{p,q}\rightarrow\Omega^{p+1,q}$ and $\overline{\partial}:\Omega^{p,q}\rightarrow\Omega^{p,q+1}$. By writing them down in local coordinates one can check that $\overline{\partial}^2=0$, hence there is a cochain complex:
$$0\rightarrow\Omega^{p,0}(M)\xrightarrow{\partial}\Omega^{p,1}(M)\xrightarrow{\partial}\Omega^{p,2}(M)\cdots$$
and we define the \textbf{Dolbeault cohomology} as
$$H^{p,q}_{\overline{\partial}}(M)=\frac{\ker(\overline{\partial}: \Omega^{p,q}(M)\to \Omega^{p,q+1}(M))}{Im(\overline{\partial}:\Omega^{p,q-1}(M)\to \Omega^{p,q}(M))}$$

Now, we want to relate this to some other sheaf cohomology; our exact-seqeunce-based proof of the de Rham theorem suggests what we need. The 
\emph{$\overline{\partial}$-Poincar\'e lemma} says that on a polydisc $D$ in $\CC^n$ (a product of discs in $\CC$), $H^{p,q}_{\overline{\partial}}(D)=0$ for all $q\geq 1$. (Poincar\'{e} was studying the problem that if $g\in C^\infty(D)$ for $D\subset \CC$ then he wanted to find $f$ with $\partial f\partial \overline{z}=g$ which can be solved on a slightly smaller disc.)

Hence, as an analogue of the de Rham theorem, by letting $\Omega^p_{hol}$ be the sheaf of holomorphic $p$-forms, Poincar\'{e} lemma tells us that we have an exact sequence
$$0 \rightarrow\Omega^p_{\rm hol} \xhookrightarrow{}\Omega^{p,0} \xrightarrow{\overline{\partial}}\dots$$ 
with $\Omega^p_{hol}\to \Omega^{p,0}$ the inclusion and the other maps $\overline{\partial}$. Note that $\Omega^p_{hol}$ is indeed the kernel of $\overline{\partial}$ on $\Omega^p_{hol}$. By the same argument from our sheaf proof of the de Rham theorem we can prove the Dolbeault theorem:

\begin{theorem}[Dolbeault theorem]
$$H^{p,q}_{\overline{\partial}}(M)\isom H^q(M,\Omega^p_{hol})$$
for $p,q\geq 0$. 
\end{theorem}


\begin{example}
\noindent
\begin{itemize}
\item When $q\geq\dim(M)$, $H^{0,q}(M)=H^q(M,\mathcal{O}_M)=0$. (So if $q\geq \dim(M)$ both are zero.) 

\item When $q\geq 1$, $H^q(\mathbb{C}^n,\mathcal{O}_{\CC^n})=0$ by the above example and the $\overline{\partial}$-Poincar\'{e} lemma. 
\item When $D$ is a polydisc in $\mathbb{C}^n$, $H^{p,0}_{\overline{\partial}}(D,\Omega^p_D)=H^0(D,\mathcal{O}_D)\otimes\Omega^p_D$ is usually nontrivial, so the $q>1$ hypothesis in the Poincar\'{e} lemma matters! 	
\end{itemize}
\end{example}

\subsection{Hodge decomposition}
In this section, we will talk about the decomposition on complex tori. Assume $X=V/\Lambda$ is a complex torus. Let $e_1,\dots e_g$ be a complex basis of $V$ and $v_1,\cdots, v_g$ the corresponding coordinate functions. 
Then $H^n(X,\CC)$ is isomorphic to the set 
$$IF^n(X)=\bigoplus_{p+q=n} IF^{p,q}(X)$$
where $IF^{p,q}(X)$ (the ``invariant forms'') are the things of the form 
$$\sum_{|I|=p,|J|=q} a_{IJ}dv_I\wedge d\overline{v}_J$$
for $a_{IJ}\in \CC$. Note that because we are on a torus, it is easy to write down this decomposition.

 
For a more general (compact) $X$, we have the following more general theory: 
\begin{theorem}
	If $X$ is a compact complex manifold with a ``nice'' metric, then we have 
	$$H^n(X , \CC)=\bigoplus_{p+q=n}H^{p,q}(X)$$ 
where $H^{p,q}(X)$ is something that is isomorphic to $H^{p,q}_{\overline{\partial}}(X)$ and is isomorphic to the space ot harmonic forms. Also we have	
$$H^{p,q}=\overline{H^{q,p}}.$$
\end{theorem}


In the above theorem, ``nice'' metric means a Euclidean metric or a ``degree $2$ approximation'' of one, i.e. K\"{a}hler metric. For example, $X=V/\Lambda$ has the Euclidean metric and any complex projective variety has a K\"{a}hler metric (since $\PP^n$ has a Fubini-Study metric). For our situation of $X=V / \Lambda$, one can show the $IF^{p,q}(X)$ we wrote down is isomorphic to $H^{p,q}(X)$ is isomorphic to  
$H^{p,q}_{\overline{\delta}}(X)\isom H^{p+q}_d(X)\isom H^q(X,\Omega^p)$. (These are all hard; once we have shown these, we can then show that in the case of a complex torus, the harmonic things are just hte invariant forms and we recover what we had above.)

What is in the background of these isomorphisms? (We only consider the case $X=V/\Lambda$, but the following also contains all of the ideas we need in general.) We start with our Euclidean metric $ds^2=\Sigma dv_i\otimes d\overline{v}_i$. This has an associated $(1,1)$-form 
$$\omega=-\frac{1}{2}Im(ds^2)=\frac{i}{2}\sum_{i=1}^g dv_i\wedge d\overline{v}_i.$$
Then we get a volume form 
$$dv=\frac{1}{g!}\bigwedge^g \omega = (-1)^{\binom{g}{2}}\left(\frac{i}{2}\right)^g(dv_1\wedge d\overline{v}_1\wedge dv_2\wedge \cdots).$$

Now that we have a volume form we can define an inner product on $\Omega^{p,q}(M)$ (but not complete, so not a Hilbert space) by 
$$(\varphi,\psi)=\sum_{|I|=p,|J|=q} \int_X \varphi_{IJ}\overline{\psi}_{IJ}dv$$
for $\varphi=\sum \varphi_{IJ}dv_I\wedge \overline{v}_J$ and similarly for $\psi$. 
This makes it into a pre-Hilbert space (i.e. a non-complete inner product space) and we can then define an adjoint map $\overline{\delta}$ of $\overline{\partial}$ satisfying $(\varphi,\overline{\partial}\psi)=(p\overline{\varphi},\psi)$. Then the \textbf{Laplace-Beltrami operator} is given by
$$\Delta=\overline{\partial\delta}+\overline{\delta\partial}:\Omega^{p,q}(M)\to \Omega^{p,q}(M).$$
Note that $\Delta$ can also be written as $(\overline{\delta}+\overline{\partial})^2$.
In our situation in coordinates we can compute 
$$\Delta(\varphi dv_I\wedge dv_J)=-\sum_i \frac{\partial^2 \varphi}{\partial v_i\partial \overline{v}_i}(dv_I\wedge dv_J),$$
and so this is really the usual Laplacian.

The main point of this formal setup is that we can very easily show that a closed form $\psi$ whose ``norm'' $(\psi,\psi)$ is minimal in its class (in the de Rham or Dolbeault cohomology) is the unique solution in that class to $\overline{\delta}\psi=0$. Note that $\overline{\partial}\psi=0$ and $\overline{\delta}\psi=0$. Therefore $\Delta \psi=0$ and $\psi$ is Harmonic. (The converse is easy to prove too.) Therefore, we conclude that elements of $H^{p,q}(X)$, which are Harmonic $(p,q)$-forms, are unique representatives for classes in $H^{p,q}_{\overline{\partial}}(X)$. In order to understand these Harmonic forms, we need to solve partial differential equations. And there are two operators that can help solving these equations:
\begin{enumerate}
\item $H:\Omega^{p,q}(M)\to \Omega^{p,q}(M)$ which is defined by
$$\varphi dv_I\wedge d\overline{v}_J\mapsto \left(\frac{1}{\text{vol}(X)}\int_X \varphi dV\right)dv_I\wedge d\overline{v}_J.$$
This projects any form onto an invariant thing (certainly satisfies $H^2=H$) and in the abelian variety cse this projects onto $IF^{p,q}(M)$. 

\item The second operator is $G$, a Green's function for $\Delta$ such that it is an inverse to the extent we can have: $\Delta G=G\Delta = 1-H$ and $HG=GH=0$. In general, such a $G$ is hard to find! (For general K\"{a}hler manifolds, one can use the Sobolev lemma. See Griffiths and Harris Chapter $0$. [farbod cite]) For complex tori, we are okay because such a $G$ has a formula given in terms of Fourier analysis: Let $\varphi$ be a function on $X$, then we can lift it to $\tilde{\varphi}$ a periodic function on $V\isom \CC^g$. For such a $\tilde{\varphi}$, we have a fourier expansion and $G\tilde{\varphi}$ is just the same expression with renormalized coefficients. 

Once we have this, we can then prove the Hodge theorem, that every class has a unique harmonic representative (ultimately the hard part of the theory is showing that we have enough harmonic forms!) because for a form $\varphi$, we can explicitly write down
$$\varphi=H\varphi + \overline{\delta}\overline{\partial}\varphi+\overline{\partial}\overline{\delta}\varphi.$$ as our Hodge decomposition.
\end{enumerate}

\subsection{Divisors v.s. line bundles}
Let $(X,\mathcal{O}_X)$ ($\mathcal{O}_X$ is the holomorphic structural sheaf of $X$) be a complex $g$-dimensional manifold (which is not necessarily compact). Everything has algebraic analogue as I will remark. (Refer to chapter 2 and also $\S\S 4-6$ in Hartshorne.) 

Let $\mathcal{K}_X$ be the sheaf (of rings) of meromorphic functions. Algebraically, $\mathcal{K}_X(\mathcal{U})$ is $\mathcal{O}_X(\mathcal{U})$ localized by the multiplicative system of ``non-zero divisors''. This defines a presheaf that will be sheaffified. If $X$ is ``integral'', then $\mathcal{K}_X(\mathcal{U})$ is simply the fraction field of $\mathcal{O}_X(\mathcal{U})$.

We have an exact sequence of sheaves 
\[
0\to \mathcal{O}_X^\times \overset{i}{\to} \mathcal{K}_X^\times \overset{j}{\to} \mathcal{K}_X^\times/\mathcal{O}_X^\times \to 0.
\]
where $\mathcal{K}_X^\times$ is the sheaf of multiplicative subgroups of $\mathcal{K}_X$. The associated long exact sequence of cohomology is then given by 
\[
H^0(X,\mathcal{K_X^\times})\overset{j_*}{\to } H^0(X,\mathcal{K}_X^\times/\mathcal{O}_X^\times) \overset{\delta}{\to} H^1(X,\mathcal{O}_X^\times)\to \cdots
\]