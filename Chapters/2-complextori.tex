% !TEX root = 7390.tex




\section{Complex Tori}\label{Chapters/2-complextori}

\subsection{Some GAGA principles}
Algebraic varieties vs. (complex) analytic spaces. 

If $X$ is an algebraic variety over $\CC$, we can associate $X^{an}$ which is a complex analytic space to it, by passing to a complex analytification; if $X$ is locally described by some set of equations in affine space, we can pass that open set the zero locus as a subset of $\CC^n$ (with its usual topology) and glue. 
Here are some facts:

\begin{itemize}
\item 
This construction is always functorial: an algebraic map $X\to Y$ can always be lifted to a holomorphic map $X^{an}\to Y^{an}$. 
\item $X$ is proper/complete if and only if $X^{an}$ is compact. 
\item $X$ is smooth/connected if and only if $X{an}$ is smooth/connected. 


\item A complex analytic space $\mathfrak{X}$ is called \textbf{algebraic/algebraizable} if there exists a variety $X/\CC$ such that $\mathfrak{X}\isom X^{an}$. (Last time we explicitly showed that $\CC/\Lambda$ is algebraic via the Weierstrass $\wp$-functions.)
\end{itemize}

\subsection{Vector bundles and associated locally free sheaves}

If $L$ is a vector bundle on $X$, then it passes to an analytic vector bundle $L^{an}$ on $X^{an}$. 
This is functorial in the sense that if $f:F\to G$ is a morphism of vector bundles it passes to $f^{an}:F^{an}\to G^{an}$. It is \emph{not} true that all holomorphic vector bundles over $X^{an}$ are algebraizable! But we have:

\begin{theorem}[Serre]
\noindent 
\begin{enumerate}
\item Suppose $X$ is a proper (complete) algebraic variety over $\CC$. If there exists a holomorphic coherent sheaf $\mathcal{F}\to X^{an}$, then there exists unique algebraic coherent sheaf $F$ over $X$ such that $F^{an}=\mathcal{F}$.
\item If there exists $\mathfrak{F}:\mathcal{F}\to \mathcal{G}$ homomorphism of holomorphic coherent sheaves on $X^{an}$, then there exists unique $f:F\to G$ such that $f^{an}=\mathfrak{f}$.
\end{enumerate}
\end{theorem}

Define $H^i(X,L)$ as the $i$-th cohomology group with values in the locally free sheaf $L$. 

\begin{theorem}[Serre]
Let $X$ be a complete algebraic variety over $\CC$ and $F$ a coherent sheaf on $X$. Then the \emph{natural maps} 
$$H^i(X,F)\to H^i(X^{an},F^{an})$$
are isomorphisms of $\CC$-vector spaces.
\end{theorem}

\subsection{Complex tori}
Let $V$ be a vector space over $\CC$, and $\Lambda\subset V$ a lattice (full rank discrete subgroup). We have $\Lambda$ act naturally on $V$ by addition; and then the quotient $X=V/\Lambda$ is a complex torus. 

Some facts about a complex torus:
\begin{itemize}
\item it is a complex manifold.
\item it inherits the structure of a complex Lie group over $\CC$.
\item it is compact (because $\Lambda$ is a maximal rank lattice).
\item it is an abelian complex Lie group.
\item meromorphic functions on $X$ correspond to meromorphic $\Lambda$-periodic functions on $V$. 
\end{itemize}

Loosely speaking, 
an (complex analytic) \emph{abelian variety} is a complex torus with ``sufficiently many'' (enough to give a closed embedding to a projective space) meromorphic functions. We will see that this is exactly what makes $X$ algebraizable and thus an algebraic abelian variety.

\subsection{Compactness implies abelian}

\begin{theorem}
Any connected \emph{compact} complex Lie group $X$ is a complex torus.
\end{theorem}

\begin{proof}
First, $X$ is abelian and so the commutator map $\Phi(x,y)=xyx^{-1}y^{-1}$ is continuous. Let $U$ be any neighbourhood of the identity element $1$, for $x\in X$, define open neighbourhoods $V_x,\tilde{V_x}$ such that $x\in V_x$, $1\in \tilde{V_x}$ and $\Phi(V_x,\tilde{V_x})\subset U$. (This can be done since $\Phi(x,1)=1$ and $\Phi$ is continuous.) 

So we have
$X=\cup_{x\in X} V_x$
and by compactness, there exist $x_1,\cdots, x_r\in X$ such that
$$X=\bigcup_{x\in \{x_1,\cdots,x_r\}} V_x.$$

Let $W=\cap_{x_1,\cdots,x_r} \tilde{V_x}$, which is a non-empty open neighbourhood of $1$. So $\Phi(X,W)\subset U$. Since $U$ is arbitrary, we have $\Phi(X,W)=1$. 

Since holomorphic functions on a compact set $X$ which is bounded must be constant, we have
$\Phi(1,y)=1$ for all $y\in W$. Since $W$ is open and non-empty, by connectivity, 
$$\Phi(x,y)=1$$
for all $x,y\in X$. 

Then, if $\pi:V\to X$ is a universal cover, $V$ inherits the structure of a simply connected complex Lie group and thus must be $\CC^g$. Moreover $\pi$ is homomorphic with discrete kernel, and by compactness of $X$ the kernel must be full rank.
\end{proof}

Another proof can be found in B. Conrad's \href{http://math.stanford.edu/~conrad/vigregroup/vigre04/abvaran.pdf}{notes}.



Remarks: Once we have $X=V/\Lambda$, we see that $V$ is a universal cover of $X$. Moreover, $\Lambda=\pi(X,0)$, and since this is already abelian it is isomorphic to $\isom H_1(X,\ZZ)$. 
Since $X$ is locally isomorphic to $V$, we can view $V$ as the tangent space at $0$, $T_0X$; then the covering map $\pi:V=T_0X\to X$ is actually the exponential map.


\subsection{Period matrix}

Given $X=V/\Lambda$, we can associate $\Pi$ a $g\times 2g$ complex matrix: fix $\{e_1,\cdots, e_g\}$ a $\CC$-basis for $V$ and $\{\lambda_1,\cdots,\lambda_{2g}\}$ a $\ZZ$-generator set for $\Lambda$. Define $\lambda_{ji}$ such that
$$\lambda_j=\sum \lambda_{ji} e_i.$$

Then the \textbf{period matrix} of $X$ is given by 
$$\Pi:=
\left(
\begin{array}{ccc}
\lambda_{1,1} & \cdots & \lambda_{1,2g}\\
\vdots & \ddots & \vdots\\
\lambda_{g,1} & \cdots & \lambda_{g,2g}\\
\end{array}
\right).$$

Clearly, $\Pi$ determines $X$ but it depends on the choices. 

Question: Given $\Pi\in M_{g\times 2g}(\CC)$, is there a complex torus such that $\Pi$ is the period matrix of $X$? 

\begin{theorem}
Let $P=\left(
\begin{array}{c}
\Pi\\
\overline{\Pi}\\
\end{array}
\right)_{2g\times 2g}$, where $\overline{\Pi}$ denote the complex conjugate matrix of $\Pi$. Then $\Pi$ is the period matrix for some $\CC^g/\Lambda$ if and only if $P$ is nonsingular. 
\end{theorem}

\begin{proof}
$\Pi$ is a period matrix if and only if the columns of $\Pi$ are $\RR$-linearly independent. 
\end{proof}

\subsection{Holomorphic maps, homomorphism and isogenies}
Suppose $X=V/\Lambda$ and $X'=V'/\Lambda'$ with dimensions $g$ and $g'$ respectively. We want to study holomorphic maps $f:X\to X'$. There are two special examples:

\begin{enumerate}
\item homomorphisms (holomorphic and respect group structure); and
\item translations (maps $X\to X$ by $x\mapsto x+x_0$ for some $x_0\in X$).
\end{enumerate}

The surprising thing is that that's \emph{all}! 

\begin{theorem}
Suppose $h:X\to X'$ is a holomorphic map between complex tori. Then 
\begin{enumerate}
\item there exists a unique homomorphism $f:X\to X'$ such that $h=t_{h(0)}\circ f$. That is,
$$h(x)=f(x)+h(0)$$
for all $x$.
\item There exists a unique $\CC$-linear map $F:V\to V'$ with $F(\Lambda)\subset \Lambda'$ inducing $f$. 
\end{enumerate}
\end{theorem}


\begin{proof}
Let $f:=t_{-h(0)}\circ h$. Then we can lift $f\circ \pi: V\to X$ to $F:V\to V'$ where $V'$ is the universal cover of $X'$. Then $F$ is holomorphic and satisfies $F(0)=0$. $F$ is a $\CC$-linear map: fix $\lambda\in \Lambda$, by construction,
$$F(v+\lambda)-F(v)\in \Lambda'$$
and so by continuity it's constant. Therefore,
$$F(v+\lambda)= F(v)+F(\lambda)$$
for all $v\in V, \lambda\in \Lambda$. We skip the remaining details.
\end{proof}


\subsection{Hom-sets}
Let $\Hom(X,X')$ be the set of all homomorphisms $f:X\to X'$. It is an abelian group. If $X=X'$, then we can define $\End(X):=\Hom(X,X')$. In this case, $\End(X)$ is actually a ring, where multiplication is given by composing endomorphisms. The above theorem gives us the following corollary:

\begin{corollary}
We have injective homomorphisms:
\begin{enumerate}
\item $\rho_{an}:\Hom(X,X')\to \Hom_\CC (V,V')$ given by $f\mapsto F$; and
\item $\rho_{int}: \Hom(X,X')\to \Hom_\ZZ(\Lambda,\Lambda')$ given by $f\mapsto F|_\Lambda$.
\end{enumerate}
\end{corollary}

Note that both of these homomorphisms respect endomorphism ring structures if $X=X'$: $\rho_*(f'\circ f)=\rho_*(f')\circ \rho_*(f)$, where $*=an, int$.

\begin{theorem}
$\Hom(X,X')\isom \ZZ^m$ for some $m\leq 4gg'$.
\end{theorem}

\begin{proof}
Use the second isomorphism in the corollary, since $\Lambda\isom \ZZ^{2g}$ and $\Lambda'\isom \ZZ^{2g'}$, so  $\Hom_\ZZ(\Lambda,\Lambda')\isom \ZZ^{4gg'}$ and $\Hom(X,X')$ embeds in this.
\end{proof}

How do these relate to period matrices?
Let $\Pi$ and $\Pi'$ be the period matrix for $X$ and $X'$ respectively. If we have $f:X\to X'$, then by picking bases we get that $\rho_{an}(f):V\to V'$ is given by some $A\in M_{g'\times g}(\CC)$ and $\rho_{int}(f):\Lambda\to \Lambda'$ given by some $R\in M_{2g'\times 2g}(\ZZ)$. Then the condition $F(\Lambda)\subset \Lambda'$ means $A\Pi=\Pi'R$. (The converse is also true: given four matrices with this property, then they correspond to a morphism between complex tori.) 

What if $X=X'$? In this case, we can get 
$$\left(
\begin{array}{cc}
A & 0\\
0 & \overline{A}\\
\end{array}\right)
\left(
\begin{array}{c}
\Pi\\
\overline{\Pi}\\
\end{array}
\right)=\left(
\begin{array}{c}
\Pi\\
\overline{\Pi}\\
\end{array}
\right)R$$

and thus 
$\rho_{int}\otimes 1\isom \rho_{an}\oplus \bar{\rho_{an}}$ in $\End(X)\otimes_\ZZ \CC$. 

\subsection{Kernels and Images}
\begin{lemma}
Given a homomorphism $f:X\to X'$. 
\begin{enumerate}
\item $Im(f)$ is a complex subtorus of $X'$.
\item $\ker(f)$ is a closed subgroup of $X$ with finitely many component. The connected component of $1=id$ is a complex torus. 
\end{enumerate}
\end{lemma}

The proof is fairly easy; for part (b) we're claiming that we have an extension 
$$1\to X_0 \to G \to \Gamma \to 1$$
with $X_0$ a complex torus and $\Gamma$ a finite abelian group. It is a good exercise to describe $\Gamma$ as a direct sum of cyclic groups in terms of $\Pi,\Pi',A,R$ (need to compute a Smith normal form somewhere).


\subsection{Isogenies} 

A homomorphism
$f:X\to X'$ is called an \textbf{isogeny} if $f$ is surjective with finite kernel. Equivalently, $f$ is surjective and $\dim(X)=\dim(X')$. 

\begin{example}[Essential example]
Suppose $X=V/\Lambda$ is a complex torus and $\Gamma\subset X$ is a finite subgroup. Then $X/\Gamma=V/\pi^{-1}(\Gamma)$ is a complex torus and $X\to X/\Gamma$ is an isogeny. 
\end{example}

In fact, that's all! It is an easy exercise to show that all isogenies $X\to X/\Gamma$ over $\CC$ are of this form. We also have the following easy lemma:

\begin{lemma}[Stein factorization]
Any surjection $f:X\to X'$ of complex tori factors as a surjection $X\to X/(\ker f)_0$ (a quotient of $X$ by a complex subtorus) and an isogeny $X/(\ker f)_0 \to X'$.
\end{lemma}

Remark: Stein factorization is a special case of a very general result (in complex theory by Stein and others, and in general in EGA III): any proper $f:X\to S$ factors as a map $X\to S'$ proper with connected fibers and $S'\to S$ finite. 
 
For $f\in \Hom(X,X')$, we define $\deg(f)$ to be $|\ker f|$ if this is finite, and $0$ if otherwise. It is easy to check that $\deg(f)=[\Lambda':\rho_{int}(f)\Lambda]$. (Remark: If $X=X'$ then this index is $\deg(\rho_{int}(f))$; note that this determinant is $\geq 0$ since $\rho_{int}\otimes 1 = \rho_{an}\oplus \bar{\rho}_{an}$, and is $0$ if and only if the kernel is infinite.)

\begin{lemma}
Supppose $f:X\to X'$ and $f':X'\to X''$ are isogenies, then $f'\circ f$ is also an isogeny. 
\end{lemma}

\begin{proof}
$\deg(f'\circ f)=\deg(f)\cdot \deg(f')$.
\end{proof}

A very important example is given by the ``multiplication-by-$n$'' map: Let $n\in \ZZ^+$, define 
$n_X:X\to X$ by $x\mapsto nx$. Denote $X[n]:=\ker(n_X)$ the set of $n$-torsions in $A$. Then we have
$$X[n]\isom \frac{\frac{1}{n} \Lambda}{\Lambda}\isom \frac{\Lambda}{n\Lambda}\isom (\ZZ/n)^{2g}.$$
Therefore, $n_X$ has degree $n^{2g}$, and so it is an isogeny.

\begin{corollary}
Complex tori are divisible groups.
\end{corollary}


\begin{example}[Tate module]
Let $\ell$ be a prime number. Define multiplication by $\ell$ maps $X[\ell^{n+1}]\to X[\ell^n]$. Then the \emph{Tate module} is given by 
$$T_\ell(X)=\varprojlim X[\ell^n].$$
In the case where $\Lambda$ is finitely generated, $T_\ell(X)$ is actually isomorphic to $\Lambda\otimes_\ZZ \ZZ_\ell$ and this is a subset of $\Lambda$. 
Note that the definition of $T_\ell(X)$ makes sense over any fields (even if we don't have $\Lambda$ when we are not over $\CC$). Here in our setting it's easy to see that a morphism $X\to X'$ is determined by the induced map $T_\ell(X)\to T_\ell(X')$. Over general fields this is much, much harder! It's the \emph{Tate conjecture} which says that we have 
$$\Hom_{\Gal}(T_\ell(X),T_\ell(X'))\isom \Hom(X,X').$$
This conjecture was only proven over number fields by Faltings as an essential part of his proof of the Mordell conjecture. 
\end{example}



\subsection{Importance of isogenies}

They are ``almost isomorphisms''. Namely, we have:

\begin{theorem}
Let $f:X\to X'$ be an isogeny and $n$ be the exponent of $\ker(f)$. (That is, $nx=0$ for all $x\in \ker(f)$.) Then there exists an isogeny $g:X'\to X$ such that 
$$f\circ g=n_X,\ g\circ f=n_X.$$
Moreover, such a $g$ is unique (up to isomorphism?).
\end{theorem}

\begin{proof}[sketch]
Since $n$ is the exponent of $\ker f$, we have $\ker(f)\subset \ker(n_X)=X[n]$. Then there exists a unique $g:X'\to X$ with $g\circ f = n_X$, defined by $g(x'):=nx$ for some (all) $x$ where $f(x)=x'$. Then use the fact that $\deg(g)\deg(f)=\deg(n_X)$ and that $\deg(f),\deg(n_X)\not= 0$, so $\deg(g)\not= 0$ to get that $g$ is an isogeny; Then we just need to check that $g\circ f = n_X'$. 
\end{proof}

Define $\End_\QQ(X):=\End(X)\otimes \QQ$ and $\Hom_\QQ(X,X'):=\Hom(X,X')\otimes \QQ$. Then the degree function extends to these via
$$\deg(rg):=r^{2g}\cdot \deg(f).$$

\begin{corollary}
\noindent
\begin{enumerate}
\item Isogeny is an equivalence relation.
\item $f\in \End(X)$ is an isogeny if and only if it is invertible in $\End_\QQ(X)$. 
\end{enumerate}
\end{corollary}



\subsection{Cohomology} 

We have a lot of cohomology theories (Betti, de Rham, Dolbeault, Hodge decomposition, \dots).

Betti cohomology is just singular cohomology of $X(\CC)$; if $X=V/\Lambda$, then we have the following facts:

\begin{itemize}
\item $\Lambda=\pi_1(X_0)\isom H_1(X,\ZZ)$.
\item By the universal coefficient theorem, we have $H^1(X,\ZZ)=\Hom(\Lambda,\ZZ)$.
\item If $n\geq 1$, we have a map $\wedge_{i=1}^n H^1(X,\ZZ)\to H^n(X,\ZZ)$ induced by cup product, and this is an isomorphism (follows from Kunneth formula). 
\item Let $Alt^n(\Lambda,\ZZ):=\bigwedge_{i=1}^n \Hom(\Lambda,\ZZ)$ be all the $\ZZ$-valued alternating $n$-forms. Then we have $H^n\isom Alt^n(\Lambda,\ZZ)$. This gives a very explicit way of thinking about cohomology.
\item $H_n(X,\ZZ)$ and $H^n(X,\ZZ)$ are free $\ZZ$-modules of rank $\binom{2g}{n}$.
\item If we set $H^n(X,\CC):= H^n(X,\ZZ)\otimes \CC$, then we have
$$H^n(X,\CC)\isom Alt_\RR^n(V,\CC)=\bigwedge_{i=1}^n\Hom_\RR(\Lambda,\CC)\isom \bigwedge_{i=1}^n H^1(X,\CC),$$ 
and the de Rham theorem tells us 
$H^n(X,\CC)\isom H_{DR}(X)$ where $H_{DR}(X)$ can be explicitly described as a complex vector space of invariant $n$-forms with basis $dx_{i_1}\wedge \cdots \wedge dx_{i_n}$ with $i_1<\cdots <i_n$. 
\end{itemize}

Now, we use the $\CC$-structure (really everthing is true for Kahler manifolds, but proofs and constructions are much more elementary for complex tori). Here we have a very nice decomposition

$$H^n(X,\CC)\isom \bigoplus_{p+q=n} H^q(\Omega_X^p)$$ 

Here, $H^{p,q}(X):=H^q(\Omega_X^p)$ is isomorphic to the Dolbeault cohomology $H^{p,q}(X)$. In general, $H^q(\Omega_X^p)$ can be explicitly described as $\bigwedge^p\Omega \otimes \bigwedge^q\overline{\Omega}$ for $\Omega=\Hom_\CC(V,\CC)$ and $\overline{\Omega}=\Hom_{\overline{\CC}}(V,\CC)$.
Also set $\Omega_X^p:=(\bigwedge^p \Omega)\otimes \mathcal{O}_X$. Right now we are not saying anything about what these vector spaces $H^{p,q}(X)$ are, but we will do that later on. We may also need ot return to this theory to prove for example vanishing results later; We will either do that, or just omit those proofs.

\subsection{Sheaves on a topological space $X$}

Let $\mathcal{O}(X)$ be the category in which objects are open subsets of $X$ and morphisms are inclusions $V\to U$ for $V\subset U$. (It turns out that you can generalize this and allow more general things for morphisms than just inclusions; this is how you get etale cohomology and other things.) Let $\mathcal{C}$ be any other category (for instance, Sets, Abelian groups and $R$-modules); A \textbf{presheaf} is a contravariant functor 
$$F:\mathcal{O}(X)\to \mathcal{C}.$$
(This means for each inclusion map $i:V\subset U$, we have the restriction map given by $rest_{V,U}: F(U)\to F(V)$, and this assignment is functorial.) A \textbf{sheaf} is a presheaf $F$ with some ``locality'' and ``gluing'' properties. Assume $\mathcal{C}$ has products. Then we require for any open cover $\{U_i\}$ of $U\in \mathcal{O}(X)$, we have an exact sequence
$$0\to F(U) \overset{rest}{\to} \prod_i F(U_i)\rightrightarrows \prod_{i,j} F(U_i\cap U_j)$$
(where the two parallel maps are ``restriction to the first index'' and ``restriction to the second index'' respectively.)

\begin{example}
Let $X=\CC$. Then we have the sheaf of holomorphic functions: $F(U)$ is the set of all holomorphic functions $U\to \CC$, and restriction maps are restriction of functions!
\end{example}

Morphisms of sheaves are natural transformations; this gives us the category of sheaves over $X$, $Shf_X$, in which the objects are sheaves on $X$ and morphisms are these.

Next, a \textbf{ringed space} is a topological space $X$ together with a sheaf of rings $\mathcal{O}_X$; we call $\mathcal{O}_X$ the \textbf{structure sheaf} (usually some sheaf of holomorphic functions in this class). 
Define a \textbf{locally ringed space} to be one such that all of the stalks $\mathcal{O}_{X,x}=\varinjlim_{U\ni x} F(U)$ are local rings. We may want to consider sheaves of $\mathcal{O}_X$-modules, that is, each $F(U)$ is a $\mathcal{O}_X(U)$-module respecting restriction maps. 

\subsection{Abelian categories and cohomology}
(Grothendieck's Tohoku paper) There are $4$ main examples of abelian categories to keep in mind:

\begin{enumerate}
\item The category of abelian groups (with homomorphisms).
\item The category of $R$-modules for $R$ a commutative ring (with homomorphisms).
\item The category of $G$-modules for $G$ a group (with $G$-equivariant homomorphisms: $\phi(g\cdot m)=g\cdot \phi(m)$).
\item The category of $\mathcal{O}_X$-modules for $(X,\mathcal{O}_X)$ a ringed space (with morphisms of sheaves the morphisms). 
\end{enumerate}

In general there's an abstract definition of an abelian category; we are not going to say it precisely but roughly it is a category $\mathcal{A}$ in which addition of morphisms, a zero object, kernels and cokernels make sense.

Suppose $F:\mathcal{A}\to \mathcal{B}$ is a covariant functor between abelian categories. If we start with an short exact sequence 
$$0\to A \to B \to C \to 0$$
in $\mathcal{A}$, we can hit it with the functor $F$ and get maps in $\mathcal{B}$, but no guarantee for exactness. We say that $F$ is \textbf{left exact} if for every such short exact sequence we do have
$$0\to F(A)\to F(B)\to F(C)$$ 
exact. 

\begin{example}
Let $\mathcal{A}$ be the category of $R$-module, $\mathcal{B}$ be the category of abelian groups and $D$ be a fixed object. Then $F(-)=\Hom_R(D,-)$ is a left exact covariant functor.
\end{example}

Cohomology lets us study the failure of exactness of 
$$0\to F(A)\to F(B)\to F(C),$$
i.e. the failure of surjectivity of $F(B)\to F(C)$. We want to continue the above exact sequence to the right and write a long exact sequence. In general, there are many ways to do this. But if $\mathcal{A}$ has ``enough injectives'' (a statement that holds for the categories we care about), then there is a ``canonical'' and ``minimal'' (in the sense that there is a universal property) way to do this. There exist unique functors $R^iF:\mathcal{A}\to \mathcal{B}$ for $i\geq 0$ with $R^0F=F$ that give us the following long exact sequence
$$0\to F(A)\to F(B) \to F(C) \overset{c_1}{\to} R^1F(A)\to R^1F(B) \to R^1F(C)\overset{c_2}{\to} R^2F(A) \to \cdots$$
plus satisfying some universal properties (an ``effaceable $\delta$-functor''). These functors are called the \emph{right derived functors}. This is the covariant version; there's a similar one for contravariant functors. In general, it is hard to compute these $R^iF$. However, in the examples that we are interested in, there are easier ways of computing them.

\begin{example}
Let $F(-):=\Hom_R(-,D)$ and $G(-):=\Hom_R(D,-)$ where $D$ is a fixed $R$-module. Both $F$ and $G$ are left-exact functors from $R$-modules to abelian groups, one contravariant and one covariant. More precisely, for any short exact sequence of $R$-modules
$$0\to L \to M \to N \to 0,$$
we have the following exact sequences
$$0\to \Hom_R(N,D)\to \Hom_R(M,D)\to \Hom_R(L,D)$$
and 
$$0\to \Hom_R(D,L)\to \Hom_R(D,M)\to \Hom_R(D,N).$$

In both cases, the derived functors are just the Ext groups $\Ext_R^i(-,D)$ and $\Ext_R^i(D,-)$
which give us the long exact sequences
$$0\to \Hom_R(N,D)\to \Hom_R(M,D)\to \Hom_R(L,D)\to \Ext_R^1(N,D)\to \Ext_R^1(M,D)\to \cdots$$
and 
$$0\to \Hom_R(D,L)\to \Hom_R(D,M)\to \Hom_R(D,N)\to \Ext_R^1(D,L)\to \Ext_R^1(D,M)\to \cdots$$

So $R^i\Hom_R(-,D)=\Ext_R^i(-,D)$ and $R^i\Hom_R(D,-)=\Ext_R^i(D,-)$ (it is a nontrivial fact that these agree!). 
\end{example}

\begin{example}
In the case of $G$-modules, let $A$ be an abelian group $A$ with a $G$ action $\phi: G\to \Aut(A)$, (that is, $A$ is a $\ZZ G$-module) with morphisms respecting $G$-action. Let $A^G$ be the group of $x\in A$ such that $g\cdot x=x$ for all $g\in G$. The functor we want in this case is $F(A)=A^G$ which takes $G$ -modules to abelian groups. Note that this functor is the same as $\Hom_{\ZZ G}(\ZZ,-)$, so its (left exact) derived functors are (abstractly) Ext-functors $\Ext_{\ZZ G}^i(\ZZ,-)$, but this doesn't really give us a good way to compute it. These are better known as the ``group cohomology of $G$ with coefficients in $A$'', denoted $H^i(G,A)$. 

How do we compute this? 
It turns out that $\ZZ$ has a very nice ``standard resolution'' (also called the ``bar resolution''), which is given by 
$$F_n=\bigotimes_{i=0}^n \ZZ G.$$
Using this, we get the following recipe: for $G$ a group and $A$ a $G$-module, we let $C^0(G,A)=A$ and $C^n(G,A)$ be the group of $A$-valued maps on $G^n=G\times \cdots\times G$ for $n\geq 1$. These $C^n$ are called the group of \textbf{$n$-cochains} of $G$ with values in $A$. We also define the \textbf{differential operators} $d_n: C^n(G,A)\to C^{n+1}(G,A)$ by 

\begin{align*}
d_n(f)(g_1,\cdots, g_{n+1}):= 
&\ g_1\cdot f(g_2,\cdots, g_{n+1})\\
&+\Sigma_{i=1}^n (-1)^i f(g_1,\cdots, g_{i-1},g_ig_{i+1},g_{i+2},\cdots, g_{n+1})\\
&+(-1)^{n+1}f(g_1,\cdots, g_n).\\
\end{align*}

The cases of most interest are $n=0,1,2$:
\begin{itemize}
\item when $n=0$, we have $f=a\in C^0(G,A)=A$, and then we get 
$$d_0(f)(g)=g\cdot a - a.$$
\item when $n=1$, then $f$ is a function with one input and we have
$$d_1(f)(g_1,g_2)=g_1\cdot f(g_2)-f(g_1g_2)+f(g_1).$$
\item when $n=2$, then $f$ is a function with two inputs and we have
$$d_2(f)(g,h,k)=g\cdot f(h,k) - f(gh,k)+f(g,hk)- f(g,h).$$
Amazingly, people wrote down this formula correctly before the general theory came out.
\end{itemize}

It can be shown that $d_n\circ d_{n+1}=0$. Thus, we can define the group of \textbf{$n$-cocycles} as $Z^n(G,A):=\ker(d_n)$ for $n\geq 0$ and the group of \textbf{$n$-coboundaries} as $B^n(G,A):=\im(d_{n-1})$ for $n\geq 1$ (or $B^0(G,A)=1$ when $n=0$). Then we define the \textbf{$n$-th cohomology group} as
$$H^n(G,A):=Z^n(G,A)/B^n(G,A).$$
Note that $H^0(G,A)=A^G$. 
\end{example}

\begin{example}
Let $(X,\mathcal{O}_X)$ be a ringed space, and $\mathcal{F}$ a $\mathcal{O}_X$-module. Let $\Gamma(-,X)$ be the ``global sections'' functor, $\mathcal{F}\mapsto \mathcal{F}(X)$, from $\mathbb{O}_X$-modules to $R$-modules for $R=\mathcal{O}_X(X)$. This functor is left-exact (to make sense of this we need to make sure we know what exact sequences of sheaves are - defining ``kernels'' is easy but to define ``images'' we need to sheafify). 

Here there is the Grothedieck (right) derived cohomology functor, $R^i\Gamma(-,X)$, which we denote $H^i(X,-)$ (we also call $H^i(X,\mathcal{F})$ the \emph{sheaf cohomology of $X$ with values in $\mathcal{F}$}); in particular, we have $H^0(X,-)=\Gamma(-,X)$. This is given abstractly by the theory earlier in this section; but like group cohomology, we will really need to work with sheaf cohomology, so we need to a way to compute them explicitly.

Here, we have a more concrete one, which is called the \emph{\v{C}ech cohomology}, $\check{H}^i(X,\mathcal{F})$. In general, the two cohomologies are not equal! Fortunately they are equal in the settings we will be interested in. 

The \v{C}ech cohomology is more explicitly computable, and always gives us a map $\phi: H^i(X,\mathcal{F})\to \check{H}^i(X,\mathcal{F})$. We have:

\begin{enumerate}
\item For $i=0,1$, $\phi$ is an isomorphism.
\item (Grothendieck): If $X$ is a Noetherian, separated scheme and $\mathcal{F}$ is a quasi-coherent $\mathcal{O}_X$-module, then $\phi$ is an isomorphism. 
\item (Godement): If $X$ is a paracompact and Hausdorff topological space, then $\phi$ is an isomorphism.
\end{enumerate}

However, $\phi$ is not an isomorphism in general! Here are some counterexamples:

\begin{itemize}
\item In his T\^{o}hoku paper (p.177), Grothendieck provides a counterexample where $H^2\not= \check{H}^2$, for $X=\Aff^2$ with the Zariski topology and $\mathcal{F}$ comes from taking $\underline{\ZZ}$ and modifying it based on a space $Y$ that is a union of two circles. This example is explicit but the proof is somehow deep!
\item A recent paper of Schr\"{o}er (arxiv post 1309.2524) gives a Hausdorff (not not paracompact) topological space constructed from $2$-dimensional discs that is a counterexample. 
\end{itemize}

\subsubsection{\v{C}ech cohomology}
So how do we define \v{C}ech cohomology? The idea is that if $\mathcal{U}$ is an open cover on $X$, the ``nerve'' of $\mathcal{U}$ approximates $X$. We define a $q$-simplex $\sigma$ of $\mathcal{U}$ as an ordered collection of $q+1$ elements in $\mathcal{U}$ with nonempty intersection that we call $|\sigma|$. Suppose $\sigma=(U_i)$ (for $0\leq i \leq q$), we define $\partial_j \sigma:=(U_i)_{i\not= j}$ and then $\partial\sigma=\Sigma_{j=0}^q (-1)^{j+i}\partial_j\sigma$. Note that $|\sigma|=\cap U_i$. 

We then define \textbf{$q$-cochains} of $\mathcal{U}$ with coefficients in $\mathcal{F}$ to be the set $C^q(\mathcal{U},\mathcal{F})$ of functions $\sigma\mapsto f_\sigma \in \mathcal{F}(|\sigma|)$. Therefore, we get the \textbf{boundary maps} $C^q(\mathcal{U},\mathcal{F})\to C^{q+1}(\mathcal{U},\mathcal{F})$ by 
$$(\delta_q \omega)(\sigma)=\Sigma_{j=0}^{q+1} (-1)^j \text{res}_{|\partial_j \sigma|,|\sigma|} \omega(\partial_j\sigma).$$

One can check that $\delta_{q+1}\circ \delta_q=0$ and hence can define \textbf{cocycles} $Z^q(\mathcal{U},\mathcal{F}):=\ker(\delta_q)$ and \textbf{coboundaries} $B^q(\mathcal{U},\mathcal{F}):=Im(\delta_{q-1})$, and then the \textbf{\v{C}ech cohomology} of $\mathcal{U}$ is given by 
$$\check{H}^q(\mathcal{U},\mathcal{F}):=Z^q(\mathcal{U},\mathcal{F})/B^q(\mathcal{U},\mathcal{F}).$$

So this gives us the cohomology $\check{H}^i(\mathcal{U},\mathcal{F})$ of an open cover $\mathcal{U}$; However, we want a cohomology $\check{H}^i(X,\mathcal{F})$ associated to the whole space $X$! There are two ways to solve this:

\begin{enumerate}
\item If $X$ has a ``good'' cover $\mathcal{U}$ (with all finite intersections of $U_i$ to be contractible), then $\check{H}^i(\mathcal{U},\mathcal{F})$ is canonical.

\item In general, we can define $\check{H}^i(\mathcal{X},\mathcal{F})$ as $\varinjlim_{\mathcal{U}} \check{H}^i(\mathcal{U},\mathcal{F})$. But then we need to make sense of this direct limit. (It might be over an index set that is a proper class?) [fix farbod]
\end{enumerate}

Remark: Let $G$ be a topological group. Let $BG=K(G,1)$ be the \emph{Eilenberg-MacLane space}. (For example, $B\ZZ=S^1$.) Note that $\pi_1=G$ and $\pi_n=0$ for all $n>1$. Then if $A$ is a $G$-module, the sheaf cohomology $H^n(BG,\underline{A})$ (here $\underline{A}$ is the constant sheaf, which is the sheafification of the constant presheaf) is isomorphic to the usual CW complex cohomology $H^n(BG,A)$ and to the group cohomology $H^n(G,A)$, if $A$ has a trivial $G$-action. (If $A$ has a nontrivial $G$-action we can still make sense of this but the $H^n(BG,\underline{A})$ needs to be reinterpreted in terms of ``local coefficient systems''.)
\end{example}

\subsection{Back to complex tori (sort of)}
We want to understand line bundles on a locally ringed space $(X,\mathcal{O}_X)$. For now, we can think of it as a complex manifold or a variety. Let $\mathcal{F}$ be a sheaf. We call $\mathcal{F}$ (globally) \textbf{free} if $\mathcal{F}=\bigoplus_{i=1}^r \mathcal{O}_X$ is the direct sum of copies of the structure sheaf; $r$ is called the \textbf{rank} of $\mathcal{F}$. $\mathcal{F}$ is called \textbf{locally free} if there exists an open cover $\{U_i\}$ such that each $\mathcal{F}|_{U_i}$ is free. (There is a correspondence between locally free sheaves $\mathcal{F}$ of rank $n$ and vector bundles of rank $n$. If $\pi:E\to X$ is a vector bundle, we get a locally free sheaf with $\mathcal{F}(U)$ being the sections of $\pi$ over $U$; conversely, if $\mathcal{F}$ is locally free, we can construct an associated line bundle as $\coprod U_i\times \CC^n$ modulo gluing data.) 

\subsubsection{Line bundles}

Line bundles are vector bundles of rank $1$ (equivalently, locally free sheaves of rank $1$). Our goal is to give a cohomological interpretation on the set of line bundles. Let $\pi:L\to X$ be a line bundle. Let $\{U_\alpha\}$ be an open cover with trivializations given by (holomorphic) $\phi_\alpha: L|_\alpha\isomto U_\alpha\times \CC$ where $L|_\alpha:=\pi^{-1}[U_\alpha]$. Define the \textbf{transition functions} for $L$ with respect to $\{\phi_\alpha\}$ as $g_{\alpha\beta}:U_\alpha\cap U_\beta \to \CC^*$ which are given by 
$$g_{\alpha\beta}(z):=\phi_\alpha\circ \phi_\beta^{-1}|_{L_z}$$
where $L_z:=\{z\}\times \CC$. By itself this is a linear (and hence holomorphic) map on $\{z\}\times \CC$, which is determined by the complex number we are calling $g_{\alpha\beta}(z)$. We check that 
$g_{\alpha\beta}\circ g_{\beta\alpha}=1$ (so it is nonzero) and also $g_{\alpha\beta}\circ g_{\beta\gamma}\circ g_{\gamma\alpha}=1$. Rewriting this latter condition gives a cocycle condition
$$g_{\alpha\beta}g_{\gamma\beta}^{-1}g_{\gamma\alpha}=1.$$

To summarize, a line bundle (trivialized by an open cover $\mathcal{U}$) determines a collection of 
\begin{itemize}
\item holomorphic functions $g_{\alpha\beta}:U_\alpha\cap U_\beta\to \CC^\times$; which satisfy 
\item $g_{\alpha\beta}\circ g_{\beta\alpha}=1$, and
\item $g_{\alpha\beta}g_{\gamma\beta}^{-1}g_{\gamma\alpha}=1$.
\end{itemize}

Conversely, if we are given $\{g_{\alpha\beta}\}$ satisfying these properties, then we can construct a line bundle $L$ with transition functions $\{g_{\alpha\beta}\}$ as a quotient of $\coprod U_\alpha\times \CC$ by the appropriate gluing relations. 

\subsubsection{Choices}
If $f_\alpha\in\mathcal{O}^\times(U_\alpha)$ is a nonvanishing holomorphic function on $U_\alpha$, and we construct new trivializations $\phi'_\alpha=f_\alpha\circ \phi_\alpha$, then the new transition functions $g_{\alpha\beta}'=\frac{f_\alpha}{f_\beta}g_{\alpha\beta}$
give the same bundle $L$. 

Now, the collection of $\{g_{\alpha\beta}\in \mathcal{O}^\times (U_\alpha\cap U_\beta)\}$ is a \v{C}ech $1$-cochain. The conditions we wrote down that it satisfies implies that it is actually a $1$-cocycles. Moreover, the ambiguity mentioned above is exactly the $1$-coboundaries. Therefore, we can then conclude the set of (isomorphism classes of) line bundles $\Pic(X)$ is isomorphic to $H^1(X,\mathcal{O}_X^*)$. Moreover, this is a group homomorphism; for the group structure on line bundles coming from tensor product and the group structure naturally showing up on $H^1(X,\mathcal{O}_X^\times)$. 

\subsubsection{Line bundles vs ``factors of automorphy''}
Let $X$ be a complex torus, and $\tilde{X}=\CC^g$ its universal cover, with $\pi:\tilde{X}\to X$ the covering map. 
\begin{theorem}
There exists a canonical exact sequence 
$$0\to H^1(\pi(X),H^0(\tilde{X},\mathcal{O}_{\tilde{X}}^\times))\overset{\phi}{\to} H^1(X,\mathcal{O}_X^\times)\to H^1(\tilde{X},\mathcal{O}_{\tilde{X}}^\times).$$
\end{theorem}

The first term in the above exact sequence is the group of \emph{``factors of automorphy''} and the latter two are Picard groups as above. (Note that in general $\mathcal{O}_X^\times$ is the sheaf of invertible elements in our rings with respect to multiplication. In our holomorphic setting, this is equivalent to the sheaf of nonvanishing functions, because you can prove that if $f$ is holomorphic and nonzero, then $1/f$ is holomorphic.)

\begin{corollary}
If $\pi^*(L)$ is trivial, then it is completely described by a ``factor of automorphy''.
\end{corollary}

\begin{corollary}
If $\Pic(\tilde{X})=0$ (eg when $\tilde{X}=\CC^g$), then $\phi$ is an isomorphism.
\end{corollary}

Now, we prove the theorem above.
\begin{proof}
The $1$-cochains $C^1(G,M)$ where $G=\pi_1(X)$ and $M=H^0(\tilde{X},\mathcal{O}_X^\times)$. Note that $M$ contains elements like holomorphic $g:\tilde{X}\to \CC^\times$. Therefore, for $(h:G\to M)\in C^1(G,M)$, we can naturally get holomorphic $f:G\times \tilde{X}\to \CC^\times$. Note that $G$ acts on $\tilde{X}$ by $g\cdot \tilde{x}$ which gives $f(\tilde{x})\cdot g=f(g\cdot \tilde{x})$.

Also, note that the $1$-cocycles satisfy 
$$(h(\mu)\cdot \lambda) (h(\lambda \mu)^{-1})(h(\lambda))=1.$$

Therefore, we have the following cocycle condition: For $\lambda,\mu\in \pi_1(X), \tilde{x}\in \tilde{X}$,
$$f(\lambda\mu,\tilde{x})=f(\mu,\lambda\tilde{x})f(\lambda,\tilde{x})$$

Also the $1$-coboundary condition: For $h_1\in M=C^0(G,M)$,
$$f(\lambda,\tilde{x})=\frac{h_1(\lambda\cdot\tilde{x})}{h_1(\tilde{x})}$$
and we define $H^1(G,M)=Z^1/B^1$ like before.

Now, each $f\in Z^1(G,M)$ gives a Line Bundle on $X$. Let $L=\tilde{X}\times \CC\to \tilde{X}$ be the trivial line bundle map. Then $G=\pi_1(X)$ acts on $\tilde{X}$, and so acts on $L$: $\lambda\cdot(\tilde{x},t)=(\lambda\cdot \tilde{x},f(\lambda,\tilde{x})t)$.
It is easy to check that this action is
\begin{itemize}
\item free (If $g\cdot *=*$ for some $g$ and $*$, then $g=id$.)
\item properly discontinuous (For all $K_1,K_2$ compact subsets, $\{g\in G:gK_1\cap K_2\not=\emptyset\}$ is finite.)
\end{itemize}

There is a theorem saying that complex manifolds came from complex manifolds moding out free properly discontinuous actions. Therefore, the induced $L'\to X$ is a holomorphic line bundle. And this gives a map $Z^1(G,M)\to H^1(X,\mathcal{O}_X^\times)$. 

Proof of Theorem: 
First we will show that there is a map $\phi_1$ given by
$$0\to H^1(G,M)\overset{\phi_1}{\to} \ker(H^1(X,\mathcal{O}_X^\times)\to H^1(\tilde{X},\mathcal{O}_X^\times)).$$
and then we will try to prove that $\phi_1$ is an isomorphism.

First, for existence of $\phi_1$, we can just take the above map. Or a more explicit solution for this is given by the following: Let $\pi:\tilde{X}\to X$ be the covering map. Then fix a cover $\{U_i\}_I$ such that for all $i\in I$, there exists $W_i\subset \pi_i^{-1}(U_i)$, with biholomorphic $\pi_i:=\pi|_{W_i}:W_i\to U_i$. For all $i,j$, there exists unique $\lambda_{i,j}\in\pi_1(X)$ such that for $x\in U_i\cap U_j$, 
$\pi_j^{-1}(x)=\lambda_{ij}\cdot \pi_i^{-1}(x)$. It is clear that $\lambda_{ij}\cdot \lambda_{jk}=\lambda_{ik}$. Therefore, now we have a map $Z^1(G,M)\to Z^1(X,\mathcal{O}_X^\times)$ (note that $Z^1(X,\mathcal{O}_X^\times)$ maps onto $H^1(X,\mathcal{O}_X^\times)$) given by 
$$f\mapsto \{g_{ij}\in \mathcal{O}_X^\times (U_i\cap U_j)\}$$
where $g_{ij}(x)=f(\lambda_{ij},\pi_i^{-1}(x))$. It is easy to check the $1$-cocycle condition
$$g_{ij}g_{jk}=g_{ik}.$$

(Explicitly, $g_{ij}(x)g_{jk}(x)=f(\lambda_{ij},\pi_i^{-1}(x))f(\lambda_{jk},\pi_j^{-1}(x))=f(\lambda_{ij}\lambda_{jk},\pi_i^{-1}(x))=f(\lambda_{ik},\pi_i^{-1}(x))=g_{ik}(x)$.)

So we have a well defined map 
$$Z^1(G,M)\to \check{Z}^1(X,\mathcal{O}_X^\times)\surjects \check{H}^1(X,\mathcal{O}_X^\times).$$
It is easy to check that the kernel of this map contains $B^1(G,M)$, so this descends to a homomorphism $H^1(G,M)\to \check{H}^1(X,\mathcal{O}_X^\times)$. (To check this: if we have a $1$-coboundary $f(\lambda,x)=h(\lambda\tilde{x})/h(x)$, this will go to a \v{C}ech $1$-cocycle
$$g_{ij}(x)=\frac{h(\lambda_{ij}\pi_i^{-1}(x))}{h(\pi_i^1(x)}=\frac{h(\pi_j^{-1}(x))}{h(\pi_i^{-1}(x))}.$$
This tells us $g_{ij}(x)$ is a \v{C}ech $1$-coboundary so is trivial in $\check{H}^1(X,\mathcal{O}_X^\times)$.)
It is also easy to check that $Im(\phi)$ lies in $\ker(H^1(X,\mathcal{O}_X^\times)\to H^1(\tilde{X},\mathcal{O}_X^\times))$.

Now, we want to show that $\phi_1$ is an isomorphism. 
To do this, we want to find an inverse map. That is, given $L\in \ker(H^1(X,\mathcal{O}_X^\times)\to H^1(\tilde{X},\mathcal{O}_X^\times))$, we want to find $f\in H^1(G,M)$ such that $L=\phi_1(f)$. 
(We want an explicit expression for this inverse map, we will use it very often when computing things.)


We know that $\pi^*L$ is trivial. So we fix $\alpha:\pi^*L\isomto \tilde{X}\times \CC$. (Note that $G$ acts on $\pi^*L$, and hence via $\alpha$, $G$ also acts on $\tilde{X}\times \CC$.) For all $\lambda\in G$, there exists automorphisms $\phi_\lambda$ of $\tilde{X}\times \CC$ such that 
$\phi_\lambda(\tilde{x},t)=(\lambda\cdot \tilde{x},f(\lambda,\tilde{x})t)$. It is easy to check that 
\begin{itemize}
\item $f(\lambda,\tilde{x})\in Z^1(G,M)$
\item For another map $\alpha':\pi^*L\to \tilde{X}\times \CC$, we have 
$$\phi_\lambda'(\tilde{x},t)=(\lambda\cdot \tilde{x},h(\lambda\tilde{x})f(\lambda\tilde{x})h(\tilde{x})^{-1}t)$$
where $\alpha'\alpha^{-1}(\tilde{x},t)=(\tilde{x},h(\tilde{x})t)$. 
\end{itemize}
\end{proof}

In fact, via a similar proof, one can also prove the following:

\begin{theorem}
Let $\mathcal{F}$ be a sheaf of abelian groups on $X$. Let $G=\pi_1(X)$ and $M=H^0(\tilde{X},\pi^*\mathcal{F})$. Then for each $n\geq 0$, there exist canonical homomorphisms $\phi_n:H^n(G,M)\to H^n(X,\mathcal{F})$. Moreover, 
\begin{itemize}
\item when $n=1$, for each $f\in Z^1(G,M)$, we have 
$(\phi_1 f)_{ij}=f(\lambda_{ij},\pi_i^{-1})$. 
\item when $n=2$, for each $f\in Z^2(G,M)$, we have 
$(\phi_2 f)_{ijk}=f(\lambda_{ij},\lambda_{jk},\pi_i^{-1})$. 
\end{itemize}

\end{theorem}

\subsection{Global sections of $H^0(X,L)$}

Our next goal is to describe global sections of $H^0(X,L)=\Gamma(L,X)$ in terms of $\phi_1(f)=L$, where $L\in \ker(H^1(X,\mathcal{O}_X^\times)\to H^1(\tilde{X},\mathcal{O}_X^\times))$. 

Since $\phi^*L$ is trivial, if we fix a trivialization $\alpha:\pi^*L\isomto \tilde{X}\times \CC$, 
this gives us an explicit cocycle 
$f\in Z^1(G,M)$ (giving the cohomology class $\phi_1^{-1}(L)$). 
Observe that there exists a canonical isomorphism between $H^0(X,L)\isomto H^0(\tilde{X},\pi^*L)^G$. And then we have (via $\alpha$) an isomorphism $H^0(\tilde{X},\pi^*L)^G\isom H^0(\tilde{X},\tilde{X}\times \CC)^G$, which is explicitly given by
$$\phi_\lambda(\tilde{x},t)=(\lambda\cdot \tilde{x}, f(\lambda,\tilde{x})t).$$

$H^0(\tilde{X},\tilde{X}\times \CC)^G$ then corresponds to the set $\{\vartheta:\tilde{X}\overset{hol}{\to}\CC: \vartheta(\lambda,\tilde{x})=f(\tilde{\lambda},\tilde{x})\vartheta(x)\}$. So $H^0(X,L)$ is in bijection with the set of theta functions $\vartheta$ with $f$ as its factor of automorphy. 
(Remark: This does depend on the trivialization and thus the specific form of $f$; but changing $\alpha$ changes $f$ in a predictable way and thus changes the set of $\vartheta$.) Also, one can check functoriality of everything we've done.


Now, let's get back to the case where $X=V/\Lambda$ is complex torus. 


\begin{proposition}
Every holomorphic line bundle on $X$ pulls back to a trivial bundle on $V$. 
\end{proposition}

\begin{proof}[Sketch of proof]
We need
\begin{itemize}
\item the \emph{exponential sequence} of sheaves:
$$0\to \underline{\ZZ}\to \mathcal{O}_V\overset{e^{2\pi i}}{\to}\mathcal{O}_V^\times \to 1$$
($\underline{\ZZ}$ is the constant sheaf of $\ZZ$.)

\item the $\overline{\partial}$-Poincar\'{e} lemma; and
\item $H^2(V,\ZZ)=0$.
\end{itemize}

This induces the exact sequence:
$$\cdots \to H^1(V,\mathcal{O}_V)\to H^1(V,\mathcal{O}_V^\times)\to H^2(V,\ZZ)\to \cdots$$
First term is trivial by the $\overline{\partial}$-Poincare lemma, and last term is trivial by $H^2(V,\ZZ)=0$. Therefore, $H^1(V,\mathcal{O}_V^\times)$ is also trivial. So, all line bundles on $V$ are trivial. 
\end{proof}



\begin{corollary}
The above exact sequence gives an isomorphism
$\Pic(X):=H^1(X,\mathcal{O}_X^\times)\isom H^1(\Omega,H^0(V,\mathcal{O}_V^\times))$.
\end{corollary}

The proof can be found in Griffiths-Harris. [farbod add reference]



\subsection{The exponential sequence and the first Chern class}

In this section, we will talk about the exponential exact sequence, and generalize to $X$. We will then get a similar long exact sequence, but here $H^2(X,\ZZ)$ is nontrivial, and the map $H^1(X,\mathcal{O}_X^\times)\to H^2(X,\ZZ)$ is extremely important! Our next goal is the Appell-Humbert theorem, which is actually due to Lefschetz.

On any complex manifold $(X,\mathcal{O}_X)$, the sequence of sheaves 
$$0\to \underline{\ZZ}\to \mathcal{O}_X\overset{exp}{\to} \mathcal{O}_X^\times \to 1$$
is exact, where the map $exp:\mathcal{O}_X\to \mathcal{O}_X^\times$ is the exponential map, given on open sets by  
$exp(U):\mathcal{O}_X(U)\to \mathcal{O}_X^\times(U)$ by 
$$f\mapsto e^{2\pi i f}.$$
It is easy to check that this map is holomorphic  with multiplicative inverse $e^{-2\pi i f}$. 


Exactness of the sequence 
$$0\to \underline{\ZZ}\to \mathcal{O}_X\to \mathcal{O}_X^\times$$
is straightforward; the kernel of $exp:\mathcal{X}(U)\to \mathcal{O}_X^\times(U)$ are the locally constant $\ZZ$-valued functions, which are $\underline{\ZZ}(U)$. What about the surjectivity of $\mathcal{O}_X\to \mathcal{O}_X^\times$? This asks about the existence of logarithms. Well, taking complex logarithms are not always ``globally'' possible! However, they are ``locally'' possible; for any $f\in \mathcal{O}_X^\times(U)$ and any $x\in U$, we can find a contractible neighborhood $V$ of $x$ with $V\subset U$, and then $res_V(f)$ does have a logarithm. So $\mathcal{O}_X(U)\to \mathcal{O}_X^\times (U)$ is surjective for small enough $U$, and this is enough for the sheaf map $\mathcal{O_X}\to \mathcal{O}_X^\times$ to be considered surjective (the cokernel is a presheaf which sheafifies to $0$).
For example, $z\in \mathcal{O}^\times(\CC-\{0\})$ has logarithm if you restrict to a small enough open set. 

So now we have the long exact sequence that measures the failure of surjectivity of the exponential function, which will give us 
$$0\to H^0(X,\underline{\ZZ})\to H^0(X,\mathcal{O}_X)\to H^0(X,\mathcal{O}_X^\times)\to H^1(X,\underline{\ZZ})\to \cdots$$

so surjectivity of the global exponential map is controlled by $H^1(X,\underline{\ZZ})$, a ``generalized winding number''. Continuing we get

$$\cdots\to H^1(X,\mathcal{O}_X)\to H^1(X,\mathcal{O}_X^\times) \to H^2(X,\ZZ)\to H^2(X,\mathcal{O}_X)\to \cdots$$
and beyond this we won't really care.

\subsubsection{The first Chern class}
Note that $\Pic(X)=H^1(X,\mathcal{O}_X^\times)$ and $H^1(X,\mathcal{O}_X^\times) \to H^2(X,\ZZ)$ in the above sequence is called the \textbf{first chern class map}, denoted by $c_1$. Also, $Im(c_1)$ is called the \textbf{N\'{e}ron-Severi group} of $X$, denoted by $NS(X)$. 

Remark: We can thus write this as $NS(X)\isom \Pic(X)/\Pic^0(X)$ where $\Pic^0(X)$ is the subgroup of $\Pic(X)$ consisting of line bundles $L$ with $c_1(L)=0$. This is just rephrasing the definition above, but we can also prove that $\Pic^0(X)$ is equal to the connected component of of the origin in $\Pic(X)$.

\begin{theorem}[Severi, N\'{e}ron]
$NS(X)$ is always a finitely generated abelian group.
\end{theorem}

Severi proved the theorem for $k=\CC$ and N\'{e}ron proved for more general fields, but we are not going to make that precise. We define $\rho(X)$ to be the rank of $NS(X)$, which is the \textbf{Picard number} of $X$. We will see that $\rho(X)\leq h^{1,1}(X)=g^2$. 

Recall that $H^2(X,\ZZ)\isom H^1(X,\ZZ)\bigwedge H^1(X,\ZZ)\isom \Hom(\Lambda,\ZZ)\bigwedge \Hom(\Lambda,\ZZ)$ which we are going to denote by $\text{Alt}^2(\Lambda,\ZZ)$. Note that $\text{Alt}^2(\Lambda,\ZZ)$ contains $\ZZ$-valued bilinear alternating $2$-forms. 

\subsection{First Chern class for complex tori}
When $X=V/\Lambda$ ($V\isom \CC^g$) is a complex torus, the exponential sequence gives 
$$H^0(V,\mathcal{O}_V)\to H^0(V,\mathcal{O}_V^\times)\to 0$$
because $H^1(V,\underline{\ZZ})=0$. So any non-vanishing global form $f$ is $e^{2\pi i g} $ for some $g\in \mathcal{O}_V$. Recall that $H^1(\Lambda,H^0(V,\mathcal{O}_V^\times))\isom H^1(X,\mathcal{O}_X^\times)$. So given $L$, one has a factor of automorphy $(f:\Lambda\times V\to \CC^\times)\in Z^1(\Lambda,H^0(V,\mathcal{O}_V^\times))$. We assume $f=e^{2\pi i g}$ for $g:\Lambda \times V\to \CC$. 

\begin{theorem}
With the canonical isomorphism discussed before, we have 
$$c_1(L)=E_L(\cdot,\cdot)$$
where
$$E_L(\lambda,\mu)=g(\mu,v+\lambda)+g(\lambda,v)-g(\lambda,v+\mu)-g(\mu, v)$$
for $\lambda,\mu\in \Lambda$ and $v\in V$.
\end{theorem}

It is a good exercise to show that 
\begin{itemize}
\item $E_L$ is independent of $v$. (Using the fact that $f\in Z^1$.)
\item $E_L$ is an alternating $2$-form, $\ZZ$-valued and also bilinear. 
\end{itemize}
 

\begin{theorem}
For any $E:V\times V\to \RR$ which is an alternating $\RR$-valued bilinear form on $H^2(X,\RR)$, the following are equivalent:
\begin{enumerate}
\item $E=c_1(L)$ for some $L\in \Pic(X)$.
\item $E(\Lambda,\Lambda)\subset \ZZ$ and $E(iv,iw)=E(v,w)$ for $v, w \in V$.
\end{enumerate}
\end{theorem}

These two theorems turn out to be very useful! We will skip the proofs for now. 

It is an easy fact from Linear Algebra that there is a one-to-to correspondence between Hermitian forms $H$ on $V$ ($H:V\times V\to \CC$ such that $H(v,w)=\overline{H(w,v)}$ for all $w,v\in CC$) and the set of alternating bilinear forms $E:V\times V\to \RR$ satisfying $E(iv,iw)=E(v,w)$ for all $i\in V$. In fact, the correspondence maps are explicitly given by 
$H\mapsto Im(H)$
and 
$E\mapsfrom E(iv,w)+iE(v,w)$. 

\begin{corollary}
$NS(X)$ is exactly the set of alternating bilinear forms $E: V\times V\to \RR$ such that $E(\Lambda,\Lambda)\subset \ZZ$ and $E(iv,iw)=E(v,w)$ for all $v,w,i\in V$. By the above correspondence, $NS(X)$ also equals the set of Hermitian forms $H:V\times V\to \CC$ such that $Im H(\Lambda,\Lambda)\subset \ZZ$. 
\end{corollary}


\subsection{Appell-Humbert Theorem}

In this section we are going to see an explicit description of $NS(X),\Pic(X),\Pic^0(X)$ for $X=V/\Lambda$ a complex torus. Recall that we had an explicit description of $NS(X)$ already: it corresponds to the set of Hermitian forms $H:V\times V\to \CC$ with $Im(\Lambda,\Lambda)\subset \ZZ$. Define $\Pic^0(X)$ to be the collection of line bundles such that the first chern group on them are trivial. That is, 
$$\Pic^0(X)=\ker(\Pic(X) \xrightarrow{c_1} H^2(X,\ZZ)).$$
So, we have the following short exact sequence:
$$0\to \Pic^0(X)\to \Pic(X)\to NS(X)\to 0$$

The Appell-Humbert theorem is going to give another (more explicit) short exact sequence that is isomorphic to the one above. 


Let $T_1=\{z\in \CC^\times: |z|=1\}$ be the unit circle in $\CC$. Given a Hermitian form $H$ in $NS(X)$, a character $\chi:\Lambda\to T_1$ is called a \textbf{semi-character} for $H$ if for all $\lambda,\mu\in \Lambda$, we have
$$\chi(\lambda+\mu)=\chi(\lambda)\chi(\mu)e^{\pi i Im H(\lambda\mu)}.$$

Let $P(\Lambda)$ be the set of $(H,\chi)$ with $H\in NS(X)$ such that $\chi$ is a semi-character for $H$.

Remarks: Note that when $H=0$, this is just a usual character. Also, $P(\Lambda)$ is a group under
$$(H_1,\chi_1)\circ (H_2,\chi_2):=(H_1+H_2,\chi_1\chi_2).$$

Note that we have the following exact sequence:
$$0\to \Hom(\Lambda,T_1)\to P(\Lambda)\to NS(X)$$
where the second map is given by $x\mapsto (0,x)$ and third map by $(H,\chi)\mapsto H$. And therefore we have a map $P(\Lambda)\to \Pic(X)\isom H^1(\Lambda,M)$ (where $M=H^0(V,\mathcal{O}^\times_V)$) given by
$$(H,\chi)\mapsto a_{(H,\chi)}(\cdot,\cdot)+B^1(\Lambda,M)$$
where $a_{(H,\chi)}(\lambda,v):=\chi(\lambda)e^{(\pi H(\lambda,v)+\pi/2 H(v,v))}$. It is easy to check that 
$$a_{(H,\chi)}(\lambda,\mu,v)=a_{(H,\chi)}(\lambda,\mu+v)a_{(H,\chi)}(\mu,v).$$ 
This gives us an isomorphism 
$$L(H,\chi)\isom \frac{V\times \CC}{\Lambda}$$
where $\Lambda$ acts as $\lambda\circ (v,t)=(v+\lambda,a(\lambda,v)t)$. We have the following facts:

\begin{enumerate}
\item $P(\Lambda)\to \Pic(X)$ is a group homomorphism. (Easy to prove.)
\item The map given by composing $P(\Lambda)\to \Pic(X) \overset{c_1}{\to} NS(X)$ is the same as the ``forget $\chi$'' map. In particular, the map $\alpha:P(\Lambda)\to NS(X)$ defined by $(H,\chi)\mapsto H$ is surjective (since $c_1$ is). 
\end{enumerate}

By these facts, we hence have the following theorem.

\begin{theorem}
The following is a commutative diagram of short exact sequences:
\[
\xymatrix{
0 \ar[r] &\Hom(\Lambda,T_1) \ar[r] \ar[d]^{\alpha'} &P(\Lambda) \ar[r] \ar[d]^{\alpha} &NS(X) \ar[r] \ar[d]^= &0\\
0 \ar[r] &\Pic^0(X) \ar[r] &\Pic(X) \ar[r] &NS(X) \ar[r] &0
}
\]
where $\alpha'$ is the restriction of $\alpha$ to $\Hom(\Lambda,T_1)$. 
\end{theorem}

Once we have shown that $\alpha'$ is an isomorphism in the proof, we will have a very explicit way to talk about $\Pic(X)$ - not only have we parametrized line bundles by $(H,\chi)$, such a pair even leads to a distinguished cocycle in the cohomology class to work with.


\begin{proof}
We already knew that the right-hand square in the above diagram commutes, and this implies that $\alpha'$ restricts to what we want (and the left-hand square commutes arbitrarily). So we just need to show that $\alpha'$ and $\alpha$ are isomorphisms. It suffices to show that $\alpha'$ is an isomorphism, since then the five-lemma implies $\alpha$ is. 

First, we start with a diagram

\[
\xymatrix{
H^1(X,\ZZ) \ar[r] &H^1(X,\mathcal{O}_X) \ar[r] &H^1(X,\mathcal{O}_X^\times) \ar[d]^= \ar[r]^{c_1} &H^2(X,\ZZ)\\
&H^1(X,\CC) \ar@{->>}[u] \ar[r]^\epsilon &H^1(X,\mathcal{O}_X^\times)}
\]



Here, the surjective map is actually a projection which comes from the folllowing: By the Hodge decomposition, we have $H^1(X,\CC)\isom H^{1,0}(X)\bigoplus H^{0,1}(X)$ for which by the Dolbeault cohomology is isomorphic to $H^0(X,\Omega_X)\bigoplus H^1(X,\mathcal{O}_X)$. So we have a projection from $H^1(X,\CC)\surjects H^1(X,\mathcal{O}_X)$. 

Next, by taking the sheaf map $\underline{\CC}^\times \to \mathcal{O}_X^\times$ which takes a locally constant function $f$ to the locally constant nonvanishing function $\exp(2\pi if)$; this induces a map $\epsilon: H^1(X,\underline{\CC})\to H^1(X,\mathcal{O}_X^\times)$ on cohomology. 

To see that the diagram commutes, note that since $\Pic^0(X)$ is the image of $\epsilon: H^1(X,\CC)\to H^1(X,\mathcal{O}_X^\times) \isom H^1(\Lambda,M))$, every element $L\in \Pic^0(X)$ is represented by a $1$-cech cocycle (that is, an element of $Z^1(X,\mathcal{O}_X^\times)$) with constant coefficients. Therefore, $H^1(X,\CC)\to H^1(X,\mathcal{O}_X^\times$ implies that the corresponding class in $H^1(\Lambda,M)$ is represented by a factor of automorphy (an element of $Z^1(\Lambda,M)$) with constant coefficients, that is, $f(\lambda,v)$ is independent of $v$. 

We claim that $\alpha'$ is surjective. Let $L\in \Pic^0(X)$ with $f\in Z^1(\Lambda,M)$ with ``constant coefficient'' (i.e. $f(\lambda,v)$ is independent of $v$). Then we have
$$f(\lambda+\mu,\tilde{x})=f(\lambda,\mu+\tilde{x})f(\mu,\tilde{x}).$$
So $f$ gives a homomorphism $f:\Lambda\to \CC^\times$. Define $g$ by $f=e^{2\pi i g}$, then obviously we have
$$g(\lambda+\mu)=g(\lambda)+g(\mu) \pmod{\ZZ}.$$
Therefore, $Im(g):\Lambda\to \RR$ is a homomorphism, which induces a linear map $Im(g):V\to \RR$. Define $\ell: V\to \CC$ by $v\mapsto Im(g)(iv)+iIm(g)(v)$. Then $e^{2\pi i \ell(v)}\in H^0(V,\mathcal{O}_V^\times)$ for all $v\in V$. Therefore we have 
$$\chi_L(\lambda,v)=f(\lambda) \cdot e^{2\pi i \ell(v)-2\pi i \ell(v+\lambda)}.$$
is in $Z^1(\Lambda,M)$, that is, a $1$-coboundary. But now $\chi_L\in \Hom(\Lambda,T_1)$. To check this, 
$$\chi_L(\lambda,v)=e^{2\pi i g(\lambda)-2\pi i \ell(v)}=e^{2\pi i (Re g(\lambda)-Im g(i\lambda))}\in T_1$$
as $Re g(\lambda)-Im g(i\lambda)\in \RR$. Moreover, this last expression is independent of $v$. Therefore, $f$ and $e^{2\pi i \ell(\cdot)}$ are homomorphisms and hence $\chi_L$ is also a homomorphism.

To show that $\alpha'$ is injective, 
suppose $\chi_1$ and $\chi_2$ both give $L\in \Pic(X)$, then we can write
$$\chi_1(\lambda)=\chi_2(\lambda)\cdot \frac{h(\lambda+v)}{h(v)}$$
where $\frac{h(\lambda+v)}{h(v)}$ is a $1$-coboundary. Since $|\chi_1|=|\chi_2|=1$,, we have $|h(\lambda+v)|=|h(v)|$ for all $v,\lambda$. Hence $h$ is bounded. By the Liouville theorem $h$ is constant and so $\chi_1=\chi_2$. 

\end{proof}


\subsection{Canonical factors}
Recall that in the proof of the Appell-Humbert Theorem, we expressed $\alpha$ as the map given by 
$$(H,\chi)\mapsto a_{(H,\chi)} (\lambda,v)=\chi(\lambda)e^{\pi (H(\lambda,v)+1/2 H(\lambda,\lambda))}$$
where $H=c_1(L)$. We define the \textbf{canonical factor} of $(H,\chi)$ to be $a_{(H,\chi)}$. In fact, most questions about line bundles boil down to explicit computations on these $a_{(H,\chi)}$! 

First, let us see some basic properties:
\begin{enumerate}
\item $\chi(n\lambda)=\chi(\lambda)^n$ for all $\lambda$.
\item $a_L(\lambda,v+w)=a_L(\lambda,v)e^{\pi H(\lambda,w)}$.
\item $\frac{1}{a_L(\lambda,v)}=a_L(-\lambda,v)e^{-\pi H(\lambda,\lambda)}$. 
\end{enumerate}

\subsection{Behaviour of line bundles under holomorphic maps}
Recall that holomorphic maps are given by compositions of translations and homomorphisms. 

\begin{lemma}
If $t_x:X\to X$ is given by $y\mapsto x+y$ and we also have a map $L=L(H,\chi)\to X$, 

\[
\xymatrix{
L \ar@{.>}[r]^{t^*_x} \ar[d] & L \ar[d]\\
X \ar[r]^{t_x} & X\\
}
\]


then the pullback map is given by  
$$t_x^* L(H,\chi)=L(H,\chi e^{2\pi i Im H(\tilde{x},\cdot)})$$
where $\tilde{x}$ is a lift of $x$ to the universal cover $V$. 
\end{lemma}


\begin{lemma}
If we have a map between complex tori $f:X'\to X$ where $X'=V'/\Lambda'$ and $X=V/\Lambda$. Suppose $L=L(H,\chi)\in \Pic(X)$, then 
$$f^* L(H,\chi)=L(f^*_{an} H,f^*_{Int}\chi).$$
\end{lemma}

\begin{corollary}[Theorem of squares]
For $v,w\in X$ and $L\in \Pic(X)$, we have $t^*_{v+w}L\isom t^*_v L\otimes t^*_wL \otimes L^{-1}$.
\end{corollary}

\begin{proof}
For $L=L(H,\chi)$, we have $t^*_x L(H,\chi)=L(H,\chi e^{2\pi i Im(H(\lambda,v))})$.
\end{proof}


\begin{corollary}[Theorem of cubes for complex tori]
Let $X_1,X_2,X_3$ be complex tori. Let $L$ be a line bundle on $X_1\times X_2\times X_3$.  If the restrictions of $L$ on $X_1\times X_2\times \{0\}, X_1\times \{0\}\times X_3$ and $\{0\}\times X_2\times X_3$ are trivial, then $L$ is also trivial on $X_1\times X_2\times X_3$.
\end{corollary}

Remark: the theorem of cubes works if $X_1,X_2,X_3$ are just complete varieties. Even more, Mumford p.55 [farbod reference] says the theorem still holds if one of them is just connected. 

\begin{corollary}
For $n\in \ZZ$, let $n_X:X\to X$ be the multiplication-by-$n$ map $x\mapsto nx$. If $L\in \Pic(X)$, then 
$$n_X^*L = L^{\frac{n^2+n}{2}}\bigotimes (-1)^* L^{\frac{n^2-n}{2}}$$
where $L^k$ is the tensor product $L\bigotimes \cdots \bigotimes L$ of $k$ copies of $L$.
\end{corollary}

\begin{proof}
Write $L=L(H,\chi)$. Then 
$$L^{\frac{n^2+n}{2}}\bigotimes (-1)^* L^{\frac{n^2-n}{2}}= L(\frac{n^2+n}{2}H + (-1)^* \frac{n^2-n}{2}H,\chi^{\frac{n^2+n}{2}}\times (-1)^*\chi ^{\frac{n^2-n}{2}}).$$
Since $(-1)^* H(u,v)= H(-u,-v)=H(u,v)$ and $(-1)^*\chi(\lambda)=\chi(-\lambda)=\frac{1}{\chi(\lambda)}$, this is equal to 
$$L(n^2H,\chi^n)=L(n^*H, n^*\chi(\cdot))=n^* L(H,\chi).$$
\end{proof}

A line bundle $L$ is called \textbf{symmetric} if $(-1)^*L=L$. 

\begin{corollary}
If $L$ is symmetric, then
$n_X^* L = L^{n^2}$.
\end{corollary}

\begin{lemma}
$L=L(H,\chi)$ is symmetric if and only if $\chi(\lambda)=\pm 1$ for all $\lambda\in \Lambda$. 
\end{lemma}

\begin{proof}
Since $(-1)^* L(H,\chi)=L(H,\frac{1}{\chi})$, we have $\chi^2(\cdot )=1$.
\end{proof}

\subsection{Dual complex tori}
Our goal of this section is to show that any complex torus $X=V/\Lambda$ has a dual $\hat{X}$ with functorial properties. Moreover, given $L\in \Pic(X)$, one has $\varphi_L:X\to \hat{X}$. We will also talk about $X\times \hat{X}$, Poincar\'{e} bundles and biextensions. 

First of all, the Appell-Humbert theorem tells us that we have an isomorphism 
$$\Hom(\Lambda,T_1)\isom \Pic^0(X).$$
But in the case of complex tori, we have $\Hom(\Lambda,T_1)\isom \Hom(\ZZ^{2g},S^1)\isom (\RR/\ZZ)^{2g}$. Is $\Pic^0(X)$ naturally a complex torus? The answer is yes! This is because we have a canonical isomorphism $\Pic^0(X)\isomfrom \hat{X}$ from the dual complex torus. 

Recall that the cotangent space at $\underline{0}$ is given by 
$$\Omega=\Hom_\CC(V,\CC)=\{\ell:V\to \CC:\ \ell(av+bw)=a\ell(v)+b\ell(w),\ \forall a,b\in \CC, v,w\in V \}$$ where $V$ is the tangent space at $0$. We have a conjugate of $\Omega$ given by
$$\overline{\Omega}=\Hom_{\overline{\CC}}(V,\CC)=\{\ell:V\to \CC:\ \ell(av+bw)=\overline{a}\ell(v)+\overline{b}\ell(w),\ \forall a,b\in \CC, v,w\in V \}.$$
Then we claim that we have a functional 
$$\overline{\Omega}\isomto \Hom_\RR(V,\RR).$$
Note that $\Hom_\RR(V,\RR)\isom \RR^{2g}$.
It is not hard to see that the inverse maps are given by $\ell \mapsto K=Im (\ell)$ and $\ell \mapsfrom K$ where $\ell(v):=-K(iv)+iK(v)$. It follows that 
$$\langle \cdot, \cdot \rangle:\overline{\Omega}\times V\to \RR$$
defined by 
$$\langle \ell, v\rangle = Im \ell(v)$$
is $\RR$-bilinear and is non-degenerate. 
By non-degenerate, we mean
\begin{itemize}
\item If $\langle \ell,v\rangle=0$ for all $\ell$, then $v=0$; and
\item If $\langle \ell,v\rangle=0$ for all $v$, then $\ell=0$.
\end{itemize}


Therefore, 
$$\hat{\Lambda}:=\{\ell \in \overline{\Omega}:\ \langle \ell,\lambda\rangle\in \ZZ\ \forall \lambda\in \Lambda\}$$
is a lattice (full rank, i.e. $\isom \ZZ^{2g}$) in $\overline{\Omega}$. We then define 
$$\hat{X}=\hat{\Omega}/\hat{\Lambda}$$
to be the \textbf{``dual''} of $X$. 
\begin{theorem}
There is a canonical isomorphism 
$$\hat{X}\isomto \Pic^0(X)$$
induced by $\overline{\Omega}\to \Hom(\Lambda,T_1)$ which maps $\ell\mapsto e^{2\pi i Im\ell (\cdot)}$ where $Im \ell(\cdot)=\langle \ell,\cdot\rangle$. 
\end{theorem}

\begin{proof}
By non-degeneracy, the map is surjective. Also the kernel by definition is just $\hat{\Lambda}$. 
\end{proof}

\begin{corollary}
$\Pic^0(X)=\hat{X}$ is a complex torus.
\end{corollary}

\subsection{Basic functorial properties of $\hat{X}$}

Here are some of the basic functorial properties of $\hat{X}$:

\begin{enumerate}
\item For $f:X_1\to X_2$ homomorphism with $f_{an}:V_1\to V_2$, then $f_{an}^*:\overline{\Omega_2}\to \overline{\Omega_1}$ satisfies 
$$f_{an}^*(\hat{\Lambda}_2)\subset \hat{\Lambda}_1.$$
So we get a homomorphism $\hat{f}:\hat{X_2}\to \hat{X_1}$. Moreover, 

\[
\xymatrix{
\hat{X}_2 \ar[d]^{\hat{f}}\ar[r]^= &\Pic^0(X_2) \ar[d]^{f^*}\ar[r]^= & \Hom(\Lambda_2,T_1) \ar[d]^{f_{int}^*}\\
\hat{X}_1 \ar[r]^= &\Pic^0(X_1) \ar[r]^= &\Hom(\Lambda_1,T_1)
}
\]
commutes.

\item If $f:X_1\to X_2, g:X_2\to X_3$ are homomorphisms, then 
$$\widehat{g\circ f}=\hat{f}\circ \hat{g}.$$
Also, $\hat{Id_X}= Id_{\hat{X}}$ with $\hat{\hat{X}}=X$ and hence $\hat{\hat{f}}=f$. 

\end{enumerate}

As a consequence, we see that $\hat{(\cdot)}$ is a contravariant functor on the category of complex tori.

\begin{lemma}
The functor $\hat{(\cdot)}$ is exact. That is, for all short exact sequences 
$$0\to X_1 \to X_2 \to X_3 \to 0,$$
we have 
$$0\to \hat{X}_3\to \hat{X}_2 \to \hat{X}_1\to 0.$$

\end{lemma} 

\begin{proof}
An easy application of the snake lemma gives us the short exact sequence
$$0\to \Lambda_1\to \Lambda_2\to \Lambda_3\to 0.$$
Since $\Lambda_3$ is a projective (free) $\ZZ$-module, we therefore have 
$$0\to \Hom(\Lambda_3,T_1)\to \Hom(\Lambda_2,T_1)\to \Hom(\Lambda_1,T_1)\to 0.$$
Lastly, $\Hom(\Lambda_i,T_1)\isom \Pic^0(X_i)\isom \hat{X_i}$.
\end{proof}


\subsection{Isogenies and duality}

If $f:X_1\to X_2$ is an isogeny, what can we say about $\hat{f}$?

\begin{proposition}
Let $f:X_1\to X_2$ be an isogeny, with dual homomorphism $\hat{f}=\hat{X}_2\to \hat{X}_1$. Then 
\begin{enumerate}[(a)]
\item $\hat{f}$ is an isogeny.
\item $\ker(\hat{f})=\Hom(\ker(f),T_1)$.
\item $\deg(\hat{f})=\deg(f)$.
\end{enumerate}
\end{proposition}

\begin{proof}[Sketch of proof]
We have $\ker(\hat{f})\isom \ker(\Hom(\Lambda_2,T_1)\overset{f_{int}^*}{\to}) \Hom(\Lambda_1,T_1)\isom \ker(\Hom(\Lambda_2/ f_{int}^*(\Lambda_1)),T_1)$. But $\Lambda_2/f_{int}^*(\Lambda_1)\isom \ker(f)$.
\end{proof}

\subsection{Line bundels v.s. duality}

Do line bundles descend under isogeny? 

\begin{proposition}
Suppose $f:X_1\to X_2$ is an isogeny. Let $L\in \Pic(X_1)$. As before, we write $L=L(h,\chi)$. The following are equivalent:
\begin{enumerate}
\item $L=f^* M$ for some $M\in \Pic(X_2)$.
\item The image of $H(f_{an}^{-1}\Lambda_2, f_{an}^{-1} \Lambda_2)$ is contained in $\ZZ$.
\end{enumerate}
\end{proposition}

\begin{proof}
First statement implies the second follows from the pull-back formula above. 
Now, assume the image of $H(f_{an}^{-1}\Lambda_2, f_{an}^{-1} \Lambda_2)$ is contained in $\ZZ$, then $H_1:=(f_{an}^{-1})^* H\in NS(X_2)$. So there exists $\tilde{M}\in \Pic(X_2)$ with $c_1(\tilde{M})=H_1$. By the pull-back formula, $c_1(f^* \tilde{M})=H$. Then $c_1(L\bigotimes (f^* \tilde{M})^{-1})=0$. Since $\hat{f}:\Pic^0(X_2)\to \Pic^0(X_1)$ is surjective, there exists $N\in \Pic^0(X_2)$ such that $f^* N = L\bigotimes (f^* \tilde{M})^{-1}$. Now, $M=N\bigotimes \tilde{M}$ does the job!
\end{proof}

So far, what we have for duality is very basic. But in the next section, we will see something highly nontrivial. 

\subsection{Line bundles and maps $X\to \tilde{X}$}
Let $L\in \Pic(X)$. We write the map $\varphi_L:X\to \tilde{X}=\Pic^0(X)$ which is given by 
$$x\mapsto t_x^* L\bigotimes L^{-1}.$$ 

Here are some facts about this $\varphi_L$:

\begin{enumerate}

\item
First of all, $\varphi_L$ is well defined, since $c_1(t_x^* L\bigotimes L^{-1})=0$.


\item $\varphi_L$ is a group homomorphism. To see this, recall that
by the theorem of squares, we have
$$f_{x+y}^* L\bigotimes L^{-1}\isom t_x^*L \bigotimes L^{-1} \bigotimes t_y^*L\otimes L.$$


\item $\varphi_L$ has the analytic representation $\varphi_H:V\to \overline{\Omega}$ given by 
$$v\mapsto H(v,\cdot ).$$
(This follows from the fact that $\varphi_L(x)=L(0,e^{2\pi i Im(H(v,\cdot))})$.)

\item $\varphi_L$ only depends on $c_1(L)=H$.
\item $\varphi_{L\bigotimes M}=\varphi_L + \varphi_M$.

\item For any $L\in \Pic(X)$, the diagram 
\[
\xymatrix{
X \ar[r]^{\varphi_L} &\hat{X} \ar[d]^{\hat{f}}\\
Y \ar[u]^{f} \ar[r]_{\varphi_{f^*(L)}} & \hat{Y}\\
}
\]
commutes. 

\end{enumerate}

\subsection{Kernel of $\varphi_L$}
Define $K(L)$ to be $\ker(\varphi_L:X\to \tilde{X})$. Here are some basic properties of $K(L)$:
\begin{enumerate}
\item Let $\Lambda(L):=\{v\in V:\ Im(H(v,\lambda))\in \ZZ\ \forall \lambda\in \Lambda\}=\varphi_H^{-1}(\hat{\Lambda})$. Then 
$$K(L)\isom \Lambda(L)/\Lambda.$$
Note that hence $K(L)$ only depends on $H=c_1(L)$. 

\item $K(L\bigotimes P)\isom K(L)$ for $P\in \Pic^0(X)$.
\item $K(L)=X$ if $L\in \Pic^0(X)$.
\item $K(L^n)=n_X^{-1} K(L)$
where $L^n=L\bigotimes \cdots \bigotimes L$ ($n$-copies) and $n_X^{-1}$ is the inverse isogeny.

\item $K(L)=n_X K(L^n)$ if $n\not= 0$. 
\end{enumerate}

For the last two facts, recall that for $L=L(H,\chi)$, we have $L^n=L(nH,\chi^n)$. Hence, $L_1\bigotimes L_2=L_1\bigotimes L_2(H_1+H_2,\chi_1\chi_2)$. Also, $\Lambda(L^n)=\{v\in V:\ Im(H(nv,\lambda))\in \ZZ,\ \forall \lambda\in \Lambda\}=\{\frac{1}{n} v\in V:\ v\in \Lambda(L)\}$. 

Recall that $H(\cdot, \cdot)$ is called \textbf{non-degenerate} if 
\begin{itemize}
\item $H(v,w)=0$ for all $v$ implies $w=0$; and
\item $H(v,w)=0$ for all $w$ implies $v=0$.
\end{itemize}

We say that $L\in \Pic(X)$ is \textbf{non-degenerate} if $c_1(L)=H$ is non-degenerate. (Equivalently, the associated alternating form $Im(H)$ is non-degenerate.) 

\begin{lemma}
\noindent
\begin{enumerate}
\item $L$ is non-degenerate if and only if $K(L)$ is finite.
\item $\deg(\varphi_L)=\det(Im(H))=[\Lambda(L):L]$.
\end{enumerate}
\end{lemma}

\subsection{Poincar\'{e} bundle:}

We have seen that for $X$ a complex torus, 
\begin{enumerate}[(a)]
\item a point in $\hat{X}=\Pic^0(X)$ gives a line bundle on $X$;
\item a point on $X=\hat{\hat{X}}=\Pic^0(\hat{X})$ gives a line bundle on $\hat{X}$.
\end{enumerate}
So, does there exist a (universal) line bundle on $X\times \hat{X}$ such that (a) and (b) are ``shadows'' of that line bundle? The answer is yes!

A \textbf{Poincar\'{e} bundle} is a holomorphic line bundle $\mathcal{P}$ on $X\times \hat{X}$ satisfying 
\begin{enumerate}
\item $\mathcal{P}|_{X\times \{L\}}\isom L$;
\item $\mathcal{P}|_{\{0\}\times\hat{X}}$ is the trivial line bundle on $\hat{X}$.
\end{enumerate}

\begin{theorem}
There exists a Poincar\'{e} bundle on $X\times \hat{X}$ uniquely determined up to isomorphism.
\end{theorem}

\begin{proof}
First note that we have 
$$X\times \hat{X}\isom V\times \overline{\Omega}/\Lambda\times \hat{\Lambda}.$$

For the existence, we define $H:(V\times \overline{\Omega})\times (V\times \overline{\Omega})\to \CC$ by 
$$((v_1,\ell_1),(v_2,\ell_2))\mapsto \overline{\ell_2(v_1)}+\ell_1(v_2)$$
which is in fact a Hermitian form. (In particular, it is non-degenerate!) Note that
$$Im(H(\Lambda\times \hat{\Lambda},\Lambda\times \hat{\Lambda}))\subset \ZZ.$$
So there exist $\mathcal{L}=L(H,\chi)$ for semi-characters $\chi$. We then define $\chi_0: \Lambda\times \hat{\Lambda}\to T_1$ by 
$$(\lambda,\ell_0)\mapsto e^{\pi i Im(\ell_0(\lambda))}$$
which is indeed a semi-character for $H$. 

Now, we claim that $\mathcal{P}=L(H,\chi_0)$ is a Poincar\'{e} bundle. To see this, we consider the associated canonical factor 
$$a_\mathcal{P}((\lambda,\ell_0),(v,\ell))=\chi((\lambda,\ell_0))\cdot e^{\pi(H((\lambda,\ell_0),(v,\ell))+\frac{1}{2}H((\lambda,\ell_0),(\lambda,\ell_0)))}.$$
Now, what is left is just the checkings:
\begin{enumerate}
\item For any $L\in \hat{X}=\Pic^0(X)$, we have $L=L(0,e^{2\pi i Im(\ell(\cdot))})$ for some $\ell\in \overline{\Omega}$. Note that 
$$a_L(\lambda,v)=e^{2\pi i Im(\ell(\lambda))}.$$
 Therefore, 
$\mathcal{P}|_{X\times \{L\}}$ corresponds to $a_\mathcal{P}|_{(\Lambda,0)\times(V\times \{\ell\})}$ but $$a_\mathcal{P}((\lambda,0),(v,\ell))=e^{\pi \ell(\lambda)}.$$
Now, multiplication by the $1$-coboundary $e^{\pi \overline{\ell(v)}}/e^{\pi \overline{\ell(v+\lambda)}}$ takes $a_\mathcal{P}((\lambda,0),(v,\ell))$ to $a_L(\lambda,v)$. Therefore, $\mathcal{P}|_{X\times \{L\}}\isom L$.
\item $\mathcal{P}_{\{0\}\times \hat{X}}$ has $a_\mathcal{P}((0,\ell_0),(0,\ell))=1$ as $1$-cocycle. So it is trivial.
\end{enumerate}

Uniqueness of the Poincar\'{e} bundle follows from the \emph{Seesaw Principle}: Let $X,Y$ be compact complex manifolds and $\mathcal{L}$ be a holomorphic line bundle on $X\times Y$. If $L|_{X\times \{z\}}$ is trivial for all $z\in \mathcal{U}$ where $\mathcal{U}$ is an open dense subset of $Y$, and if $L|_{\{x_0\}\times Y}$ is trivial for some $x_0\in X$, then $L$ is trivial. 




\end{proof}

Here are some remarks:
\begin{itemize}
\item Poincar\'{e} line bundles are non-degenerate.
\item Let $T$ be any \emph{normal} complex analytic space and $X$ be a complex torus. If $L$ is a line bundle on $X\times T$ such that 
\begin{enumerate}
\item $L|_{X\times \{t\}}\in \Pic^0(X)$ for all $t\in T$ (If $T$ is connected, it suffices to check that there exists $t\in T$ such that this is true); and
\item $L|_{\{0\}\times T}$ is trivial;
\end{enumerate} 

then there exists a unique holomorphic $\psi: T\to \hat{X}$ such that $L\isom (id\times \psi)^* \mathcal{P}$. That is, $\mathcal{P}$ factors through $\psi$. (The proof of this uses \emph{Zariski's main theorem} and a more general Seesaw Principle.)

\end{itemize}

\subsection{A few applications of Poincar\'{e} bundles}

Let $L_1,L_2$ be line bundles on $X$. We say that $L_1$ is \textbf{analytically equivalent} to $L_2$, denoted by $L_1\sim_{an} L_2$ if there exist a connected complex analytic space $T$, a line bundle $L$ on $X\times T$ and $t_1,t_2\in T$ such that 
$$L|_{X\times \{t_i\}}\isom L_i.$$

\begin{proposition}
Let $L_1,L_2$ be line bundles on a complex torus $X=V/\Lambda$.
The following are equivalent:
\begin{enumerate}
\item $L_1\sim_{an} L_2$.
\item $L_1\bigotimes L_2^{-1}\in \Pic^0(X)$.
\item $\varphi_{L_1}=\varphi_{L_2}$.
\item $c_1(L_1)=c_1(L_2)$.
\end{enumerate}

\end{proposition}

For a complex torus $X$, we use $\Pic^H(X)$ to denote the classes in $\Pic(X)/\Pic^0(X)$.

\begin{corollary}
Let $X=V/\Lambda$ be a complex torus. Then
there is a correspondence between analytic equivalence classes of $X$ and $\Pic^H(X)$.
\end{corollary}

Now we prove the proposition.

\begin{proof}
We already know that (2), (3) and (4) implies one and other. To show that (2) implies (1), assume that $L_1\bigotimes L_2^{-1} \in \Pic^0(X)$ and $p:X\times \hat{X}\to X$ is the projection map. Then $L=p^* L_2\bigotimes \mathcal{P}$ is a line bundle on $X\times \hat{X}$ such that 
$L|_{X\times \{0\}}\isom L_2$ and $L|_{X\times \{L_1\bigotimes L_2^{-1}\}}\isom L_1$. 

Now to show that (1) implies (4), suppose $L_1\sim_{an} L_2$, that is, there exist complex analytic space $T$, a line bundle $L$ on $X\times T$ such that $L|_{X\times \{t_i\}}\isom L_i$. Consider $T\to H^2(X,\ZZ)$ given by $t\mapsto c_1(L|_{X\times \{t\}})$ is continuous and the image is discrete and so it is a constant map. This implies that $c_1(L_1)=c_1(L_2)$. 

\end{proof}


\begin{lemma}
Let $L,L'\in \Pic(X)$. Suppose $L$ is non-degenerate, then $L\sim_{an} L'$ if and only if there exists $x\in X$ such that $L'\isom t_x^* L$. 
\end{lemma}

\begin{proof}
The ``if'' direction is always true. To show the ``only if'' direction, $L\sim_{an} L'$ is equivalent to saying that $L'\bigotimes L^{-1}\in \Pic^0(X)$. But since $L$ is non-degenerate, $\varphi_L:X\to \Pic^0(X)$ defined by $x\mapsto t_x^*L\bigotimes L^{-1}$ is surjective. So there exists $x\in X$ such that $t_x^*L \bigotimes L^{-1}\isom L'\bigotimes L^{-1}$. 
\end{proof}

Given a homomorphism $f:X\to \hat{X}$, does there exist a line bundle $L$ such that $f=\varphi_L$?

\begin{theorem}
Let $X=V\Lambda$ be a complex torus and $f:X\to \hat{X}$ a homomorphism with $f_{an}:V\to \overline{\Omega}$. The following are equivalent:
\begin{enumerate}
\item $f=\varphi_L$ for $L\in \Pic(X)$.
\item $F:V\times V\to \CC$ defined by 
$$(v,w)\mapsto f_{an}(v)(w)$$
is Hermitian.
\end{enumerate}
\end{theorem}

The proof to this theorem is pretty elementary, straight-forward and uses the following lemma:
\begin{lemma}
Let $M\in \Pic(X)$ and $n\in \ZZ$. Then the following are equivalent:
\begin{enumerate}
\item $M=L^n$ for some $L\in \Pic(X)$.
\item $X[n]\subset K(M)$ where $K(M)$ is the kernel of $\varphi_M:X\to \hat{X}$. 
\end{enumerate}

\end{lemma}

\subsection{The Poincare-Bundle as a Biextension} \label{section:biextension}
A Poincar\'{e} bundle $\mathcal{P}$ is a biextension of $X\times \hat{X}$ by $\CC^\times$. Then we will define an object $Bi-ext(B\times C,A)$. 



\begin{paragraph}
{}All abelian varieties in this lecture are over the complex numbers. The following definition of biextensions can be found for instance in Mumford's paper 'Biextensions of Formal Groups' \begin{definition} Let $A$, $B$, $C$ be abelian groups. A biextension of $B\times C$ by $A$ is a set $G$ along with 
\newline
1. An action of $A$ on $G$.
\newline
2. A surjective map $ \pi: G\rightarrow B\times C$, \begin{equation*} \pi(g)=(\pi_B(g), \pi_C(g)) \end{equation*} which induces a bijection $G/A\xrightarrow{\sim} B\times C$
\newline
3. Maps \begin{equation*}
+_1: G\times_B G\rightarrow G
\end{equation*}
\begin{equation*}
+_2: G\times_C G\rightarrow G
\end{equation*}
so that the following conditions are satisfied
\newline 
1. $\forall b\in B$, the fibre over $b\times C$ in $G$, $G_b':=\pi_B^{-1}(b)=\pi^{-1}(b\times C)$ is an abelian group with respect to the restriction of $+_1$. $\pi_C$ is a surjective homomorphism of $G_b'$ onto $C$, the kernel of $\pi_C$ is isomorphic to $A$. 
\newline 
2.Likewise, the fibre $G_c$ over $B\times c$ for $c\in C$ is an abelian group with respect to the restriction of $+_2$. $\pi_B$ is a surjective homomorphism of $G_c$ onto $B$, the kernel of $\pi_B$ is isomorphic to $A$.
\newline 
3. Given $x,y,u,v\in G$, with
\begin{equation*}
\pi(x)=(b_1,c_1), \pi(y)=(b_1,c_2), \pi(u)=(b_2,c_1), \pi(v)=(b_2,c_2)
\end{equation*}
the following compatibility relation holds
\begin{equation*}
(x +_1 y)+_2 (u +_1 v)=(x +_2 u) +_1 (y+_2 v)
\end{equation*}
($G\times_B G$ is the fibred product $G\times_B G:=\{(g_1,g_2)\in G \times G\mid \pi_B(g_1)=\pi_B(g_2)\}$, likewise, the set $G\times_C G:=\{(g_1,g_2)\in G \times G\mid \pi_C(g_1)=\pi_C(g_2)\}$)
\end{definition}
The definition of a biextension seems hard to grasp at first glance. The example of the Poincare Bundle with the zero-section removed as a bi-extension of an abelian variety and its dual by $\CC^{\times}$ should be understood to put the definition in perspective. Let $X=V/\Lambda$ be an abelian variety and $\hat{X}=\Omega/\hat{\Lambda}$ be the dual abelian variety. Let $P\rightarrow X\times \hat{X}$ be the Poincare-Bundle on $X\times \hat{X}$. The example we have in mind is that of the Poincare Bundle with the zero-section removed, ie, $A=\CC^{\times}$, $B=X$, $C=\hat{X}$ and $G=P/\{0\}$ with $\pi: P/\{0\}\rightarrow X\times \hat{X}$ the projection map restricted to the complement of the zero section, $A$ acts by scalar multiplication. We note in passing that the trivial $\CC^{\times}$ bundle on a vector space $W$, $L_0:=\CC^{\times}\times W$ is an abelian group with group operation \begin{equation*}
(l_1,w_1)+(l_2,w_2):=(l_1 l_2, w_1+w_2)
\end{equation*}
with identity $(1,0)$ and inverse $(l,w)^{-1}=(\frac{1}{l},-w)$. If $\Pi:L\rightarrow W/\Lambda$ is any line bundle on an abelian variety, then $\Pi^*(L/\{0\})\simeq \CC^{\times}\times W$ is an abelian group and this group structure descends to a natural group structure on $L/\{0\}$.
$G_L\simeq L\{0\}$ by hypothesis, $G_x'$ is a line bundle on $\hat{X}$ with zero section removed. Points of $G\times_C G$ (resp $G\times_B G$) correspond to pairs of points $(l_1,l_2)$ on a line bundle over the abelian variety $X$ (resp $\hat{X}$), the maps $+_1$ and $+_2$ are determined so as to correspond to the group operations on the line bundles with zero section removed. The reader need not work out condition 3, it is in fact a nontrivial result which follows from Lang Duality.
\end{paragraph}

\begin{paragraph}
{}Equivalence classes of biextensions can be suitably expressed in the context of cohomology, we do not however pursue this theme any further.
\end{paragraph}

\subsection{Cohomologies of Line Bundles on Complex Tori}
\begin{paragraph}
{} We will now proceed to discuss the notions of characteristics of line bundles $L$ on $X$, theta-functions as sections of line-bundles and more generally describe all the cohomology groups $H^i(X,L)$. We shall then prove some vanishing theorems for cohomology and compute the alternating sums of the cohomological dimensions from which we can deduce Riemann-Roch.
\end{paragraph}
\begin{paragraph}
{} Fix $H\in NS(X)$, let $\text{Pic}^H(X)$ denote the line bundles on $X$ with chern class $H$. Given a suitable decomposition of $\Lambda=\Lambda_1\oplus \Lambda_2$ (which are in some way orthogonal) we can distinguish a line-bundle $L_0\in Pic^H(X)$. If $H$ is nondegenerate, $L\in Pic^H(X)$ is a translate $L=t_c^* L_0$ and $c$ is called the characteristic of $L_0$ with respect to the decomposition of $\Lambda$. This will allow us to explicitly describe $K(L)=\text{ker} \phi_L$. Let $E=\text{Im} H$, this is a $\ZZ$ valued alternating form. 
\begin{lemma}
Suppose that $2g$ is the rank of the lattice $\Lambda$.
There exists a $\ZZ$-basis for $\Lambda$, 
$\mathit{U}=\{\lambda_1,\dots, \lambda_g, \mu_1,\dots, \mu_g \}$ such that the matrix for $E$ wrt $\mathit{U}$ is

\end{lemma}
\begin{proof}
Pick any basis to begin with. Since $H$ is hermitian, $E$ is skew symmetric, so the matrix for $E$ in this arbitrary basis looks like
\begin{equation*} \left(\begin{array}{cc}
F& A\\
-A^T & G\end{array}
\right )
\end{equation*}
where $F$ and $G$ are also skew symmetric. It's easy to see that we may further assume that $F=G=0$. By row and column operations over $\ZZ$ we may reduce $A$ to a diagonal matrix, over $\ZZ$ we essentially use the fact that the gcd of two numbers can be expressed as a linear combination of these numbers. So $\exists U, V\in GL_n(\ZZ)$ such that $UAV=\text{diag}(d_1,d_2,\dots, d_g)$ with $d_i$ dividing $d_{i+1}$ (note that in the case where we work over a field we may in fact insist that $V=U^{-1}$).
\begin{equation*} \left(\begin{array}{cc}
U& 0\\
0 & V^T\end{array}
\right )
\left(\begin{array}{cc}
0& A\\
-A^T & 0\end{array}
\right )
\left(\begin{array}{cc}
U^T& 0\\
0 & V\end{array}
\right )
=\left(\begin{array}{cc}
0& D\\
-D & 0\end{array}
\right )
\end{equation*}
\end{proof}

$(d_1,\dots ,d_g)$ is uniquely determined by $E$ or $H$ or $L$.
\begin{definition} We call the tuple $(d_1,\dots, d_g)$ the type of $E$ or $H$ or $L$ and if all the $d_i=1$ we call $L$ a principle polarization.
\end{definition}
We see that $K(\Lambda)=\text{ker} \phi_L\simeq K_1\oplus K_2$ with $K_i\simeq \oplus \ZZ/d_i \ZZ$. If all $d_i>0$ then $H$ or $L$ or $E$ is non-degenerate.
\begin{definition}
A basis $\{\lambda_1,\dots, \lambda_g, \mu_1,\dots, \mu_g\}$ be as before giving rise to the matrix $\left(\begin{array}{cc}
0& D\\
-D & 0\end{array}
\right )$ is called a canonical or symplectic-basis for $\Lambda$. A sub-lattice $\Lambda_1\subset \Lambda$ is called totally isotropic for $E$ if $E(\lambda, \lambda')=0$ $\forall \lambda, \lambda'\in \Lambda$.
\end{definition}

\begin{definition}
{} A decomposition $\Lambda=\Lambda_1\oplus \Lambda_2$ is called a decomposition for $E$ or $H$ or $L$ if both $\Lambda_1$ and $\Lambda_2$ are totally isotropic.
\end{definition}
\begin{definition}
A decomposition $V=V_1\oplus V_2$ of $V$ into real vector spaces such that $(V_1\cap \Lambda)\oplus (V_2\cap \Lambda)$ is a decomposition for $\Lambda$ is called the decomposition of $V$ for $E$ or $H$ or $L$.
\end{definition}
Let $H\in NS(X)$, $V=V_1\oplus V_2$ a decomposition for $H$, define $\chi_0:V\rightarrow T_1$ by 
\begin{equation*}
\chi_0(v)=e^{\pi i Im H(v_1,v_2)}=e^{\pi i E(v_1,v_2)}
\end{equation*}
where $v=v_1+v_2$ with $v_1\in V_1$ and $v_2\in V_2$. It is easily seen that for $v,w\in V$,
\begin{equation*}
\chi_0(v+w)=\chi_0(v)\chi_0(w) e^{\pi i E(v,w)}e^{-2\pi i E(v_2,w_1)}
\end{equation*}
(keep in mind that $E(\lambda, \mu)\in \ZZ$ for $\lambda, \mu\in \Lambda$)
\begin{corollary}
$(\chi_0)_{\mid \Lambda}$ is a semicharacter for $H$
\end{corollary}
\begin{definition}
$L_0:=L(H,\chi_0)\in \text{Pic}^H(X)$ is a distinguished element of $\text{Pic}^H(X)$ with respect to the decomposition of $V$.
\end{definition}
\end{paragraph}

\begin{paragraph}{}Throughout this lecture $X=V/\lambda$ is a complex torus.\begin{lemma}
Let $V=V_1\oplus V_2$ be a decomposition for $H\in NS(X)$, assume that $H$ is non-degenerate 
\begin{enumerate}
\item $L_0=L(H,\chi_0)$ is the unique element in $Pic^H(X)$ where the semi character $\chi_0$ is trivial on $\Lambda_1=V_1\cap \Lambda$ and $\Lambda_2=V_2\cap \Lambda$.
\item For $L\in \Pic^H(X)$, there is a $c\in V$ uniquely determined up to translation by $\Lambda(L)$ such that if $\bar{c}$ denotes the class of $c$ modulo $\Lambda(L)$,
\begin{equation*}
L\simeq t_{\bar{c}}^* L_0 
\end{equation*}
\end{enumerate}
\end{lemma}
\begin{proof}
\begin{enumerate}
\item We saw last time that $\chi_0$ restricted to $\lambda$ was a character, since $\chi_0(\lambda):=e^{\pi i E(\lambda_1,\lambda_2)}$ and each $V_i$ is totally isotropic, it follows that $\chi_0$ is trivial on both $\Lambda_1$ and $\Lambda_2$.
\item
Certainly, since $L$ and $L_0$ have the same Chern class $H$ it follows that there is a $c\in V$ such that $L\simeq t_{c}^* L_0$, $c$ modulo $\Lambda$ is thus uniquely determined upto the kernel of $\phi_L=\Lambda(L)/\Lambda$, and thus $c\in V$ is uniquely determined up to $\Lambda(L)$.
\end{enumerate}
\end{proof}
\begin{definition} Let $L$ be a non-degenerate line bundle. $c$ modulo $\Lambda(L)$ is called the class of the line bundle $L$ with respect to the decomposition of $V=V_1\oplus V_2$ into totally isotropic spaces for $H$.
\end{definition}
We shall discuss proceed to describe the global sections of line-bundles. In doing so, we need to extend the canonical factor to $V\times V$. Recall that the canonical factor associated to a line-bundle is defined by the equation 
\begin{equation*}
a_L(\lambda, v):=\chi(\lambda) e^{H(v,\lambda)+\frac{1}{2} H(\lambda, \lambda)}
\end{equation*}
\begin{definition}
Let $L$ be a non-degenerate line bundle with characteristic $c$. We extend the canonical factor $a_L$ to a function on $V\times V$ using the characteristic of $L$, 
\begin{equation*}
a_L(u,v):=\chi_0(u) e^{2\pi i E(c,u)} e^{\pi(H(u,v)+\frac{1}{2} H(u,u))}
\end{equation*}
\end{definition}
Note that the term $\chi_0(u) e^{2\pi i E(c,u)}$ restricts to $\chi$ on $\lambda\times \lambda$ since $c$ is the characteristic and translation by $c$ changes the character by $e^{2\pi i E(c,u)}$. 
\begin{proposition}\label{thelemma}
Let $L$ be a line bundle with $H=c_1(L)$ non degenerate, let $V=V_1\oplus V_2$ be a decomposition for $V$ into isotropic subspaces with respect to $L$, for $u,v,w\in V$,
\begin{enumerate}
\item $a_L(u,v+w)= a_L(u,v) e^{\pi H(w,u)}$
\item $a_L(u+v,w)=a_L(u,v+w) a_L(v,w) e^{2\pi i E(u_1, u_2)}$
\item $a_L(u,v)^{-1}= a_L(-u,v) \chi_0(u)^{-2} e^{-\pi H(u,u)}$
\item If $L'=t_w^* L$ then $a_{L'}(u,v)=a_L(u,v)e^{2\pi i E(w,u)}$
\end{enumerate}

\end{proposition}
\begin{proof}
Excercise
\end{proof}
\begin{lemma}
Let $L$ be a non-degenerate line bundle and $V=V_1\oplus V_2$ decomposition of $V$
\begin{enumerate}
\item $\Lambda(L)=\Lambda(L)_1\oplus \Lambda(L)_2$ where $\Lambda(L)_i:=V_i\cap \Lambda(L)$
\item $K(L)=K_1\oplus K_2$ where $K_i=\Lambda(L)_i/\Lambda_i\simeq \oplus_{j=1}^g \mathbb{Z}/ d_i \mathbb{Z}$ where $(d_1,\dots, d_g)$ is the type of $L$.
\end{enumerate}
\end{lemma}
\begin{proof}
$(2)$ follows from $(1)$ it suffices to prove $(1)$. The containment $\Lambda(L)\supseteq \Lambda(L)_1\oplus \Lambda(L)_2$ is clear since we are using a decomposition of $V$ into isotropic spaces. $v\in \Lambda(L)$, $v=v_1+v_2$ a decomposition for $v$ we need to show that $v_1,v_2\in \Lambda(L)$, indeed we need only show that $v_1\in \Lambda(L)$. Note that $E(v_1, \lambda_1)=E(v_2, \lambda_2)=0$. Hence,
\begin{equation*}
E(v_1,\lambda)=E(v_1,\lambda_2)=E(v, \lambda_2)\in \ZZ
\end{equation*}
thus $v_1\in \Lambda(L)$.
\end{proof}
Now define two intermediate lattices between $\Lambda$ and $\Lambda(L)$, namely, $\Lambda'=\Lambda(L)_1\oplus \Lambda_2$ and $\Lambda''=\Lambda_1\oplus \Lambda(L)_2$, let $X'=V/\Lambda'= X/K_1$ and $X''=V/\Lambda''=X/K_2$ (with $K_i$ defined in Lemma 1.5). Let $P_i:X\rightarrow X/K_i=:X_i$ be the quotient isogenies. The restrictions of $a_L$ to $\Lambda'\times V$ and $\Lambda''\times V$ are $1-cocycles$ defining line bundles $M_1\rightarrow X_1$ and $M_2\rightarrow X_2$ which pull back to $L$ under $P_1$ and $P_2$ to the line bundle $L$ on $X$. Moreover since $P_i$ comes from taking a $v$ modulo $\Lambda$ to $v$ $\Lambda'$ or $\Lambda''$, the characteristic of $M_i$ is the same as the characteristic of $L$.
\end{paragraph}
\subsection{Theta Functions}
\begin{paragraph}
{}Assume $H$ is positive definite, we shall describe a basis for $H^0(X,L)$. Let $H=c_1(L)$ be the chern class of $L$. Let $X=V/\Lambda$ and $\pi:V\rightarrow X$ the exponential map. From previous lectures recall that the global sections may be described as functions $\vartheta:V\rightarrow \mathbb{C}$ with automorphy factor $f\in Z^1(\Lambda, H^0(V, O_V^*))$, ie, satisfying
\begin{equation*}
\vartheta(v+\lambda)=f(v,\lambda) \vartheta(v)
\end{equation*}
Note that changing $f$ by a co-boundary amounts to replacing the space space with an isomorphic vector space.
\end{paragraph}
\subsection{Classical Factors of Automorphy}
\begin{paragraph}
{} We are trying to get a more suitable factor replacing $H$ which is assumed to be positive definite. Let $L(H,\chi)$, $V=V_1\oplus V_2$ a decomposition for $L$, also $E=Im H$. 
\begin{lemma}
$V=V_2\oplus i V_2$ and so $V_2$ generates $V$ as a $\mathbb{C}$ vector space. 
\end{lemma}
\begin{proof}
$U=V_2\cap i V_2 $ is a $\mathbb{C}$ subspace. For all $v,w\in U$, we have that 
\begin{equation*}
H(v,w)=E(iv,w)+iE(v,w)=0
\end{equation*}
since $iv$ is also in $U\subseteq V_2$. $H$ is positive definite and so $U=0$. Since $V_2$ has dimension $g$ so does $iV_2$ and so $V=V_2\oplus i V_2$.
\end{proof}
Since $E=0$ on $V_2$, then $H=\text{Re} H=E(i.,.)$ is a symmetric real valued form on $V_2$. 
\end{paragraph}
\begin{paragraph}
{}Let $B:V\times V\rightarrow \CC$ be the $\CC$ bilinear extension of $H_{\mid V_2\times V_2}$, note that since $V=V_2\oplus i V_2$ a unique $\CC$ bilinear extension $B$ exists. $H-B$ is then a hermitian form, it satisfies the properties 
\begin{enumerate}
\item $(H-B)(v,w)=\begin{cases}
0 &\mbox{if } w\in V_2\\
2i E(v,w) &\mbox{if } v\in V_2
\end{cases}$
\item $\text{Re}(H-B)$ is positive definite on $V_1$
\end{enumerate}
\begin{definition}
The classical factor of automorphy $e_L:\Lambda\times V\rightarrow \CC^{\times}$ is defined as
\begin{equation*}
e_L(\lambda, v)=\chi(\lambda) e^{\pi((H-B)(\lambda,v)-\frac{1}{2} (H-B)(\lambda, \lambda))}=a_L(\lambda,v) (\frac{e^{\frac{\pi}{2} B(v,v)}}{e^{\frac{\pi}{2} B(v+\lambda,v+\lambda)}})
\end{equation*}
\end{definition}
\begin{definition}
Classical theta functions are the functions with automorphy factor $e_L$.
\end{definition}
\begin{enumerate}
\item When the type is $(1,1,\dots, 1)$ the polarization is said to be principal as noted earlier. The classical theta function associated to a principal polarization is unique up to scalar multiple. These were studied by Riemann, so they really are classical.
\item $e_L(\lambda, \lambda_2)=1$ allows us expand theta functions in fourier series.
\end{enumerate}

\subsection{Global sections of a positive definite line bundle}
\begin{definition}
Let $L$ be a line bundle with $E:=\text{Im} c_1(L)$ and type $(d_1,\dots, d_g)$. The Pfaffian of $E$, $Pf(E):=d_1\dots d_g$. This is indeed $\sqrt{\text{det} E}$.
\end{definition}
\end{paragraph}

Later we will see that 
If $L$ is a positive definite line-bundle on $X$, then, $h^0(X,L)=Pf(E)$.


\begin{lemma}
$H^0(X,L)\isom H^0(X,L_0)$.
\end{lemma}

\begin{proof}
The idea is to present $H^0$ with classical theta functions. We claim that $H^0(X,L)\to H^0(X,L_0)$ by $\vartheta(\cdot)\mapsto e^{\pi(H-B)(\cdot,C)}\vartheta(\cdot-C)=\tilde{\vartheta}(\cdot)$ is an isomorphism of $\CC$-vector spaces. Once we have that, then we only need to show $\tilde{\vartheta}(\cdot)$ is a theta function for the $1$-cocycle $e_{L_0}(\cdot,\cdot)$. This is an easy computation, and the statement follows from the fact that if $L=t_\epsilon^* L^0$ then $e_L(\lambda,v)=e_{L_0}(\lambda,v)e^{2\pi i E(C,\lambda)}$ and \ref{thelemma}.
\end{proof}

\begin{lemma}
$h^0(X,L_0)\leq Pf(E)$.
\end{lemma}

\begin{proof}
Let $\vartheta\in H^0(X,L_0)$ be a classical theta function. That is $\vartheta(v+\lambda)=e_{L_0}(\lambda,v)\vartheta(v)$ for all $\lambda\in \Lambda$ and $v\in V$. If $\lambda_2\in \Lambda_2$ (where $\Lambda=\Lambda_1\bigoplus \Lambda_2$), then $e_{L_0}(\lambda_2,v)=1$, and so $\vartheta(v+\lambda_2)=\vartheta(v)$ for all $\lambda_2\in \Lambda_2, v\in V$. And therefore, we have a Fourier series! The Fourier series for $\vartheta$ is of the following form: 
$$\vartheta(v)=\sum_{\lambda\in \Lambda(L)_1}\alpha_\lambda e^{\pi (H-B)(v,\lambda)}.$$
One way to get this is by properties of $H-B$ from earlier discussions. Alternatively, one can use the fact that there exist suitable coordinates $V=(v_1,\cdots, v_g)^T$ and $\lambda=(\lambda_1,\cdots,\lambda_g)^T$ such that $e^{\pi(H-B)(v,\lambda)}=e^{-2\pi i v^T\lambda}$. 
Also note that for $\lambda_1\in \Lambda_1$, we have $\vartheta(v+\lambda_1)=e_{L_0}(\lambda_1,v)\vartheta(v)$. Therefore, 
$\alpha_{\lambda-\lambda_1}=\alpha_\lambda e_{L_0}(\lambda_1,0)^{-1}e^{\pi(H-B)(\lambda_1,\lambda)}$ for all $\lambda_1\in \Lambda_1,\lambda\in \Lambda(L)_1$. 
Now if we fix $a_\lambda$ for $\lambda\in \Lambda(L)_1$ varying over representatives of $\Lambda(L)_1/\Lambda_1=K(L)_1$, then all other $a_\lambda$ are fixed by our above expression of $\alpha_{\lambda-\lambda_1}$. Therefore, $h^0\leq |K(L)_1|=d_1\cdots d_g$. 
\end{proof}

Remark: When $X=V/\Lambda$ is a complex torus and $H\in NS(X)$ is positive definite of type $(d_1,\cdots,d_g)$ with symplectic basis $\{\lambda_1,\cdots,\lambda_g;\mu_1,\cdots,\mu_g\}$ for $\Lambda$ for $Im(H)=E$. Then $D=diag(d_1,\cdots, d_g)$ with $d_i>0$ and $V=V_1\oplus V_2=\langle \lambda_i\rangle\bigoplus \langle \mu_i\rangle$. Define $e_i=\frac{1}{d_i}\mu_i$. We have seen that $\{e_i:\ 1\leq i \leq g\}$ is a $\CC$-vector space basis. Consider the period matrix with respect to $\{e_i\}$ for $V$ and $\{\lambda_i,\mu_i\}$ for $\Lambda$ such that $\Pi = (Z|D)_{g\times 2g}$ for some $Z\in M_{g\times g}(\CC)$. One can check that 
\begin{enumerate}
\item $Z=Z^T$ and $Im(Z)>0$. That is, $Z$ is in {\em Siegel's upper half-space of degree} $g$. 
\item with respect to $\{e_1,\cdots,e_g\}$ basis for $V$:
\begin{itemize}
\item  $(Im Z)^{-1}$ is the matrix for $H$.
\item  $B(v,w)=v^T(Im Z)^{-1}w$.
\item $(H-B)(v,w)=-2iv^Tw_1$ where $w=w_1+w_2\in V_1\bigoplus V_2$.
\end{itemize} 
\end{enumerate}

Next, we want to find an explicit basis for $H^0(X,L)$. Suppose $L\in \Pic(X)$ with $char(L)=c$ with respect to $V_1\bigoplus V_2$. Our goal is to express one $\vartheta$ very explicitly, then show that we can produce a lot more by twisting. We define $\vartheta^c:V\to \CC$ by 
$$v\mapsto e^{-\pi(H(v,c)+1/2H(c,c)+1/2 B(v+c,v+c))}\times \sum_{\lambda\in \Lambda_1} e^{\pi((H-B)(v+c,\lambda)-1/2(H-B)(\lambda,\lambda))}.$$

\begin{theorem}
$\vartheta^c$ is a canonical theta function for $L=t_{\overline{\CC}}^* L_0$. 
\end{theorem}

\begin{proof}
We sketch the proof. 
\begin{itemize}
\item $\vartheta^c$ is holomorphic on $V$. To see this, we need to show that $f(v)=\sum_{\lambda\in \Lambda_1} |e^{\pi((H-B)(v+c,\lambda)-1/2(H-B)(\lambda,\lambda))}|$ converges uniformly on every compact subset of $V_1$. This can be achieved by fixing a norm $||\cdot||:V\to \RR$ such that $||\Lambda||\subset \ZZ$. Since $Re(H-B)$ is positive definite on $V_1$ and $V_1$ is discrete, there exists $R>0$ such that $|e^{1/2(H-B)(\lambda,\lambda))}|\geq e^{-R||\lambda||^2}$. Also, for all $r>0$, there exists $R'>0$ such that if $||v||<r$, then $|e^{\pi(H-B)(v,\lambda))}|<e^{R'||\lambda||}$. So 
$$f(v)\leq \sum_{\lambda\in \Lambda_1} (e^{R'||\lambda||}/e^{R||\lambda||^2})<\infty.$$

\item We need to check the equation of the automorphy. That is, we want to show that $\vartheta^c(v+\lambda)=a_L(\lambda,v)\vartheta^c(v)$ for all $(\lambda,v)\in \Lambda\times V$. For $c=0$, this boils down to \ref{thelemma}. For $c\not=0$, it follows from 
$$\vartheta^c(v)=e^{-\pi(H(v,c))+1/2H(c,c)}\vartheta^0(v+c).$$


\end{itemize}
\end{proof}

Remark: The proof actually shows that $\vartheta^c$ is a canonical theta function for $M_2$ (descent of $L$ to $V/(\Lambda_1\bigoplus \Lambda(L)_2)$). 

So far we constructed \emph{one} canonical theta for $L$, that is, an element of $H^0(X,L)$. We can get more as follows: for $\overline{\omega}\in K(L)=\Lambda(L)/\Lambda$, define 
$$\vartheta_{\overline{\omega}}^c(\cdot):=a_L(\omega,\cdot)^{-1}\vartheta^c(\cdot + \omega)$$
where $\omega\in \Lambda(L)$ is any lift of $\overline{\omega}$. It is easy to check that this definition is independent of the choice of $\omega$. 

\begin{lemma}
For all $\overline{\omega}\in K(L)$, $\vartheta_{\overline{\omega}}^c$ is a \emph{canonical} theta function for $L$. 
\end{lemma}

\begin{proof}
It follows from \ref{thelemma}.
\end{proof}

\begin{theorem}
Let $L=L(H,\chi)$ be a positive definite line bundle on $X$ and $C=char(L)$ be the characteristic with respect to $V=V_1\bigoplus V_2$ for $L$. The set of canonical theta functions
$\{\vartheta_{\overline{\omega}}^c:\ \overline{\omega}\in K(L)_1\}$ is a basis of canonical theta function for the $\CC$-vector space $H^0(X,L)$. 
\end{theorem}

\begin{proof}
First, we already know $h^0(X,L)\leq Pf(E)$. So it suffices to show that $\vartheta_{\overline{\omega}}^c$ is independent of $(\# K(L)_1=Pf(E))$.

Next, the expression from earlier generalizes to $\vartheta_{\overline{\omega}}^c(v)=e^{-\pi(H(v,c)+1/2H(c,c))}\theta_{\overline{\omega}}^0(v+c)$. So we may further assume that $c=0$.

Furthermore, by $e_L(\lambda,v)=a_L(\lambda,v)\frac{e^{\pi B(v,v)}}{e^\pi B(v+\lambda,v+\lambda)}$. Therefore, we see that $\vartheta_{\overline{\omega}}^c(v)=e^{-\pi/2 B(v,v)}\vartheta_{\overline{\omega}}^c(v)$ is classical. So it suffices to show that $\{\vartheta_{\overline{\omega}}^c(\cdot):\ \overline{\omega}\in K(K)_1\}_{c=0}$ with is independent.

Let $\omega_1,\cdots,\omega_N\in \Lambda(L)_1$ be representations for $K(L)_1$. We will show that $\vartheta_{\overline{\omega_1}}^0(\cdot)=e^{-\pi/2 B(\cdot,\cdot)}\vartheta_{\overline{\omega_i}}^0$ with $1\leq i\leq N$ are linearly independent. 

Note that 
$$\vartheta_{\overline{\omega_i}}^0=e^{-\pi/2B(v,v)}\vartheta_{\overline{\omega_i}}^0(v)=\cdots
=\sum_{\lambda\in \Lambda_1-\omega_1} e^{-\pi/2(H-B)(\lambda,\lambda)}e^{\pi (H-B)(v,\lambda)}.$$
Recall that a general Fourier series is of the form
$$\sum_{\lambda\in \Lambda(L)_1} \alpha_\lambda e^{\pi (H-B)(v,\lambda)}.$$
Note that $\vartheta_{\overline{\omega_i}}^0(\cdot)$ only have coefficients in the coset $-\omega_i+\Lambda_i$ (in $\Lambda(L)_1/\Lambda_1$) and cosets are disjoint. So they are independent. 
\end{proof}



\begin{corollary}
$h^0(L)=d_1\cdots d_g$.
\end{corollary}

\begin{corollary}
For any $L\in L(H,\chi)$ and $L'\in L(H,\chi')$ with $c=char(L)$ and $c'=char(L)$, define $\tau:V\to \CC^\times$ by 
$$v\mapsto e^{\pi i Im H(c',c)-\pi H(v,c-c')-\pi/2 H(c'-c,c'-c)},$$
then the map $H^0(X,L)\to H^0(X,L')$ by $\vartheta\mapsto \tau t_{c'-c}^\times \vartheta$ is an isomorphism. Moreover, it is diagonal with respect to the bases given above. 
\end{corollary}

\subsection{Positive semi-definite line bundles}
Recall that if $L$ is degenerate, then $K(L)=\ker(\varphi_L)$ is not finite. Equivalently, $\Lambda(L)/\Lambda$ is infinite, where $\Lambda(L)=\{v\in V:\ Im H(v,\lambda)\in \ZZ\ \forall \lambda\in \Lambda\}$. Therefore, they can not be lattices of the same rank.
Let $\Lambda(L)_0$ be the connected component of $\Lambda(L)$ containing $0$,m which is also a subspace of $V$. By an earlier fact that $\varphi_L^{an}(v)=H(v,\cdot)$, we have $\Lambda(L)_0=\{v\in V: H(v,x)=0\ \forall x\in V\}$. Therefore $K(L)_0=\Lambda(L)_0/\Lambda(L)_0\cap \Lambda$ is a subtorus of $X$. 

Now define $\overline{X}=X/K(L)_0=\overline{V}/\overline{\Lambda}$ where $\overline{V}=V/\Lambda(L)_0$ and $\overline{\Lambda}=\Lambda/(\Lambda(L)_0\cap \Lambda)$. Let $p:X\to \overline{X}$ be the projection map.

\begin{lemma}
\noindent
\begin{enumerate}
\item There exists $\overline{L}$ on $\overline{X}$ with $L\isom p^* \overline{L}$ if and only if $L|_{K(L)_0}$ is trivial.
\item If $\overline{L}$ exists, then $h^0(X,L)=h^0(\overline{X},\overline{L})$ and $\overline{L}$ is non-degenerate. 
\end{enumerate}
\end{lemma}

\begin{theorem}
If $L=L(H,\chi)$ is positive semi-definite on $X$, then $h^0(X,L)=Pf_{reduced}(E)$ if $L|_{K(L)_0}$ is trivial and $0$ otherwise, where $Pf_{reduced}(E)=1$ if all $d_i=0$ and $\prod_{d_i\not= 0}d_i$ otherwise.
\end{theorem}

Now we will go into some deeper results. For $L\in \Pic(X)$ on $X=V/\Lambda$ a $g$-dimensional complex torus with $H=c_1(L)$. Assume $H$ has $r$ positive and $s$ negative eigenvalues. Note then that $r+s\leq g$. Our goal is to prove the following theorem:

\subsection{All cohomologies and analytic Riemann-Roch}

\begin{theorem}[Mumford-Kempf-Deligne,$\cdots$]
\noindent
\begin{enumerate}
\item $H^q(X,L)=0$ if $q<s$ or $q>g-r$.  
\item $h^q(X,L)=\binom{g-s-r}{q-s}h^s(x,L)$ for $s\leq q\leq g-r$.
\item $h^s(X,L)=Pf_{reduced}(E)$ for $L|_{K(L)_0}$ trivial and $0$ otherwise.
\end{enumerate}
\end{theorem}

\begin{corollary}
\noindent
\begin{enumerate}
\item If $L$ is non-degenerate, then $r+s=g$. So cohomology is non-trivial only for $q=s=g-r$. We call such $q$ the \emph{index of $L$} and denote it by $i(L)$.

\item If $h^s\not=0$, then $H^q(X,L)$ are nontrivial for all $s\leq q\leq g-r$.
\item If $L$ is positive-definite, then $r=g$ and $s=0$ and so only $H^0$ is non-trivial.
\item If $L$ is positive-semi-definite and if $L|_{K(L)_0}$ is trivial, then $f<g$ and $s=0$ and so $H^0,\cdots,H^{g-r}$ are all non-trivial.
\end{enumerate}
\end{corollary}

Recall that the Riemann-Roch Theorem is a statement relating algebraic objects 
$$\chi(X,L):=\sum_{i=s}^{g-r}(-1)^i h^i(X,L)$$
to topological objects.
Now we state a deeper result, by Deligne.

\begin{corollary}[Analytic Riemann-Roch theorem for complex tori]
As usual, define
$$\chi(X,L):=\sum_{i=s}^{g-r}(-1)^i h^i(X,L).$$
Assume $c_i(L)=H$ has $s$ negative eigenvalues, then 
$$\chi(X,L)=(-1)^s Pf(E)$$
where $E=Im(H)$ and $Pf(E)=d_1\cdots d_g$.

\end{corollary}

\begin{proof}
If $L_{K(L)_0}$ is non-trivial, then there is nothing to prove.
We may hence assume that 
$$\chi(X,L)=\sum_{q=s}^{g-r} (-1)^q \binom{g-r-s}{q-s} Pf_{reduced}(E).$$
By putting $N=g-r-s$, since $(1-1)^N=\sum_{i=0}^N (-1)^i\binom{N}{i}$ equals to $1$ if $N=0$ and $0$ otherwise, we hence have $\chi(X,L)=(-1)^s Pf(E)$. 

\end{proof}

Remarks: 
\begin{enumerate}
\item $\deg(\varphi_L)=\det(E)=(Pf(E))^2=\chi(X,L)^2$.
\item It is useful (for instance to get an algebraic/geometric statement) to use $\frac{1}{g!}(L^g)$ instead of $(-1)^s Pf(E)$. (Later we will be able to express $\frac{1}{g!}(L^g)$ in terms of $\bigwedge^g c_1(L)$.)
\item For $f:X'\to X$ and $L\in \Pic(X)$, we have $\chi(X',f^* L)=(\deg (f))\chi(X,L)$.
\item It is a good exercise to show that for a Poincar\'{e} bundle $\mathcal{P}$ on $X\times \hat{X}$, we have $h^q(X\times\hat{X},\mathcal{P})=\CC$ for $q=g$ and $0$ otherwise. (Hint: we have given the first chern class $c_i(\mathcal{P})$ explicitly.)
\end{enumerate}

\subsection{Vanishing theorem of Mumford and Kempf}

Recall that we have by $\overline{\partial}$-Poincar\'{e} an exact sequence of sheaves:
$$0\to \Omega_{hol}^p\to \Omega^{p,0}\overset{\overline{\partial}}{\to} \Omega^{p,1}\overset{\overline{\partial}}{\to}\cdots$$
and in particular when $p=0$, we have
$$0\to \mathcal{O}_X\injects \Omega^{0,0}\overset{\overline{\partial}}{\to} \Omega^{0,1}\overset{\overline{\partial}}{\to}\cdots$$
Since $L$ is locally free, we also get exact sequence of sheaves 
$$0\to L \to \Omega^{0,0}(L)\overset{\overline{\partial}}{\to} \Omega^{0,1}(L)\to \cdots$$
with $\Omega^{0,q}(L)=\Omega^{0,q}\bigotimes_{\mathcal{O}_X}L$ which is called the \emph{Dolbeault resolution}. 


By tensoring with the \emph{locally free} sheaf $L$, we get an exact sequence:

When $p=0$:
\[
0\to L \to \Omega^{0,0}(L)\overset{\bar{\partial}=\bar{\partial}\otimes id}{\to} \Omega^{0,1}(L)\to \cdots
\]

where $\Omega^{p,q}(L):=\Omega^{p,q}\bigotimes L$ is the sheaf of $C^\infty$-forms of type $(p,q)$ with values in $L$. Now, we apply $\Gamma(X,-)$ to get the complex 
\[
0\to H^0(X,L)\to \Gamma(X,\Omega^{0,0}(L))\to \Gamma(X,\Omega^{0,1}(L))\to \cdots
\]

and therefore we get cohomology groups $H_{\bar{\partial}}^{0,q}(X,L)$. By Dolbeault theorem, this is isomorphic to $H^q(X,L)$. Moreover, we have a Hodge theorem stating $H^{0,q}_{\bar{\partial}}(X,L)$ is isomorphic to $H^q(L)$ which is the subvector space $\ker \Delta$ of $\Gamma(X,\Omega^{0,q}(L))$, i.e. ``Harmonic forms with values in $L$ with respect to the Laplacian $\Delta$''. 

Question: What is $\Delta$? We need $\Delta=\bar{\partial}\bar{\delta}+\bar{\delta}\bar{\partial}$. What is $\bar{\delta}$? We want $\bar{\delta}$ to be the ``adjoint'' of $\bar{\delta}$, with respect to a global ``inner product'' on $\Gamma(X,\Omega^{0,q}(L))$. Recall previously we have a metric $ds^2$ on $X$ which gives rise to a $(1,1)$-form $\omega$ and hence a $(n,n)$-form $dv=\frac{1}{g!}\bigwedge^g \omega$ which is a volume form. Therefore, we have an inner product 
$$(\varphi,\psi)=\sum\int \varphi_{IJ}\psi_{IJ}dv.$$

However, this is not enough - we need to fix some ``metric data'' on $L$. 

\begin{definition}
A \textbf{Hermitian metric} on a line bundle $L$ is a positive definite Hermitian form on each fibre $L_x$ depending smoothly on $x\in X$. In other words, $h\in \Gamma_{sm}(X,(L\bigotimes L)^v)$ such that 
$$h_P(\eta,\bar{\xi})=\overline{h_P(\xi,\bar{\eta})}$$
and 
$h_P(\xi,\bar{\xi})>0$ if $\xi\not= 0$
for all $\xi,\eta\in L_P$.  
\end{definition}


Irrelevant remark: If $(L,h)$ is  Hermitian line bundle, then the first chern class $c_1(L,h)$ gives a \emph{curvature form} which is also a $(1,1)$-form on $X$. It turns out that $[c_1(L,h)]=c_1(L)$. 

\subsection{Construction on $L\to X$}

Let $L=(H,\chi)$ be a Hermitian line bundle and $X$ be our complex torus. If we consider the $(0,0)$ case as in above, then notice that 
\[
\Gamma(X,\Omega^{0,0}(L))=\{f:V\to \CC:\ f\in C^\infty,\ 
f(x+\lambda)=a_L(\lambda,v)f(v)\ \forall (v,\lambda)\in V\times \Lambda\}
\]
are what we call ``smooth'' theta functions. For $f,g\in \Gamma(X,\Omega^{0,0}(L))$, define $h=\langle f,g\rangle$ by 
\[
h(v)=\langle f,g \rangle (v) = f(v)\overline{g(v)}e^{-\pi H(v,v)}.
\]
It is easy to check that $h(v+\lambda)=h(v)$ and that $h$ is $C^\infty$, so $h$ gives an element of $\Gamma(X,\Omega^{0,0})$. Therefore, $\langle \cdot ,\cdot \rangle: \Gamma(X,\Omega^{0,0}(L))\times \Gamma(X,\Omega^{0,0}(L))\to \Gamma(X,\Omega^{0,0}))$ defines a \emph{Hermitian metric}. 

Recall that a \emph{K\"{a}hler metric} $ds^2$ on $V/\Lambda=X$ is given by the following: fix $\{e_1,\cdots, e_g\}$ a basis for $V$ such that $H=c_1(L)$ with respect to the basis is diagonal. (Hermitian matrices are diagonalizable.) We let $h_1,\cdots, h_g$ diagonal entries of the matrix, that is, let $\{v_1,\cdots,v_g\}$ be the coordinate functions corresponding to $\{e_1,\cdots, e_g\}$. Then 
$H(v,w)=\sum h_i v_i w_i$.

If we fix $K_1,\cdots, K_g\in \RR^{>0}$, then 
\[
ds^2=\sum_{i=1}^g K_i dv_i\otimes d\overline{v_i}
\]
defines a K\"{a}hler metric
 and so the associated $(1,1)$ form $\omega=1\frac{1}{2}Im(ds^2)$ is closed ($d\omega=0$) and given by 
\[
\omega=\frac{i}{2}\sum_{j=1}^g K_j dv_j\bigwedge d\overline{v_j}
\]
and hence we have a volume form 
\[
dv=\frac{1}{g!}\bigwedge^g \omega = (i/2)^g (\prod K_i) dv_1\wedge d\overline{v_1}\wedge dv_2\wedge\cdots
\]
which is also a $(g,g)$-form. Now, we define 
\[
(\cdot, \cdot):\Gamma(X,\Omega^{0,0}(L))\times \Gamma(X,\Omega^{0,0}(L))\to \CC
\]
by 
\[
(f,g)=\int_X \langle f,g\rangle dv
\]

\subsection{The $(0,q)$ case}
Let $\omega=\sum_{|I|=q}\varphi_I d\overline{v_I}, \omega'=\sum_{|I|=q}\psi_I d\overline{v_I}\in \Gamma(X,\Omega^{0,q}(L))$ for $\varphi_I\in \Gamma(X,\Omega^{0,0}(L))$. Define 
\[
(\omega,\omega')=\sum_{|I|=q} K^{-I}(\varphi_I,\psi_I)
\]
where $K^{-I}=\prod_{i\in I} K_i^{-1}$. Recall the notations $\partial_i=\frac{\partial}{\partial v_i}$ and $\overline{\partial_i}=\frac{\partial}{\partial \overline{v_i}}$. Also recall that $\Gamma(X,\Omega^{0,0}(L))=\{f:V\to \CC: f(v+\lambda)=a_L(\lambda,v)f(v)\}$. Since $a_L(\lambda,v)$ is holomorphic, so $\overline{\partial_i}a_L=0$ and so $\overline{\partial_i}$ is a linear operator on $\Gamma(X,\Omega^{0,0}(L))$. So there exists $\overline{\partial}:\Gamma(X,\Omega^{0,q}(L))\to \Gamma(X,\Omega^{0,q+1}(L))$ given by 
\[
\overline{\partial}(\varphi d\overline{v_I})=\sum_i (\overline{\partial_i}\varphi) d\overline{v_i}\wedge d\overline{v_I}.
\]
Note that this is the map on the complex we had. Define $\overline{\delta_i}$ the adjoint of $\overline{\partial_i}$ and $\overline{\delta}$ the adjoint of $\partial$ with respect to $(\cdot,\cdot)$. Explicitly, we have the following lemma.

\begin{lemma}
Let $\varphi\in \Gamma(X,\Omega^{0,0}(L))$.
\begin{enumerate}[(a)]
\item $\overline{\delta_i}\varphi=-\partial_i\varphi+\pi h_i \overline{v_i}\varphi$
\item $\overline{\delta}(\varphi d\overline{v_J})=\sum_{i=1}^{q+1}(-1)^{i-1}\frac{1}{K_{j_i}}(\overline{\delta_{j_i}}\varphi)d\overline{v_{J-j_i}}$ where $J=(j_1<\cdots <j_{q+1})$.
\end{enumerate}
\end{lemma}

\begin{proof}
(b) follows from (a). To show (a), we need to prove that for all $\varphi,\psi\in \Gamma(X,\Omega^{0,0}(L))$, we have
\[
(\overline{\partial_i}\varphi, \psi)=(\varphi, -\partial_i \psi +\pi h_i\overline{v_i}\psi)
\]
but $\langle \overline{\partial_i}\varphi,\psi\rangle-\langle \varphi, -\partial_i \psi +\pi h_i\overline{v_i}\psi\rangle=\overline{\partial_i}\langle \varphi,\psi\rangle$
but $\int_X \overline{\partial_i}\langle \varphi, \psi\rangle=(i/2)^g(\prod K_i)\int_X d(\langle \varphi, \psi\rangle dv_1\wedge d\overline{v_1}\wedge dv_2\wedge\cdots )=0$ by Stokes theorem. 
\end{proof} 
 
\begin{lemma}
Let $\Delta=\overline{\partial}\overline{\delta}+\overline{\delta}\overline{\partial}$ and $\varphi d\overline{v_I}\in \Gamma(X,\Omega^{0,q}(L))$ where $I=(i_1<\cdots <i_q)$ and $\varphi\in \Gamma(X,\Omega^{0,0}(L))$.
Then 
\[
\Delta(\varphi d\overline{v_I})=\sum_{i=1}^g \frac{1}{K_i}\overline{\delta_i}\overline{\partial_i}\varphi d\overline{v_I}+\pi \sum_{j=1}^q \frac{1}{K_{i_j}}h_{i_j}\varphi d\overline{v_I}.
\]
\end{lemma}
 
The proof is just computations. 

\begin{corollary}
$\Delta$ acts on subvector space of ``monomials'' $A_I=\{\varphi_Id\overline{v_I}\}$ (for a fix $I$) of $\Gamma(X,\Omega^{0,q}(L))$. So $\ker\Delta=H^q(L)=\bigoplus_{|I|=q}\ker\Delta|_{A_I}=:\bigoplus_{|I|=q}H_I^q(L)$. 
\end{corollary}

Now, we fix ``nice'' $h_i,k_i$ so that the computations would be easier.
\begin{enumerate}
\item (after permuting and scaling $\{e_1,\cdots,e_g\}$) we may assume $h_i=1$ for $1\leq i\leq r$, $h_i=-1$ for $r+1\leq i\leq r+s$ and $h_i=0$ for $r+s< i\leq g$. 
\item $K_i=\frac{1}{s+1}$ for $i\leq r$ and $K_i=0$ for $i>r$.
\end{enumerate}

Given a multi-index set $I=\{i_1<\cdots <i_q\}$, we write 
$$R_I=|I\cap \{1,\cdots, r\}|$$
and 
$$S_I=|I\cap \{r+1,\cdots, r+s\}|.$$

Below we give the key proposition.

\begin{proposition}
With $K_i$ and $h_i$ as above, for any $pd\overline{v_I}\in \Gamma(X,\Omega^{0,q}(L))$, we have 
$$(\Delta(\varphi_Id\overline{v_I}),\varphi_Id\overline{v_I})\geq \pi((s+1)R_I-S_I)(\varphi d\overline{v_I},\varphi d\overline{v_I}).$$
\end{proposition}
\begin{proof}
Note that
$$(\Delta(\varphi_Id\overline{v_I}),\varphi_Id\overline{v_I})=\sum_{i=1}^g \frac{1}{K_i}(\overline{\delta_i}\overline{\partial_i}\varphi d\overline{v_I},\varphi d\overline{v_I})+\pi(\sum_{j=1}^q \frac{h_{i_j}}{k_{i_j}})(\varphi d\overline{v_I},\varphi d \overline{v_I}).$$
Since $(\overline{\delta_i}\overline{\partial_i}\varphi d\overline{v_I},\varphi d\overline{v_I})\geq 0$ and 
$(\sum_{j=1}^q \frac{h_{i_j}}{k_{i_j}})(\varphi d\overline{v_I},\varphi d \overline{v_I})=(s+1)R_I-S_I$, the result follows.
\end{proof}

\begin{corollary}
$H_I^q(L):=\ker\Delta|_{\{\varphi_I d\overline{v_I}\}}=0$ if $R_I>0$. 
\end{corollary}
\begin{proof}
Let $\varphi d\overline{v_I}\in H^q(L)$, i.e. $\Delta(\varphi d\overline{v_I})=0$. Therefore the above proposition implies that $0\geq \pi((s+1)R_I-S_I)(\varphi d\overline{v_I},\varphi d\overline{v_I})\geq 0$. Since $R_I\geq 1$ and $S_I\leq s$, so $(\varphi d\overline{v_I},\varphi d\overline{v_I})=0$. Therefore, $\varphi d\overline{v_I}=0$.
\end{proof}

\begin{theorem}[Mumford-Kempf]
Let $X=V/\Lambda$ be a complex torus. Suppose $H=c_i(L)$ has $r$ positive and $s$ negative eigenvalues, then $H^q(X,L)=0$ if $q>g-r$ or $q<s$. 
\end{theorem}

\begin{proof}
Let $q>g-r$. Then for any $I$, $|I|=q$ and so we have $R_I>0$. So $H_I^q(L)=0$ but $H^q(L)=\bigoplus_{|I|=q} H_I^q(L)=0$ but $H^q(X,L)=H^q(L)=0$ by Hodge.

For $q<s$ we use Serre duality: If $X$ is a smooth compact complex manifold and $L$ is a holomorphic line (or vector) bundle over $X$ and $K$ is a canonical bundle (which is a line bundle), then there is a canonical line bundle isomorphism $H^q(X,L)\isom H^{g-q}(X,K\bigotimes L^v)^\times$. 
In our situation, $X$ is a complex torus and $K=\Omega^g$ which is free of rank $\binom{g}{g}=1\isom \mathcal{O}_X$ and $L^v\isom L^{-1}$. Therefore, by Serre duality, we have 
$$H^q(X,L)\isom H^{g-q}(X,L^{-1})^\times.$$
Note that $c_1(L^{-1})=-c_1(L)$ has $s$ positive eigenvalues. Since $q<s$, $g-q>g-s$ and therefore $H^q(X,L)=0$.  
\end{proof}

Remark: Coherent duality of Grothendieck gives a more general version of Serre duality.

\subsection{$q=s,s+1,...,g-r$}

\begin{theorem}
For $s\leq q \leq g-r$, we have 
$$h^q(X,L)=\dim H^q(X,L)=\binom{g-r-s}{q-s} h^s(X,L).$$
Indeed, 
$$H^q(X,L)\isom \bigoplus_{i=1}^{\binom{g-r-s}{q-s}}H^s(X,L)$$
\end{theorem}
\begin{proof}
First, we claim that $H^q(L)=\bigoplus_{|I|=q, R_I=0, S_I=s} H_I^q(L)$. 
Really, we only need to show that $S_I=s$. To show this, we may assume that $R_i=0$ from above. We know that $S_I\leq s$. Note that 
$$\Delta(\varphi d\overline{v_I})=\psi d\overline{v_I}$$
where $\psi=\sum_{j=1}^g \frac{1}{k_j} \overline{\delta_j}\overline{\partial_j}\varphi-(\pi \sum_{j=1}^q \frac{h_{i_j}}{K_{i_j}})\varphi$. We can write $\pi S_I=(\pi \sum_{j=1}^q \frac{h_{i_j}}{K_{i_j}})\varphi$. If we pick any $J=I\cap \{r+1,\cdots,r+s\}$ then $R_J=0$ and $S_J=S_I$, and hence $\Delta(\varphi d\overline{v_J})=\Delta(\varphi d\overline{v_I})$. 
Moreover, the map $H_I^q(L)\to H_J^{S_I}(L)$ given by $\varphi d\overline{v_I}\mapsto \varphi d\overline{v_J}$ is an isomorphism! Since $S_I<s$, by the vanishing theorem, we have $H_J^{S_I}(L)=0$ and so $S_I=s$. 

Next, we claim that $H^q(L)\isom H^s(L)^{\binom{g-r-s}{q-s}}$. Let $J=(r+1<\cdots <r+s)$. Note then that $R_J=0$ and $S_J=s$. By the first claim, we have $H^q(L)=\bigoplus_{|I|=q, R_I=0, S_I=s} H_I^q(L)$ which is isomorphic to $H_J^s(L)$. By picking $q$ numbers from $\{1,\cdots, g\}$ such that all of $\{r+1,\cdots,r+s\}$ are being picked but not picking any of $\{1,\cdots, r\}$, there are exactly $\binom{g-r-s}{q-s}$ ways and hence the theorem.
\end{proof}

\subsection{$q=s$}
In the last section we have reduced every to the case $q=s$ and so we can focus on this for now. Note that if $s=0$, then this is a positive semi-definite case and therefore we are done! 

If $s>0$, it then suffices to find another complex torus $X',L'$ such that $L'$ is positive semi-definite on $X'$ and $H^s(X,L)\isom H^0(X',L')$. Here we introduce \emph{Wirtinger's trick}. The idea is o change the ``complex structure''. Recall that the connected component $\Lambda(L)_0$ of $\Lambda(L)$ is a radical of $H$. Also we have fixed a basis $\{e_1,\cdots, e_g\}$ two sections earlier such that $H$ is diagonal matrix with eigenvalues $1,-1,0$. Note that $\Lambda(L)_0=\text{span}\{e_{r+s+1},\cdots, e_g\}$.
Define $V_+=\text{span}\{e_1,\cdots,e_r\}$ and $V_-=\text{span}\{e_{r+1},\cdots,e_{r+s}\}$. Then as $\CC$ vector spaces, we have
$$\CC^g\isom V = V_+\bigoplus V_- \bigoplus V_0.$$


Let $\RR^{2g}\isom W$ be the underlying $\RR$-vector space. Let $j$ be the complex structure on $w$ corresponding to $V$. In particular, $j:W\to W$ given by $v\mapsto iv$ is $\RR$-linear. We therefore have 
$$\RR^g\isom W = W_+\bigoplus W_- \bigoplus W_0.$$
induced by above. We define $j'(w):=j(w)$ if $w\in W_+\bigoplus W_0$ and $j'(w):=-j(w)$ if $w\in W_-$. Now let $V'=(W,j')$ be a complex vector space and note that $\Lambda\subset V'$ is obviously a lattice. Let 
$$X':=V'/\Lambda$$
which is a new complex torus. Recall that for $H=c_i(L)$, we have 
$$H(v,w)=E(j(v),w)+i E(v,w)$$
where $E=Im(H)$ and $L=L(H,X)$. We now define 
$$H'(v,w):=E(j'(v),w)+iE(v,w).$$
One can check that $Im(H'(\Lambda,\Lambda))\subset \ZZ$. Since $Im(H)=Im(H')$, $X$ works for $H'$ as well. We can then define
$$L':=L(H',X).$$

\begin{theorem}
$H^s(X,L)\isomto H^0(X',L')$ and $L'$ positive semi-definite. 
\end{theorem}

\begin{proof}
The idea is to let $f:W\to \CC^\times$ which is given by $w\mapsto e^{\pi H'(w_-,w_+)}$ where $w=w_+ + w_- +w_0$. Then one can consider the map $\{\varphi d\overline{v_J}:\ \varphi\in \Gamma(X,\Omega^{0,0}(L))\}\to \Gamma(X',\Omega^{0,0}(L'))$ given by mapping $\varphi d\overline{v_J}$ which is a $(0,s)$-form on $V$ to $\varphi f$ which is a $C\infty$ function on $V'$. Then this induces the above isomorphism.
\end{proof}

\begin{definition}
For line bundles $L_1,\cdots, L_g\in \Pic(X)$ where $X$ is a $g$ dimensional abelian variety, the \textbf{intersection number} of $L_1,\cdots, L_g\in \Pic(X)$ is given by
$$(L_1,\cdots,L_g):=\int_X c_1(L)\bigwedge c_1(L_2)\bigwedge \cdots \bigwedge c_1(L_g).$$
\end{definition}

Remarks: 
\begin{enumerate}
\item $c_1(L)\in H^2(X,\ZZ)\injects H^2(X,\CC)\isom H^{2,0}\bigoplus H^{1,1}\bigoplus H^{0,2}$. In fact, we have $c_1(L)\in H^2(X,\ZZ)\cap H^{1,1}$. 
\item Note that $c_1(L)\bigwedge c_1(L_2)\bigwedge \cdots \bigwedge c_1(L_g)$ is a $(g,g)$-form which can be viewed as a volume form, and so the integral makes sense.
\item $c_1(L)\in H^2(X,\ZZ)$ is Poincar\'{e} dual to $\{D\}\in H_{2g-2}(X,\ZZ)$, for any divisor $D$ associated to $L$. (Also $\{D\}\in H_{2g-2}(X,\ZZ)$ makes sense because it is ``triangulizable''.) So $(L_1,\cdots, L_g)$ can be thought of as \emph{intersection} (in the topological sense) of associated divisor.
\end{enumerate}

\begin{definition}
For $L\in \Pic(X)$, the \textbf{self intersection} of $L$ is defined by 
$$(L^g):=(L,\cdots, L)=\int_X \bigwedge^g c_1(L).$$
\end{definition}

\begin{theorem}
If $L\in \Pic(X)$ is type $(d_1,\cdots,d_g)$ where $X$ is a complex torus, and $H$ has $s$ negative eigenvalues, then
$$\frac{1}{g!}(L^g)=(-1)^s Pf(L)$$
where $Pf(L)=d_1\cdots d_g$. 
\end{theorem}

\begin{corollary}
$$\chi(X,L)=\frac{1}{g!}(L^g).$$
\end{corollary}

\begin{lemma}
If $L$ is type $(d_1,\cdots,d_g)$. Let $\{\lambda_1,\cdots,\lambda_g; \mu_1,\cdots,\mu_g\}$ be a symplectic basis for $E=Im(H)$ with corresponding coordinate functions given by $\{x_1,\cdots, x_g; y_1,\cdots, y_g\}$, then 
$$c_1(L)=-\sum_{j=1}^g d_j dx_j\wedge dy_j$$
as a $(1,1)$-form.
\end{lemma}

The proof follows from the definitions.

Remark: If $L$ is degenerate, then there exists $d_i=0$ and so $\bigwedge^g c_1(L)=0$ and $Pf(L)=0$. Therefore, in this case $\frac{1}{g!}(L^g)=0$. So we may assume that $L$ is non-degenerate. In the non-degenerate case, note that $s=g-r$ which is the index of $L$ and $\chi(X,L)=h^s(X,L)$. 

\begin{lemma}
$\int_X \bigwedge_{j=1}^g dx_j\wedge dy_j=(-1)^{g+s}.$
\end{lemma}

Remark: This lemma can be thought of as the orientation associated to the symplectic basis is $\pm 1$ depending on $s$.

\begin{proof}
We have seen that $\{\mu_1,\cdots, \mu_g\}$ is a $\CC$-vector space basis for $V$ with $\{v_1,\cdots,g_g\}$ the corresponding coordinate functions. Then $(1/2)^g \bigwedge_{j=1}^g dv_j\wedge d\overline{v_j}$ gives the natural \emph{positive} orientation. Let $\Pi$ be the period matrix with respect to $\{\lambda_1,\cdots, \lambda_g; \mu_1,\cdots, \mu_g\}$. Note that $\{\lambda_1,\cdots, \lambda_g; \mu_1,\cdots, \mu_g\}$ is a $\ZZ$-basis for $\Lambda$ and $\{\mu_1,\cdots,\mu_g\}$ is a $\CC$-basis for $V$. Write $\Pi=(Z|I)$. Then we have 
$$(i/2)^g \bigwedge_{j=1}^g dv_j\wedge d\overline{v_j} = (-1)^g \det(Im(Z))\bigwedge_{j=1}^g dx_j\wedge dy_j.$$
It then suffices to show that $(-1)^s\det(Im(Z))>0$. 

Let $y\in M_{g\times g}(\CC)$ be the matrix of $H$ with respect to the $\CC$-basis $\{\mu_1,\cdots, \mu_g\}$ of $V\isom \CC^g$, then the matrix of $H$ with respect to $\{\lambda_1,\cdots, \lambda_g; \mu_1,\cdots, \mu_g\}$ of $V\isom \RR^{2g}$ is 
$$\Pi^t Y \overline{\Pi}=\binom{Z^t}{I}Y(\overline{Z}| I)=
\left(
\begin{array}{cc}
Z^t Y \overline{Z} & Z^t Y\\
Y\overline{Z} & Y\\
\end{array}
\right).$$ 
Therefore, the matrix of $E$ with respect to $\{\lambda_1,\cdots, \lambda_g; \mu_1,\cdots, \mu_g\}$ is just $Im(H)=Im \left(
\begin{array}{cc}
Z^t Y \overline{Z} & Z^t Y\\
Y\overline{Z} & Y\\
\end{array}
\right)$. On the other hand, the matrix is also given by $\left(
\begin{array}{cc}
0 & D\\
-D & 0\\
\end{array}
\right)$. This implies that $Y$ is real, and $D=(Im Z^t)Y$. Therefore, the result follows.
\end{proof}

Note that $\bigwedge_{i=1}^g c_1(L)=(-1)^g (g!)(d_1\cdots d_g)\bigwedge_{i=1}^g dx_j\wedge dy_j$. Integrating this expression gives the following corollary:

\begin{corollary}
$(L^g)=\int_X \bigwedge_{i=1}^g c_1(L)=(-1)^s g! Pf(L)$.
\end{corollary}

The proof is done with Algebraic/Geometric Riemann Roch.

\begin{corollary}
Let $f:X'\to X$ be a surjective homomorphism of complex tori. For any $L\in \Pic(X)$, we have 
$$\chi(X', f^* L)=(\deg f)(\chi(X,L)).$$
\end{corollary}

\begin{proof}
If $f$ is not an isogeny, then we have seen that $f^* L$ is degenerate and the statement follows trivially. If $f$ is an isogeny, then $\deg (f)=[\Lambda:f_{Int}(\Lambda')]$ and 
$$\int_{X'} \bigwedge_{i=1}^g c_1(f^* L)=(\deg f)\int_X \Lambda^g c_1(L).$$
Note that equality sign follows from the surjectivity assumption (so that the ``change of variable'' is okay). Now the rest is done by Geometric Riemann-Roch.
\end{proof}