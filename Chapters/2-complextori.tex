% !TEX root = 7390.tex




\section{Complex Tori}\label{Chapters/2-complextori}

\subsection{Some GAGA principles}
Algebraic varieties vs. (complex) analytic spaces. 

If $X$ is an algebraic variety over $\CC$, we can associate $X^{an}$ which is a complex analytic space to it, by passing to a complex analytification; if $X$ is locally described by some set of equations in affine space, we can pass that open set the zero locus as a subset of $\CC^n$ (with its usual topology) and glue. 
Here are some facts:

\begin{itemize}
\item 
This construction is always functorial: an algebraic map $X\to Y$ can always be lifted to a holomorphic map $X^{an}\to Y^{an}$. 
\item $X$ is proper/complete if and only if $X^{an}$ is compact. 
\item $X$ is smooth/connected if and only if $X{an}$ is smooth/connected. 


\item A complex analytic space $\mathfrak{X}$ is called \textbf{algebraic/algebraizable} if there exists a variety $X/\CC$ such that $\mathfrak{X}\isom X^{an}$. (Last time we explicitly showed that $\CC/\Lambda$ is algebraic via the Weierstrass $\wp$-functions.)
\end{itemize}

\subsection{Vector bundles and associated locally free sheaves}

If $L$ is a vector bundle on $X$, then it passes to an analytic vector bundle $L^{an}$ on $X^{an}$. 
This is functorial in the sense that if $f:F\to G$ is a morphism of vector bundles it passes to $f^{an}:F^{an}\to G^{an}$. It is \emph{not} true that all holomorphic vector bundles over $X^{an}$ are algebraizable! But we have:

\begin{theorem}[Serre]
\noindent 
\begin{enumerate}
\item Suppose $X$ is a proper (complete) algebraic variety over $\CC$. If there exists a holomorphic coherent sheaf $\mathcal{F}\to X^{an}$, then there exists unique algebraic coherent sheaf $F$ over $X$ such that $F^{an}=\mathcal{F}$.
\item If there exists $\mathfrak{F}:\mathcal{F}\to \mathcal{G}$ homomorphism of holomorphic coherent sheaves on $X^{an}$, then there exists unique $f:F\to G$ such that $f^{an}=\mathfrak{f}$.
\end{enumerate}
\end{theorem}

Define $H^i(X,L)$ as the $i$-th cohomology group with values in the locally free sheaf $L$. 

\begin{theorem}[Serre]
Let $X$ be a complete algebraic variety over $\CC$ and $F$ a coherent sheaf on $X$. Then the \emph{natural maps} 
$$H^i(X,F)\to H^i(X^{an},F^{an})$$
are isomorphisms of $\CC$-vector spaces.
\end{theorem}

\subsection{Complex tori}
Let $V$ be a vector space over $\CC$, and $\Lambda\subset V$ a lattice (full rank discrete subgroup). We have $\Lambda$ act naturally on $V$ by addition; and then the quotient $X=V/\Lambda$ is a complex torus. 

Some facts about a complex torus:
\begin{itemize}
\item it is a complex manifold.
\item it inherits the structure of a complex Lie group over $\CC$.
\item it is compact (because $\Lambda$ is a maximal rank lattice).
\item it is an abelian complex Lie group.
\item meromorphic functions on $X$ correspond to meromorphic $\Lambda$-periodic functions on $V$. 
\end{itemize}

Loosely speaking, 
an (complex analytic) \emph{abelian variety} is a complex torus with ``sufficiently many'' (enough to give a closed embedding to a projective space) meromorphic functions. We will see that this is exactly what makes $X$ algebraizable and thus an algebraic abelian variety.

\subsection{Compactness implies abelian}

\begin{theorem}
Any connected \emph{compact} complex Lie group $X$ is a complex torus.
\end{theorem}

\begin{proof}
First, $X$ is abelian and so the commutator map $\Phi(x,y)=xyx^{-1}y^{-1}$ is continuous. Let $U$ be any neighbourhood of the identity element $1$, for $x\in X$, define open neighbourhoods $V_x,\tilde{V_x}$ such that $x\in V_x$, $1\in \tilde{V_x}$ and $\Phi(V_x,\tilde{V_x})\subset U$. (This can be done since $\Phi(x,1)=1$ and $\Phi$ is continuous.) 

So we have
$X=\cup_{x\in X} V_x$
and by compactness, there exist $x_1,\cdots, x_r\in X$ such that
$$X=\bigcup_{x\in \{x_1,\cdots,x_r\}} V_x.$$

Let $W=\cap_{x_1,\cdots,x_r} \tilde{V_x}$, which is a non-empty open neighbourhood of $1$. So $\Phi(X,W)\subset U$. Since $U$ is arbitrary, we have $\Phi(X,W)=1$. 

Since holomorphic functions on a compact set $X$ which is bounded must be constant, we have
$\Phi(1,y)=1$ for all $y\in W$. Since $W$ is open and non-empty, by connectivity, 
$$\Phi(x,y)=1$$
for all $x,y\in X$. 

Then, if $\pi:V\to X$ is a universal cover, $V$ inherits the structure of a simply connected complex Lie group and thus must be $\CC^g$. Moreover $\pi$ is homomorphic with discrete kernel, and by compactness of $X$ the kernel must be full rank.
\end{proof}

Another proof can be found in B. Conrad's \href{http://math.stanford.edu/~conrad/vigregroup/vigre04/abvaran.pdf}{notes}.



Remarks: Once we have $X=V/\Lambda$, we see that $V$ is a universal cover of $X$. Moreover, $\Lambda=\pi(X,0)$, and since this is already abelian it is isomorphic to $\isom H_1(X,\ZZ)$. 
Since $X$ is locally isomorphic to $V$, we can view $V$ as the tangent space at $0$, $T_0X$; then the covering map $\pi:V=T_0X\to X$ is actually the exponential map.


\subsection{Period matrix}

Given $X=V/\Lambda$, we can associate $\Pi$ a $g\times 2g$ complex matrix: fix $\{e_1,\cdots, e_g\}$ a $\CC$-basis for $V$ and $\{\lambda_1,\cdots,\lambda_{2g}\}$ a $\ZZ$-generator set for $\Lambda$. Define $\lambda_{ji}$ such that
$$\lambda_j=\sum \lambda_{ji} e_i.$$

Then the \textbf{period matrix} of $X$ is given by 
$$\Pi:=
\left(
\begin{array}{ccc}
\lambda_{1,1} & \cdots & \lambda_{1,2g}\\
\vdots & \ddots & \vdots\\
\lambda_{g,1} & \cdots & \lambda_{g,2g}\\
\end{array}
\right).$$

Clearly, $\Pi$ determines $X$ but it depends on the choices. 

Question: Given $\Pi\in M_{g\times 2g}(\CC)$, is there a complex torus such that $\Pi$ is the period matrix of $X$? 

\begin{theorem}
Let $P=\left(
\begin{array}{c}
\Pi\\
\overline{\Pi}\\
\end{array}
\right)_{2g\times 2g}$, where $\overline{\Pi}$ denote the complex conjugate matrix of $\Pi$. Then $\Pi$ is the period matrix for some $\CC^g/\Lambda$ if and only if $P$ is nonsingular. 
\end{theorem}

\begin{proof}
$\Pi$ is a period matrix if and only if the columns of $\Pi$ are $\RR$-linearly independent. 
\end{proof}

\subsection{Holomorphic maps, homomorphism and isogenies}
Suppose $X=V/\Lambda$ and $X'=V'/\Lambda'$ with dimensions $g$ and $g'$ respectively. We want to study holomorphic maps $f:X\to X'$. There are two special examples:

\begin{enumerate}
\item homomorphisms (holomorphic and respect group structure); and
\item translations (maps $X\to X$ by $x\mapsto x+x_0$ for some $x_0\in X$).
\end{enumerate}

The surprising thing is that that's \emph{all}! 

\begin{theorem}
Suppose $h:X\to X'$ is a holomorphic map between complex tori. Then 
\begin{enumerate}
\item there exists a unique homomorphism $f:X\to X'$ such that $h=t_{h(0)}\circ f$. That is,
$$h(x)=f(x)+h(0)$$
for all $x$.
\item There exists a unique $\CC$-linear map $F:V\to V'$ with $F(\Lambda)\subset \Lambda'$ inducing $f$. 
\end{enumerate}
\end{theorem}


\begin{proof}
Let $f:=t_{-h(0)}\circ h$. Then we can lift $f\circ \pi: V\to X$ to $F:V\to V'$ where $V'$ is the universal cover of $X'$. Then $F$ is holomorphic and satisfies $F(0)=0$. $F$ is a $\CC$-linear map: fix $\lambda\in \Lambda$, by construction,
$$F(v+\lambda)-F(v)\in \Lambda'$$
and so by continuity it's constant. Therefore,
$$F(v+\lambda)= F(v)+F(\lambda)$$
for all $v\in V, \lambda\in \Lambda$. We skip the remaining details.
\end{proof}


\subsection{Hom-sets}
Let $\Hom(X,X')$ be the set of all homomorphisms $f:X\to X'$. It is an abelian group. If $X=X'$, then we can define $\End(X):=\Hom(X,X')$. In this case, $\End(X)$ is actually a ring, where multiplication is given by composing endomorphisms. The above theorem gives us the following corollary:

\begin{corollary}
We have injective homomorphisms:
\begin{enumerate}
\item $\rho_{an}:\Hom(X,X')\to \Hom_\CC (V,V')$ given by $f\mapsto F$; and
\item $\rho_{int}: \Hom(X,X')\to \Hom_\ZZ(\Lambda,\Lambda')$ given by $f\mapsto F|_\Lambda$.
\end{enumerate}
\end{corollary}

Note that both of these homomorphisms respect endomorphism ring structures if $X=X'$: $\rho_*(f'\circ f)=\rho_*(f')\circ \rho_*(f)$, where $*=an, int$.

\begin{theorem}
$\Hom(X,X')\isom \ZZ^m$ for some $m\leq 4gg'$.
\end{theorem}

\begin{proof}
Use the second isomorphism in the corollary, since $\Lambda\isom \ZZ^{2g}$ and $\Lambda'\isom \ZZ^{2g'}$, so  $\Hom_\ZZ(\Lambda,\Lambda')\isom \ZZ^{4gg'}$ and $\Hom(X,X')$ embeds in this.
\end{proof}

How do these relate to period matrices?
Let $\Pi$ and $\Pi'$ be the period matrix for $X$ and $X'$ respectively. If we have $f:X\to X'$, then by picking bases we get that $\rho_{an}(f):V\to V'$ is given by some $A\in M_{g'\times g}(\CC)$ and $\rho_{int}(f):\Lambda\to \Lambda'$ given by some $R\in M_{2g'\times 2g}(\ZZ)$. Then the condition $F(\Lambda)\subset \Lambda'$ means $A\Pi=\Pi'R$. (The converse is also true: given four matrices with this property, then they correspond to a morphism between complex tori.) 

What if $X=X'$? In this case, we can get 
$$\left(
\begin{array}{cc}
A & 0\\
0 & \overline{A}\\
\end{array}\right)
\left(
\begin{array}{c}
\Pi\\
\overline{\Pi}\\
\end{array}
\right)=\left(
\begin{array}{c}
\Pi\\
\overline{\Pi}\\
\end{array}
\right)R$$

and thus 
$\rho_{int}\otimes 1\isom \rho_{an}\oplus \bar{\rho_{an}}$ in $\End(X)\otimes_\ZZ \CC$. 

\subsection{Kernels and Images}
\begin{lemma}
Given a homomorphism $f:X\to X'$. 
\begin{enumerate}
\item $Im(f)$ is a complex subtorus of $X'$.
\item $\ker(f)$ is a closed subgroup of $X$ with finitely many component. The connected component of $1=id$ is a complex torus. 
\end{enumerate}
\end{lemma}

The proof is fairly easy; for part (b) we're claiming that we have an extension 
$$1\to X_0 \to G \to \Gamma \to 1$$
with $X_0$ a complex torus and $\Gamma$ a finite abelian group. It is a good exercise to describe $\Gamma$ as a direct sum of cyclic groups in terms of $\Pi,\Pi',A,R$ (need to compute a Smith normal form somewhere).


\subsection{Isogenies} 

A homomorphism
$f:X\to X'$ is called an \textbf{isogeny} if $f$ is surjective with finite kernel. Equivalently, $f$ is surjective and $\dim(X)=\dim(X')$. 

\begin{example}[Essential example]
Suppose $X=V/\Lambda$ is a complex torus and $\Gamma\subset X$ is a finite subgroup. Then $X/\Gamma=V/\pi^{-1}(\Gamma)$ is a complex torus and $X\to X/\Gamma$ is an isogeny. 
\end{example}

In fact, that's all! It is an easy exercise to show that all isogenies $X\to X/\Gamma$ over $\CC$ are of this form. We also have the following easy lemma:

\begin{lemma}[Stein factorization]
Any surjection $f:X\to X'$ of complex tori factors as a surjection $X\to X/(\ker f)_0$ (a quotient of $X$ by a complex subtorus) and an isogeny $X/(\ker f)_0 \to X'$.
\end{lemma}

Remark: Stein factorization is a special case of a very general result (in complex theory by Stein and others, and in general in EGA III): any proper $f:X\to S$ factors as a map $X\to S'$ proper with connected fibers and $S'\to S$ finite. 
 
For $f\in \Hom(X,X')$, we define $\deg(f)$ to be $|\ker f|$ if this is finite, and $0$ if otherwise. It is easy to check that $\deg(f)=[\Lambda':\rho_{int}(f)\Lambda]$. (Remark: If $X=X'$ then this index is $\deg(\rho_{int}(f))$; note that this determinant is $\geq 0$ since $\rho_{int}\otimes 1 = \rho_{an}\oplus \bar{\rho}_{an}$, and is $0$ if and only if the kernel is infinite.)

\begin{lemma}
Supppose $f:X\to X'$ and $f':X'\to X''$ are isogenies, then $f'\circ f$ is also an isogeny. 
\end{lemma}

\begin{proof}
$\deg(f'\circ f)=\deg(f)\cdot \deg(f')$.
\end{proof}

A very important example is given by the ``multiplication-by-$n$'' map: Let $n\in \ZZ^+$, define 
$n_X:X\to X$ by $x\mapsto nx$. Denote $X[n]:=\ker(n_X)$ the set of $n$-torsions in $A$. Then we have
$$X[n]\isom \frac{\frac{1}{n} \Lambda}{\Lambda}\isom \frac{\Lambda}{n\Lambda}\isom (\ZZ/n)^{2g}.$$
Therefore, $n_X$ has degree $n^{2g}$, and so it is an isogeny.

\begin{corollary}
Complex tori are divisible groups.
\end{corollary}


\begin{example}[Tate module]
Let $\ell$ be a prime number. Define multiplication by $\ell$ maps $X[\ell^{n+1}]\to X[\ell^n]$. Then the \emph{Tate module} is given by 
$$T_\ell(X)=\varprojlim X[\ell^n].$$
In the case where $\Lambda$ is finitely generated, $T_\ell(X)$ is actually isomorphic to $\Lambda\otimes_\ZZ \ZZ_\ell$ and this is a subset of $\Lambda$. 
Note that the definition of $T_\ell(X)$ makes sense over any fields (even if we don't have $\Lambda$ when we are not over $\CC$). Here in our setting it's easy to see that a morphism $X\to X'$ is determined by the induced map $T_\ell(X)\to T_\ell(X')$. Over general fields this is much, much harder! It's the \emph{Tate conjecture} which says that we have 
$$\Hom_{\Gal}(T_\ell(X),T_\ell(X'))\isom \Hom(X,X').$$
This conjecture was only proven over number fields by Faltings as an essential part of his proof of the Mordell conjecture. 
\end{example}



\subsection{Importance of isogenies}

They are ``almost isomorphisms''. Namely, we have:

\begin{theorem}
Let $f:X\to X'$ be an isogeny and $n$ be the exponent of $\ker(f)$. (That is, $nx=0$ for all $x\in \ker(f)$.) Then there exists an isogeny $g:X'\to X$ such that 
$$f\circ g=n_X,\ g\circ f=n_X.$$
Moreover, such a $g$ is unique (up to isomorphism?).
\end{theorem}

\begin{proof}[sketch]
Since $n$ is the exponent of $\ker f$, we have $\ker(f)\subset \ker(n_X)=X[n]$. Then there exists a unique $g:X'\to X$ with $g\circ f = n_X$, defined by $g(x'):=nx$ for some (all) $x$ where $f(x)=x'$. Then use the fact that $\deg(g)\deg(f)=\deg(n_X)$ and that $\deg(f),\deg(n_X)\not= 0$, so $\deg(g)\not= 0$ to get that $g$ is an isogeny; Then we just need to check that $g\circ f = n_X'$. 
\end{proof}

Define $\End_\QQ(X):=\End(X)\otimes \QQ$ and $\Hom_\QQ(X,X'):=\Hom(X,X')\otimes \QQ$. Then the degree function extends to these via
$$\deg(rg):=r^{2g}\cdot \deg(f).$$

\begin{corollary}
\noindent
\begin{enumerate}
\item Isogeny is an equivalence relation.
\item $f\in \End(X)$ is an isogeny if and only if it is invertible in $\End_\QQ(X)$. 
\end{enumerate}
\end{corollary}



\subsection{Cohomology} 

We have a lot of cohomology theories (Betti, de Rham, Dolbeault, Hodge decomposition, \dots).

Betti cohomology is just singular cohomology of $X(\CC)$; if $X=V/\Lambda$, then we have the following facts:

\begin{itemize}
\item $\Lambda=\pi_1(X_0)\isom H_1(X,\ZZ)$.
\item By the universal coefficient theorem, we have $H^1(X,\ZZ)=\Hom(\Lambda,\ZZ)$.
\item If $n\geq 1$, we have a map $\wedge_{i=1}^n H^1(X,\ZZ)\to H^n(X,\ZZ)$ induced by cup product, and this is an isomorphism (follows from Kunneth formula). 
\item Let $Alt^n(\Lambda,\ZZ):=\bigwedge_{i=1}^n \Hom(\Lambda,\ZZ)$ be all the $\ZZ$-valued alternating $n$-forms. Then we have $H^n\isom Alt^n(\Lambda,\ZZ)$. This gives a very explicit way of thinking about cohomology.
\item $H_n(X,\ZZ)$ and $H^n(X,\ZZ)$ are free $\ZZ$-modules of rank $\binom{2g}{n}$.
\item If we set $H^n(X,\CC):= H^n(X,\ZZ)\otimes \CC$, then we have
$$H^n(X,\CC)\isom Alt_\RR^n(V,\CC)=\bigwedge_{i=1}^n\Hom_\RR(\Lambda,\CC)\isom \bigwedge_{i=1}^n H^1(X,\CC),$$ 
and the de Rham theorem tells us 
$H^n(X,\CC)\isom H_{DR}(X)$ where $H_{DR}(X)$ can be explicitly described as a complex vector space of invariant $n$-forms with basis $dx_{i_1}\wedge \cdots \wedge dx_{i_n}$ with $i_1<\cdots <i_n$. 
\end{itemize}

Now, we use the $\CC$-structure (really everthing is true for Kahler manifolds, but proofs and constructions are much more elementary for complex tori). Here we have a very nice decomposition

$$H^n(X,\CC)\isom \bigoplus_{p+q=n} H^q(\Omega_X^p)$$ 

Here, $H^{p,q}(X):=H^q(\Omega_X^p)$ is isomorphic to the Dolbeault cohomology $H^{p,q}(X)$. In general, $H^q(\Omega_X^p)$ can be explicitly described as $\bigwedge^p\Omega \otimes \bigwedge^q\overline{\Omega}$ for $\Omega=\Hom_\CC(V,\CC)$ and $\overline{\Omega}=\Hom_{\overline{\CC}}(V,\CC)$.
Also set $\Omega_X^p:=(\bigwedge^p \Omega)\otimes \mathcal{O}_X$. Right now we are not saying anything about what these vector spaces $H^{p,q}(X)$ are, but we will do that later on. We may also need ot return to this theory to prove for example vanishing results later; We will either do that, or just omit those proofs.

\subsection{Sheaves on a topological space $X$}

Let $\mathcal{O}(X)$ be the category in which objects are open subsets of $X$ and morphisms are inclusions $V\to U$ for $V\subset U$. (It turns out that you can generalize this and allow more general things for morphisms than just inclusions; this is how you get etale cohomology and other things.) Let $\mathcal{C}$ be any other category (for instance, Sets, Abelian groups and $R$-modules); A \textbf{presheaf} is a contravariant functor 
$$F:\mathcal{O}(X)\to \mathcal{C}.$$
(This means for each inclusion map $i:V\subset U$, we have the restriction map given by $rest_{V,U}: F(U)\to F(V)$, and this assignment is functorial.) A \textbf{sheaf} is a presheaf $F$ with some ``locality'' and ``gluing'' properties. Assume $\mathcal{C}$ has products. Then we require for any open cover $\{U_i\}$ of $U\in \mathcal{O}(X)$, we have an exact sequence
$$0\to F(U) \overset{rest}{\to} \prod_i F(U_i)\rightrightarrows \prod_{i,j} F(U_i\cap U_j)$$
(where the two parallel maps are ``restriction to the first index'' and ``restriction to the second index'' respectively.)

\begin{example}
Let $X=\CC$. Then we have the sheaf of holomorphic functions: $F(U)$ is the set of all holomorphic functions $U\to \CC$, and restriction maps are restriction of functions!
\end{example}

Morphisms of sheaves are natural transformations; this gives us the category of sheaves over $X$, $Shf_X$, in which the objects are sheaves on $X$ and morphisms are these.

Next, a \textbf{ringed space} is a topological space $X$ together with a sheaf of rings $\mathcal{O}_X$; we call $\mathcal{O}_X$ the \textbf{structure sheaf} (usually some sheaf of holomorphic functions in this class). 
Define a \textbf{locally ringed space} to be one such that all of the stalks $\mathcal{O}_{X,x}=\varinjlim_{U\ni x} F(U)$ are local rings. We may want to consider sheaves of $\mathcal{O}_X$-modules, that is, each $F(U)$ is a $\mathcal{O}_X(U)$-module respecting restriction maps. 

\subsection{Abelian categories and cohomology}
(Grothendieck's Tohoku paper) There are $4$ main examples of abelian categories to keep in mind:

\begin{enumerate}
\item The category of abelian groups (with homomorphisms).
\item The category of $R$-modules for $R$ a commutative ring (with homomorphisms).
\item The category of $G$-modules for $G$ a group (with $G$-equivariant homomorphisms: $\phi(g\cdot m)=g\cdot \phi(m)$).
\item The category of $\mathcal{O}_X$-modules for $(X,\mathcal{O}_X)$ a ringed space (with morphisms of sheaves the morphisms). 
\end{enumerate}

In general there's an abstract definition of an abelian category; we are not going to say it precisely but roughly it is a category $\mathcal{A}$ in which addition of morphisms, a zero object, kernels and cokernels make sense.

Suppose $F:\mathcal{A}\to \mathcal{B}$ is a covariant functor between abelian categories. If we start with an short exact sequence 
$$0\to A \to B \to C \to 0$$
in $\mathcal{A}$, we can hit it with the functor $F$ and get maps in $\mathcal{B}$, but no guarantee for exactness. We say that $F$ is \textbf{left exact} if for every such short exact sequence we do have
$$0\to F(A)\to F(B)\to F(C)$$ 
exact. 

\begin{example}
Let $\mathcal{A}$ be the category of $R$-module, $\mathcal{B}$ be the category of abelian groups and $D$ be a fixed object. Then $F(-)=\Hom_R(D,-)$ is a left exact covariant functor.
\end{example}

Cohomology lets us study the failure of exactness of 
$$0\to F(A)\to F(B)\to F(C),$$
i.e. the failure of surjectivity of $F(B)\to F(C)$. We want to continue the above exact sequence to the right and write a long exact sequence. In general, there are many ways to do this. But if $\mathcal{A}$ has ``enough injectives'' (a statement that holds for the categories we care about), then there is a ``canonical'' and ``minimal'' (in the sense that there is a universal property) way to do this. There exist unique functors $R^iF:\mathcal{A}\to \mathcal{B}$ for $i\geq 0$ with $R^0F=F$ that give us the following long exact sequence
$$0\to F(A)\to F(B) \to F(C) \overset{c_1}{\to} R^1F(A)\to R^1F(B) \to R^1F(C)\overset{c_2}{\to} R^2F(A) \to \cdots$$
plus satisfying some universal properties (an ``effaceable $\delta$-functor''). These functors are called the \emph{right derived functors}. This is the covariant version; there's a similar one for contravariant functors. In general, it is hard to compute these $R^iF$. However, in the examples that we are interested in, there are easier ways of computing them.

\begin{example}
Let $F(-):=\Hom_R(-,D)$ and $G(-):=\Hom_R(D,-)$ where $D$ is a fixed $R$-module. Both $F$ and $G$ are left-exact functors from $R$-modules to abelian groups, one contravariant and one covariant. More precisely, for any short exact sequence of $R$-modules
$$0\to L \to M \to N \to 0,$$
we have the following exact sequences
$$0\to \Hom_R(N,D)\to \Hom_R(M,D)\to \Hom_R(L,D)$$
and 
$$0\to \Hom_R(D,L)\to \Hom_R(D,M)\to \Hom_R(D,N).$$

In both cases, the derived functors are just the Ext groups $\Ext_R^i(-,D)$ and $\Ext_R^i(D,-)$
which give us the long exact sequences
$$0\to \Hom_R(N,D)\to \Hom_R(M,D)\to \Hom_R(L,D)\to \Ext_R^1(N,D)\to \Ext_R^1(M,D)\to \cdots$$
and 
$$0\to \Hom_R(D,L)\to \Hom_R(D,M)\to \Hom_R(D,N)\to \Ext_R^1(D,L)\to \Ext_R^1(D,M)\to \cdots$$

So $R^i\Hom_R(-,D)=\Ext_R^i(-,D)$ and $R^i\Hom_R(D,-)=\Ext_R^i(D,-)$ (it is a nontrivial fact that these agree!). 
\end{example}

\begin{example}
In the case of $G$-modules, let $A$ be an abelian group $A$ with a $G$ action $\phi: G\to \Aut(A)$, (that is, $A$ is a $\ZZ G$-module) with morphisms respecting $G$-action. Let $A^G$ be the group of $x\in A$ such that $g\cdot x=x$ for all $g\in G$. The functor we want in this case is $F(A)=A^G$ which takes $G$ -modules to abelian groups. Note that this functor is the same as $\Hom_{\ZZ G}(\ZZ,-)$, so its (left exact) derived functors are (abstractly) Ext-functors $\Ext_{\ZZ G}^i(\ZZ,-)$, but this doesn't really give us a good way to compute it. These are better known as the ``group cohomology of $G$ with coefficients in $A$'', denoted $H^i(G,A)$. 

How do we compute this? 
It turns out that $\ZZ$ has a very nice ``standard resolution'' (also called the ``bar resolution''), which is given by 
$$F_n=\bigotimes_{i=0}^n \ZZ G.$$
Using this, we get the following recipe: for $G$ a group and $A$ a $G$-module, we let $C^0(G,A)=A$ and $C^n(G,A)$ be the group of $A$-valued maps on $G^n=G\times \cdots\times G$ for $n\geq 1$. These $C^n$ are called the group of \textbf{$n$-cochains} of $G$ with values in $A$. We also define the \textbf{differential operators} $d_n: C^n(G,A)\to C^{n+1}(G,A)$ by 

\begin{align*}
d_n(f)(g_1,\cdots, g_{n+1}):= 
&\ g_1\cdot f(g_2,\cdots, g_{n+1})\\
&+\Sigma_{i=1}^n (-1)^i f(g_1,\cdots, g_{i-1},g_ig_{i+1},g_{i+2},\cdots, g_{n+1})\\
&+(-1)^{n+1}f(g_1,\cdots, g_n).\\
\end{align*}

The cases of most interest are $n=0,1,2$:
\begin{itemize}
\item when $n=0$, we have $f=a\in C^0(G,A)=A$, and then we get 
$$d_0(f)(g)=g\cdot a - a.$$
\item when $n=1$, then $f$ is a function with one input and we have
$$d_1(f)(g_1,g_2)=g_1\cdot f(g_2)-f(g_1g_2)+f(g_1).$$
\item when $n=2$, then $f$ is a function with two inputs and we have
$$d_2(f)(g,h,k)=g\cdot f(h,k) - f(gh,k)+f(g,hk)- f(g,h).$$
Amazingly, people wrote down this formula correctly before the general theory came out.
\end{itemize}

It can be shown that $d_n\circ d_{n+1}=0$. Thus, we can define the group of \textbf{$n$-cocycles} as $Z^n(G,A):=\ker(d_n)$ for $n\geq 0$ and the group of \textbf{$n$-coboundaries} as $B^n(G,A):=\im(d_{n-1})$ for $n\geq 1$ (or $B^0(G,A)=1$ when $n=0$). Then we define the \textbf{$n$-th cohomology group} as
$$H^n(G,A):=Z^n(G,A)/B^n(G,A).$$
Note that $H^0(G,A)=A^G$. 
\end{example}

\begin{example}
Let $(X,\mathcal{O}_X)$ be a ringed space, and $\mathcal{F}$ a $\mathcal{O}_X$-module. Let $\Gamma(-,X)$ be the ``global sections'' functor, $\mathcal{F}\mapsto \mathcal{F}(X)$, from $\mathbb{O}_X$-modules to $R$-modules for $R=\mathcal{O}_X(X)$. This functor is left-exact (to make sense of this we need to make sure we know what exact sequences of sheaves are - defining ``kernels'' is easy but to define ``images'' we need to sheafify). 

Here there is the Grothedieck (right) derived cohomology functor, $R^i\Gamma(-,X)$, which we denote $H^i(X,-)$ (we also call $H^i(X,\mathcal{F})$ the \emph{sheaf cohomology of $X$ with values in $\mathcal{F}$}); in particular, we have $H^0(X,-)=\Gamma(-,X)$. This is given abstractly by the theory earlier in this section; but like group cohomology, we will really need to work with sheaf cohomology, so we need to a way to compute them explicitly.

Here, we have a more concrete one, which is called the \emph{\v{C}ech cohomology}, $\check{H}^i(X,\mathcal{F})$. In general, the two cohomologies are not equal! Fortunately they are equal in the settings we will be interested in. 

The \v{C}ech cohomology is more explicitly computable, and always gives us a map $\phi: H^i(X,\mathcal{F})\to \check{H}^i(X,\mathcal{F})$. We have:

\begin{enumerate}
\item For $i=0,1$, $\phi$ is an isomorphism.
\item (Grothendieck): If $X$ is a Noetherian, separated scheme and $\mathcal{F}$ is a quasi-coherent $\mathcal{O}_X$-module, then $\phi$ is an isomorphism. 
\item (Godement): If $X$ is a paracompact and Hausdorff topological space, then $\phi$ is an isomorphism.
\end{enumerate}

However, $\phi$ is not an isomorphism in general! Here are some counterexamples:

\begin{itemize}
\item In his T\^{o}hoku paper (p.177), Grothendieck provides a counterexample where $H^2\not= \check{H}^2$, for $X=\Aff^2$ with the Zariski topology and $\mathcal{F}$ comes from taking $\underline{\ZZ}$ and modifying it based on a space $Y$ that is a union of two circles. This example is explicit but the proof is somehow deep!
\item A recent paper of Schr\"{o}er (arxiv post 1309.2524) gives a Hausdorff (not not paracompact) topological space constructed from $2$-dimensional discs that is a counterexample. 
\end{itemize}

\subsubsection{\v{C}ech cohomology}
So how do we define \v{C}ech cohomology? The idea is that if $\mathcal{U}$ is an open cover on $X$, the ``nerve'' of $\mathcal{U}$ approximates $X$. We define a $q$-simplex $\sigma$ of $\mathcal{U}$ as an ordered collection of $q+1$ elements in $\mathcal{U}$ with nonempty intersection that we call $|\sigma|$. Suppose $\sigma=(U_i)$ (for $0\leq i \leq q$), we define $\partial_j \sigma:=(U_i)_{i\not= j}$ and then $\partial\sigma=\Sigma_{j=0}^q (-1)^{j+i}\partial_j\sigma$. Note that $|\sigma|=\Cap U_i$. 

We then define \textbf{$q$-cochains} of $\mathcal{U}$ with coefficients in $\mathcal{F}$ to be the set $C^q(\mathcal{U},\mathcal{F})$ of functions $\sigma\mapsto f_\sigma \in \mathcal{F}(|\sigma|)$. Therefore, we get the \textbf{boundary maps} $C^q(\mathcal{U},\mathcal{F})\to C^{q+1}(\mathcal{U},\mathcal{F})$ by 
$$(\delta_q \omega)(\sigma)=\Sigma_{j=0}^{q+1} (-1)^j \text{res}_{|\partial_j \sigma|,|\sigma|} \omega(\partial_j\sigma).$$

One can check that $\delta_{q+1}\circ \delta_q=0$ and hence can define \textbf{cocycles} $Z^q(\mathcal{U},\mathcal{F}):=\ker(\delta_q)$ and \textbf{coboundaries} $B^q(\mathcal{U},\mathcal{F}):=Im(\delta_{q-1})$, and then the \textbf{\v{C}ech cohomology} of $\mathcal{U}$ is given by 
$$\check{H}^q(\mathcal{U},\mathcal{F}):=Z^q(\mathcal{U},\mathcal{F})/B^q(\mathcal{U},\mathcal{F}).$$

So this gives us the cohomology $\check{H}^i(\mathcal{U},\mathcal{F})$ of an open cover $\mathcal{U}$; However, we want a cohomology $\check{H}^i(X,\mathcal{F})$ associated to the whole space $X$! There are two ways to solve this:

\begin{enumerate}
\item If $X$ has a ``good'' cover $\mathcal{U}$ (with all finite intersections of $U_i$ to be contractible), then $\check{H}^i(\mathcal{U},\mathcal{F})$ is canonical.

\item In general, we can define $\check{H}^i(\mathcal{X},\mathcal{F})$ as $\varinjlim_{\mathcal{U}} \check{H}^i(\mathcal{U},\mathcal{F})$. But then we need to make sense of this direct limit. (It might be over an index set that is a proper class?) [fix farbod]
\end{enumerate}

Remark: Let $G$ be a topological group. Let $BG=K(G,1)$ be the \emph{Eilenberg-MacLane space}. (For example, $B\ZZ=S^1$.) Note that $\pi_1=G$ and $\pi_n=0$ for all $n>1$. Then if $A$ is a $G$-module, the sheaf cohomology $H^n(BG,\underline{A})$ (here $\underline{A}$ is the constant sheaf, which is the sheafification of the constant presheaf) is isomorphic to the usual CW complex cohomology $H^n(BG,A)$ and to the group cohomology $H^n(G,A)$.  
\end{example}

\subsection{Back to complex tori (sort of)}
We want to understand line bundles on a locally ringed space $(X,\mathcal{O}_X)$. For now, we can think of it as a complex manifold or a variety. Let $\mathcal{F}$ be a sheaf. We call $\mathcal{F}$ (globally) \textbf{free} if $\mathcal{F}=\bigoplus_{i=1}^r \mathcal{O}_X$ is the direct sum of copies of the structure sheaf; $r$ is called the \textbf{rank} of $\mathcal{F}$. $\mathcal{F}$ is called \textbf{locally free} if there exists an open cover $\{U_i\}$ such that each $\mathcal{F}|_{U_i}$ is free. (There is a correspondence between locally free sheaves $\mathcal{F}$ of rank $n$ and vector bundles of rank $n$. If $\pi:E\to X$ is a vector bundle, we get a locally free sheaf with $\mathcal{F}(U)$ being the sections of $\pi$ over $U$; conversely, if $\mathcal{F}$ is locally free, we can construct an associated line bundle as $\coprod U_i\times \CC^n$ modulo gluing data.) 

\subsubsection{Line bundles}

Line bundles are vector bundles of rank $1$ (equivalently, locally free sheaves of rank $1$). Our goal is to give a cohomological interpretation on the set of line bundles. Let $\pi:L\to X$ be a line bundle. Let $\{U_\alpha\}$ be an open cover with trivializations given by (holomorphic) $\phi_\alpha: L|_\alpha\isomto U_\alpha\times \CC$ where $L|_\alpha:=\pi^{-1}[U_\alpha]$. Define the \textbf{transition functions} for $L$ with respect to $\{\phi_\alpha\}$ as $g_{\alpha\beta}:U_\alpha\cap U_\beta \to \CC^*$ which are given by 
$$g_{\alpha\beta}(z):=\phi_\alpha\circ \phi_\beta^{-1}|_{L_z}$$
where $L_z:=\{z\}\times \CC$. By itself this is a linear (and hence holomorphic) map on $\{z\}\times \CC$, which is determined by the complex number we are calling $g_{\alpha\beta}(z)$. We check that 
$g_{\alpha\beta}\circ g_{\beta\alpha}=1$ (so it is nonzero) and also $g_{\alpha\beta}\circ g_{\beta\gamma}\circ g_{\gamma\alpha}=1$. Rewriting this latter condition gives a cocycle condition
$$g_{\alpha\beta}g_{\gamma\beta}^{-1}g_{\gamma\alpha}=1.$$

To summarize, a line bundle (trivialized by an open cover $\mathcal{U}$) determines a collection of 
\begin{itemize}
\item holomorphic functions $g_{\alpha\beta}:U_\alpha\cap U_\beta\to \CC^\times$; which satisfy 
\item $g_{\alpha\beta}\circ g_{\beta\alpha}=1$, and
\item $g_{\alpha\beta}g_{\gamma\beta}^{-1}g_{\gamma\alpha}=1$.
\end{itemize}

Conversely, if we are given $\{g_{\alpha\beta}\}$ satisfying these properties, then we can construct a line bundle $L$ with transition functions $\{g_{\alpha\beta}\}$ as a quotient of $\coprod U_\alpha\times \CC$ by the appropriate gluing relations. 

\subsubsection{Choices}
If $f_\alpha\in\mathcal{O}^\times(U_\alpha)$ is a nonvanishing holomorphic function on $U_\alpha$, and we construct new trivializations $\phi'_\alpha=f_\alpha\circ \phi_\alpha$, then the new transition functions $g_{\alpha\beta}'=\frac{f_\alpha}{f_\beta}g_{\alpha\beta}$
give the same bundle $L$. 

Now, the collection of $\{g_{\alpha\beta}\in \mathcal{O}^\times (U_\alpha\cap U_\beta)\}$ is a \v{C}ech $1$-cochain. The conditions we wrote down that it satisfies implies that it is actually a $1$-cocycles. Moreover, the ambiguity mentioned above is exactly the $1$-coboundaries. Therefore, we can then conclude the set of (isomorphism classes of) line bundles $\Pic(X)$ is isomorphic to $H^1(X,\mathcal{O}_X^*)$. Moreover, this is a group homomorphism; for the group structure on line bundles coming from tensor product and the group structure naturally showing up on $H^1(X,\mathcal{O}_X^\times)$. 

\subsubsection{Line bundles vs ``factors of automorphy''}
Let $X$ be a complex torus, and $\tilde{X}=\CC^g$ its universal cover, with $\pi:\tilde{X}\to X$ the covering map. 
\begin{theorem}
There exists a canonical exact sequence 
$$0\to H^1(\pi(X),H^0(\tilde{X},\mathcal{O}_{\tilde{X}}^\times))\overset{\phi}{\to} H^1(X,\mathcal{O}_X^\times)\to H^1(\tilde{X},\mathcal{O}_{\tilde{X}}^\times).$$
\end{theorem}

The first term in the above exact sequence is the group of \emph{``factors of automorphy''} and the latter two are Picard groups as above.

\begin{corollary}
If $\pi^*(L)$ is trivial, then it is completely described by a ``factor of automorphy''.
\end{corollary}

\begin{corollary}
If $\Pic(\tilde{X})=0$ (eg when $\tilde{X}=\CC^g$), then $\phi$ is an isomorphism.
\end{corollary}

